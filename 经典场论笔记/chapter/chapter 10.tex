\chapter{Yang-Mills理论与't Hooft-Polyakov磁单极}

Yang-Mills理论即非Abel规范场论,而't Hooft-Polyakov磁单极是一类非Abel规范场的孤子解

\section{Yang-Mills理论}

$U(1)$整体对称性局域化得到电磁场理论,$U(N)$整体对称性局域化得到Yang-Mills理论,是非线性的

$SU(N)$群的元素$U$可以写成
\begin{gather*}
    U=\e^{\i \theta^a T_a}
    \\
    \intertext{Hermite矩阵$T_a$满足}
    \mathrm{Tr}(T_a)=0
    \\
    \intertext{归一化条件}
    \mathrm{Tr}(T_a T_b)=\frac{1}{2} \delta_{ab}
\end{gather*}
独立Hermite矩阵又$N^2-1$个
\begin{gather*}
    \mathrm{Tr}([T_a,T_b])=0
    \\
    \implies [-\i T_a, -\i T_b]=C_{abc} (-\i T_c)
    \\
    C_{abc}=-2 \i \mathrm{Tr}([T_a, T_b] T_c)
\end{gather*}

\begin{itemize}
    \item 称物质场取$SU(N)$规范群的基础表示,若物质场为$N$维复标量场$\Phi$,局域$SU(N)$规范变换
    \begin{gather*}
        \Phi \to \Phi'=U(x) \Phi=\e^{\i \epsilon_a(x) T_a} \Phi
    \end{gather*}
    协变导数
    \begin{gather*}
        \D_{\mu}=\partial_{\mu}+A_{\mu},\quad A_{\mu}=-\i T_a A_{\mu}^a
    \end{gather*}
    规范势$A_{\mu}$变换规则
    \begin{gather*}
        A_{\mu} \to A'_{\mu}=U A_{\mu} U^{-1}+U \partial_{\mu} U^{-1}
    \end{gather*}
    当规范变换参数$\epsilon_a(x)$为无穷小量时,
    \begin{gather*}
        \delta A_{\mu}=A'_{\mu}-A_{\mu}=\partial_{\mu} \varepsilon+[A_{\mu},\varepsilon],\quad \varepsilon=-\i T_a \epsilon_a(x)
    \end{gather*}
    定义规范场强
    \begin{gather*}
        F_{\mu \nu}=-\i T_a F_{\mu \nu}^a=[\D_{\mu}, \D_{\nu}]=\partial_{\mu} A_{\nu} - \partial_{\nu} A_{\mu}+[A_{\mu}, A_{\nu}]
        \\
        F_{\mu \nu}^c=\partial_{\mu} A_{\nu}^c - \partial_{\nu} A_{\mu}^c + C_{abc} A_{\mu}^a A_{\nu}^b
        \\
        F=\d A+A \wedge A
        \\
        \intertext{变换规则}
        F_{\mu \nu} \to F'_{\mu \nu} = U F_{\mu \nu} U^{-1}
    \end{gather*}
    \item 称物质场取$SU(N)$规范群的伴随表示,若物质场为$N \times N$矩阵$\phi$
    \begin{gather*}
        \phi(x)=-\i T_a \phi^a(x)
        \\
        \intertext{变换规则}
        \phi(x) \to U(x) \phi(x) U^{-1}(x)
        \\
        \intertext{协变导数}
        \D_{\mu} \phi=\partial_{\mu} \phi + [A_{\mu}, \phi]
    \end{gather*}
    \end{itemize}

规范场作用量取为
\begin{gather*}
    S_g=a \int \d^4 x \frac{1}{2} \mathrm{Tr}(F_{\mu \nu} F^{\mu \nu}) + b \int \d^4 x \mathrm{Tr}(F \wedge F)
\end{gather*}
$a=\frac{1}{e^2}$,$e$为规范场耦合常数,注意到
\begin{gather*}
    \mathrm{Tr}(F \wedge F)= \d \mathrm{Tr}(A \wedge \d A + \frac{2}{3} A \wedge A \wedge A)
\end{gather*}
第二项积分无贡献
\begin{gather*}
    S_g=\int \d^4 x \frac{1}{2e^2} \mathrm{Tr}(F_{\mu \nu} F^{\mu \nu})=-\int \d^4 x \frac{1}{4e^2} F^a_{\mu \nu} F^{a \mu \nu}
\end{gather*}
对$A_{\mu}$变分
\begin{gather*}
    \delta F_{\mu \nu} = \D_{\mu} \delta A_{\nu} - \D_{\nu} \delta A_{\mu}
    \\
    \D_{\mu} \delta A_{\nu}=\partial_{\mu} \delta A_{\nu} + [A_{\mu}, \delta A_{\nu}]
    \\
    \partial_{\mu} \mathrm{Tr}(A_1 A_2)=\mathrm{Tr}((\D_{\mu} A_1) A_2)+\mathrm{Tr}(A_1 \D_{\mu} A_2)
    \\
    \intertext{分部积分得到}
    \delta S_g=\frac{2}{e^2} \int \d^4 x \mathrm{Tr}(-\D_{\mu} F^{\mu \nu} \delta A_{\nu})=\frac{1}{e^2} \int \d^4 x (\D_{\mu} F^{\mu \nu})^a \delta A_{\nu}^a
\end{gather*}

\begin{itemize}
    \item 若物质场取基础表示,作用量
    \begin{gather*}
        S_m=-\int \d^4 x \left[(\D_{\mu} \Phi)^{\dagger} \D^{\mu} \Phi+ \mathscr{U}(\Phi^{\dagger} \Phi)\right]
    \end{gather*}
    对$A_{\mu}^a$变分
    \begin{gather*}
        \delta S_m=\i \int \d^4 x \left[(\D^{\mu} \Phi)^{\dagger} T_a \Phi-\Phi^{\dagger} T_a \D^{\mu} \Phi\right] \delta A_{\mu}^a
    \end{gather*}
    得到规范场场方程
    \begin{gather*}
        -\frac{1}{e^2} (\D_{\mu} F^{\mu \nu})^a=\i \left[(\D^{\mu} \Phi)^{\dagger} T_a \Phi-\Phi^{\dagger} T_a \D^{\mu} \Phi\right]
    \end{gather*}
    对$\Phi$变分得到物质场场方程
    \begin{gather*}
        \D_{\mu} \D^{\mu} \Phi=\pt{\mathscr{U}}{\Phi^{\dagger}}
        \\
        \D_{\mu} \D^{\mu} \Phi^{\dagger}=\pt{\mathscr{U}}{\Phi}
    \end{gather*}、
    \item 若物质场取伴随表示,作用量
    \begin{gather*}
        S_m=\int \d^4 \left[\frac{1}{e^2} \mathrm{Tr}(\D_{\mu} \phi \D^{\mu} \phi)-\mathscr{U}(\phi^a \phi^a)\right]
        \\
        \delta S_m=\frac{2}{e^2} \int \d^4 \mathrm{Tr}([\phi, \D^{\mu} \phi] \delta A_{\mu}) + \int \d^4 \left[\frac{1}{e^2} (\D_{\mu} \D^{\mu} \phi)^a-\pt{\mathscr{U}}{\phi^a}\right] \delta \phi^a
        \\
        \intertext{得到规范场场方程和物质场场方程}
        \D_{\mu} F^{\mu \nu}=[\phi, \D^{\nu} \phi]
        \\
        \frac{1}{e^2} (\D_{\mu} \D^{\mu} \phi)^a=\pt{\mathscr{U}}{\phi^a}
    \end{gather*}
\end{itemize}

\section{'t Hooft-Polyakov磁单极}

考虑具有$SU(2)$规范对称性的系统,作用量
\begin{gather*}
    S=\int \d^4 x \left[\frac{1}{2e^2} \mathrm{Tr}(F_{\mu \nu} F^{\mu \nu})+\frac{1}{e^2} \mathrm{Tr}(\D_{\mu} \phi \D^{\mu} \phi) - \frac{\lambda}{4}(|\phi|^2-\nu^2)^2 \right]
    \\
    \intertext{能动量张量}
    T^{\mu \nu}=\frac{1}{e^2} \left[{F^{a\mu}}_{\rho} F^{a\nu \rho}-\frac{1}{4} \eta^{\mu \nu} F_{\rho \sigma}^a F^{a \rho \sigma}\right]+\frac{1}{e^2} (\D^{\mu} \phi)^a (\D^{\nu} \phi)^a + \eta^{\mu \nu} \left[\frac{1}{e^2} \mathrm{Tr}(\D_{\mu} \phi \D^{\mu} \phi) - \frac{\lambda}{4}(|\phi|^2-\nu^2)^2\right]
    \\
    \mathscr{H}=T^{00}=\frac{1}{2e^2} \left[\bm{E}^a \cdot \bm{E}^a + \bm{B}^a \cdot \bm{B}^a + (\D_0 \phi)^a (\D_0 \phi)^a + (\D_i \phi)^a (\D_i \phi)^a\right] + \mathscr{U}(|\phi|)
    \\
    \mathscr{U}(|\phi|)=\frac{\lambda}{4}(|\phi|^2-\nu^2)^2,\quad E_i^a=F^{a0i},\quad B_i^a=\frac{1}{2} \varepsilon_{ijk} F^{ajk}
\end{gather*}
$\phi$称为Higgs场

\begin{itemize}
    \item 真空场位形
    \begin{gather*}
        \mathscr{H}=0 \iff F^{a \mu \nu}=\D^{\mu} \phi=\mathscr{U}(|\phi|)=0
        \\
        \intertext{真空解$\phi$为常数}
        |\phi|^2=\nu^2
        \\
        \intertext{在$U(1)$变换下,真空解保持不变}
        \e^{-\alpha \frac{\phi}{\nu}} \phi \e^{\alpha \frac{\phi}{\nu}} = \phi
    \end{gather*}
    在$SU(2)$规范变换下,真空解不能保持不变,形成一个解的等价类
    \item 有限能量解\\
    定义Higgs真空
    \begin{gather*}
        \mathscr{M}_{H}=\{\phi \vert \mathscr{U}(|\phi|)=0\}=S^2
    \end{gather*}
    对于有限能量解,在无穷远处场位形趋于Higgs真空,$\phi$构成了两者之间的映射
    \begin{gather*}
        \phi:S_{\infty}^2 \to \mathscr{M}_H=S^2
    \end{gather*}
    根据覆盖$S^2$的次数,可以进行拓扑学分类,称为$S^2$的第二同伦群$\pi_2(S^2)=\mathbb{Z}$,覆盖次数
    \begin{gather*}
        n=\frac{1}{8\pi \nu^3} \int_{S_{\infty}^2} \varepsilon_{abc} \phi^a \d \phi^b \wedge \d \phi^c
    \end{gather*}
    在无穷远处,同样有$U(1)$对称性
    \begin{gather*}
        U(1)=\e^{-\alpha \frac{\phi}{\nu}}
    \end{gather*}
    覆盖$n \neq 0$时,$A_{\mu}$必须取非0场位形,总能量有限,
    \begin{gather*}
        (D_{\mu} \phi)^a=\partial_{\mu} \phi^a + \varepsilon_{abc} A_{\mu}^b \phi^c \sim 0
        \\
        \intertext{无穷远处}
        \phi^a \phi^a \sim \nu^2, \quad \phi^a \partial_{\mu} \phi^a \sim 0
        \\
        A_{\mu}^a \sim \frac{1}{\nu^2} \varepsilon_{abc} \phi^b \partial_{\mu} \phi^c + \frac{1}{\nu} \phi^a a_{\mu}(x)
        \\
        \mathscr{F}_{\mu \nu}=\frac{\phi^a}{\nu} F_{\mu \nu}^a \sim \partial_{\mu} a_{\nu} - \partial_{\nu} a_{\mu} + \frac{1}{\nu^3} \varepsilon_{abc} \phi^a \partial_{\mu} \phi^b \partial_{\nu} \phi^c
        \\
        \intertext{最后一项额外项带来磁荷}
        g=\int_{S_{\infty}^2} \mathscr{F}=4 \pi n
    \end{gather*}
    $\mathscr{F}^{\mu \nu}$满足无源Maxwell方程组,从无穷远处的$U(1)$对称性看来,这样的解为磁单极子,称为't Hooft-Polyakov磁单极
    \item 求解磁单极解\\
    在$n=1$时才有静态解,考虑$n=1$情形,选取$A_0=D_0 \phi=E_i^a=0$
    \begin{gather*}
        \mathscr{H}=\frac{1}{2e^2} \left[\bm{B}^a \cdot \bm{B}^a+(\D_i \phi)^a (\D_i \phi)^a\right] + \mathscr{U}(|\phi|)
        \\
        \intertext{Higgs场取如下形式}
        \phi^a=\frac{x^a}{r^2} h(r)
        \\
        h(r)=
        \begin{cases}
            0,\quad r \to 0
            \\
            \nu r,\quad r \to \infty
        \end{cases}
        \\
        \intertext{相应的规范场取如下形式}
        A_i^a=-\varepsilon_{aij} \frac{x^j}{r^2}(1-k(r))
        \\
        k(r)=
        \begin{cases}
            0,\quad r \to 0
            \\
            1,\quad r \to \infty
        \end{cases}
    \end{gather*}
    代入场方程即可求解
    \item Bogomolnyi能限
    \begin{gather*}
        \intertext{能量密度}
        \mathscr{H}=\frac{1}{2e^2} \left[(E_i^a-\D_i \phi^a \sin \theta)^2+(B_i^a - \D_i \phi^a \cos \theta)^2+(\D_0 \phi)^2\right]+\mathscr{U}(|\phi|)+\frac{1}{e^2} \left[E_i^a \D_i \phi^a \sin \theta + B_i^a \D_i \phi^a \cos \theta\right]
    \end{gather*}
    注意到$\D_i B_i=0$
    \begin{gather*}
        \frac{1}{\nu} \int \d^3 \bm{x} B_i^a \D_i \phi^a=\frac{1}{\nu} \int_{S_{\infty}^2} (\bm{B}^a \phi^a) \cdot \d \bm{S}=g
        \\
        \frac{1}{\nu} \int \d^3 \bm{x} E_i^a \D_i \phi^a=\frac{1}{\nu} \int_{S_{\infty}^2} (\bm{E}^a \phi^a) \cdot \d \bm{S}=q
    \end{gather*}
    总能量
    \begin{align*}
        \mathscr{E}&=\int \d^3 \bm{x} \mathscr{H}
        \\
        &=\frac{\nu}{e^2} \left[q \sin \theta + g \cos \theta\right]+\int \d^3 \bm{x}\left\{\frac{1}{2e^2}\left[(E_i^a-\D_i \phi^a \sin \theta)^2+(B_i^a - \D_i \phi^a \cos \theta)^2+(\D_0 \phi)^2\right] + \mathscr{U}(|\phi|) \right\}
        \\
        & \geq \frac{\nu}{e^2} \left[q \sin \theta + g \cos \theta\right]
        \\
        & \geq \frac{\nu}{e^2} \sqrt{q^2+g^2}
    \end{align*}
    在Bogomolnyi能限下有如下BPS方程
    \begin{gather*}
        E_i^a=\D_i \phi^a \sin \theta_m,\quad B_i^a=\D_i \phi^a \cos \theta_m,\quad \D_0 \phi^a=0
        \\
        \tan \theta_m=\frac{q}{g}
    \end{gather*}
    对于't Hooft-Polyakov磁单极,$q=\theta_m=0$,BPS方程化为
    \begin{gather*}
        E_i^a=0,\quad \D_0 \phi^a=0,\quad B_i^a=D_i \phi^a
    \end{gather*}
    对于$n=1$的磁单极子,解得
    \begin{gather*}
        h(r)=\nu r \coth(\nu r)-1,\quad k(r)=\frac{\nu r}{\sinh(\nu r)}
    \end{gather*}
\end{itemize}