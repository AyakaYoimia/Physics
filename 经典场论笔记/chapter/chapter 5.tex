\chapter{超导的有效理论以及涡旋解}

\section{Ginzburg-Landau理论}

Ginzburg-Landau理论原始版本是非相对论理论,在静态场位形下无需区分非相对论与相对论,考虑相对论性的复标量场与电磁场耦合系统
\begin{gather*}
    S=\int \d^4 x \left[-\frac{1}{4} F_{\mu \nu} F^{\mu \nu}-\overline{\D_\mu \phi} \D^\mu \phi-\frac{g}{2}(|\phi|^2-u)^2\right] \quad u>0
    \\
    T^{\mu \nu}={F^{\mu}}_{\rho} F^{\nu \rho}-\frac{1}{4} \eta^{\mu \nu} F_{\rho \sigma} F^{\rho \sigma}+\overline{\D^{\mu} \phi} \D^{\nu} \phi+\D^{\mu} \phi \overline{\D^{\nu} \phi}-\eta^{\mu \nu} \left[\overline{\D_\rho \phi} \D^\rho \phi+\frac{g}{2} (|\phi|^2-u)^2\right]
    \\
    \intertext{能量密度}
    \mathscr{H}=T^{00}=\frac{1}{2}(\bm{E}^2+\bm{B}^2)+|\D_0 \phi|^2+\displaystyle \sum_{i=1}^{3}|\D_i \phi|^2+\frac{g}{2} (|\phi|^2-u)^2
    \\
    \intertext{选取轴规范}
    A^0=0
    \\
    \intertext{静态场位形下,$\bm{E}=0,\D_0 \phi=0$,Ginzburg-Landau能量泛函}
    H=\int \d^3 \bm{x} \left[\frac{1}{2}\bm{B}^2+\displaystyle \sum_{i=1}^{3}|\D_i \phi|^2+\frac{g}{2} (|\phi|^2-u)^2\right]
    \\
    \intertext{最小作用量原理}
    \delta H=0
\end{gather*}
$\phi$称为序参量,耦合常数$e=\frac{q_{cp}}{\hbar}$,$q_{cp}$为Cooper对的电荷量
\begin{gather*}
    \intertext{记除磁能之外的能量为}
    H_m=\int \d^3 \bm{x} \left[+\displaystyle \sum_{i=1}^{3}|\D_i \phi|^2+\frac{g}{2} (|\phi|^2-u)^2\right]
    \\
    \delta H_m=-\int \d^3 \bm{x} \bm{J}_A \cdot \delta \bm{A}
    \\
    J_A^i=\i e(\phi \overline{\D_i \phi}-\overline{\phi} \D_i \phi)
    \\
    \intertext{设}
    \phi(x)=|\phi| \e^{\i \theta(x)}
    \\
    H_m=\int \d^3 \bm{x}\left[(\partial_i |\phi|)^2+|\phi|^2 (\partial_i \theta-e A_i)^2+\frac{g}{2}(|\phi|^2-u)^2\right]
    \\
    \bm{J}_A=2e|\phi|^2(\nabla \theta-e\bm{A})
    \\
    \delta H=-\int \d^3 \bm{x} J_A^i \delta A_i=\frac{1}{e} \int \d^3 \bm{x} J_A^i \delta (\partial_i \theta)=-\frac{1}{e} \int \d^3 \bm{x} (\partial_i J_A^i) (\delta \theta)=0
\end{gather*}
固定$|\phi|$
\begin{align*}
    \delta H &=\int \d^3 \bm{x} \left[B_i \delta B^i+\frac{1}{e} J_A^i \delta(\partial_i \theta)-J_A^i \delta A_i\right]
    \\
    &=\int \d^3 \bm{x} \left[B_i \epsilon^{ijk} \delta (\partial_j A_k)-J_A^i \delta A_i+\frac{1}{e} J_A^i \delta(\partial_i \theta)\right]
    \\
    &=\int \d^3 \bm{x} \left[-\epsilon^{ijk} (\partial_j B_i) \delta A_k - J_A^i \delta A_i+\frac{1}{e} J_A^i \delta(\partial_i \theta)\right]
    \\
    &=\int \d^3 \bm{x} \left[(\epsilon^{ijk} \partial_j B_k-J_A^i) \delta A_i+\frac{1}{e} J_A^i \delta(\partial_i \theta)\right]
    \\
    \implies & \quad \nabla \cdot \bm{J}_A=0,\quad \nabla \times \bm{B}=\bm{J}_A
\end{align*}

系统能量最小时,
\begin{gather*}
    |\phi|=\sqrt{u},\quad \partial_i \theta-e A_i=0
    \\
    \bm{J}_A=0,\quad \bm{B}=\nabla \times \bm{A}=\frac{1}{e} \nabla \times \nabla \theta=0
\end{gather*}
此时称为超导相,而$|\phi|=0,\partial_i \theta \neq e A_i$的相称为正常相

\begin{enumerate}
    \item 对相位场和磁矢势场微扰
    \begin{gather*}
        \nabla \cdot \delta \bm{J}_A=0,\quad \nabla \times \delta \bm{B}=\delta \bm{J}_A
        \\
        \implies 
        \begin{cases}
            \nabla^2 \delta \theta=e\nabla \cdot \delta \bm{A}
            \\
            \nabla \times (\nabla \times \delta \bm{A})=2eu(\nabla \delta \theta-e\delta \bm{A})
        \end{cases}
        \\
        \intertext{选取规范}
        \nabla \cdot \delta \bm{A}=2eu \delta \theta
        \\
        \begin{cases}
            \nabla^2 \delta \theta=2e^2 u\delta \theta
            \\
            \nabla^2 \delta \bm{A}=2e^2 u\delta \bm{A}
        \end{cases}
    \end{gather*}
    解为指数衰减,特征长度为穿透深度
    \begin{gather*}
        \lambda=\frac{1}{\sqrt{2e^2 u}}
    \end{gather*}
    \item 对模长场微扰
    \begin{gather*}
        |\phi|=\sqrt{u}+\rho
        \\
        H_m \approx \int \d^3 \bm{x} \left[(\partial_i \rho)^2+2gu\rho^2\right]
        \\
        \implies \nabla^2 \rho=2gu \rho
        \\
        \intertext{解为指数衰减,特征长度为关联长度}
        \xi=\frac{1}{\sqrt{2gu}}
    \end{gather*}
    \item 引入第三个参数
    \begin{gather*}
        \kappa=\frac{\lambda}{\xi}=\sqrt{\frac{g}{e^2}}
    \end{gather*}
    $\kappa<1$为第一类超导体,$\kappa>1$为第二类超导体
\end{enumerate}

正常相变为超导相能量密度减少
\begin{gather*}
    \Delta=\frac{1}{2}gu^2=\frac{1}{8e^2\lambda^2 \xi^2}
\end{gather*}
若正常相下存在外磁场$B$,当$\frac{1}{2} B^2 < \Delta$时系统才会倾向于处于超导相

\section{涡旋解}

超导体中,可能存在管状区域,中心处于正常态,超导体内部绕管状区域的闭合回路上
\begin{gather*}
    \Phi=\oint_L \bm{A} \cdot \d \bm{l}=\frac{1}{e} \oint_L \nabla \theta \d \bm{l}=\frac{\Delta \theta}{e}=\frac{2N\pi}{e}
    \\
    \intertext{得到磁通量子化,量子化单位为}
    \Phi_0=\frac{2\pi}{e}
\end{gather*}

Bogomolnyi证明了仅对于第二类超导体,通量$N\Phi_0$的涡旋线在分解为$N$个通量为$\Phi_0$的涡旋线时更加稳定。设单位横截面积涡旋线数密度为$n$,在半径$\xi$的管状区域内接近正常态,为了使管状区域不重叠,限制$n<\frac{1}{\pi \xi^2}$,产生涡旋线单位体积需要能量$n \pi \xi^2 \Delta$,
\begin{gather*}
    \intertext{单位体积涡旋态能量}
    W_V=\begin{cases}
        n\pi \xi^2 \Delta+\frac{1}{2} B^2(1-n\pi \lambda^2),\quad n<\frac{1}{\pi \lambda^2}
        \\
        n\pi \xi^2 \Delta,\quad n>\frac{1}{\pi \lambda^2}
    \end{cases}
    \\
    \intertext{正常态单位体积比超导态多出的能量}
    W_N \approx \Delta
    \\
    \intertext{超导态为了排除磁场,单位体积需要能量}
    W_S=\frac{1}{2} B^2
\end{gather*}
\begin{itemize}
    \item 对于第一类超导体,$n<\frac{1}{\pi \xi^2}<\frac{1}{\pi \lambda^2}$
    \begin{itemize}
        \item 当$B<\sqrt{2\Delta}$时,$W_V>\frac{1}{2}B^2+n\pi(\xi^2-\lambda^2)\Delta>W_S$
        \item 当$B>\sqrt{2\Delta}$时,$W_V>\Delta[1+n\pi(\xi^2-\lambda^2)]>W_N$
    \end{itemize}
    不存在涡旋态
    \item 对于第二类超导体,磁场分为三个区域,临界值
    \begin{gather*}
        B_{c1}=\sqrt{2\Delta} \frac{\xi}{\lambda},\quad B_{c2}=\sqrt{2\Delta} \frac{\lambda}{\xi}
    \end{gather*}
    根据Bogomolnyi证明的结论,$n=\frac{B}{\Phi_0}$
    \begin{itemize}
        \item 当$B<B_{c1}$时,$n<\frac{B_{c1}}{\Phi_0}=\frac{1}{4\pi \lambda^2} \sim \frac{1}{\pi \lambda^2}$
        \begin{gather*}
            W_V = n\pi \xi^2 \Delta+\frac{1}{2} B^2(1-n\pi \lambda^2)=\frac{1}{2} B^2+n\pi(\xi^2 \Delta-\frac{1}{2} B^2 \lambda^2)>\frac{1}{2}B^2=W_S
            \\
            W_N=\Delta=\frac{1}{2} B_{c1}^2 \frac{\lambda^2}{\xi^2}>\frac{1}{2} B^2=W_S
        \end{gather*}
        系统处于超导态
        \item 当$B>B_{c1}$时,$n>\frac{B_{c1}}{\Phi_0}=\frac{1}{4\pi \lambda^2} \sim \frac{1}{\pi \lambda^2}$
        \begin{gather*}
            W_V = n\pi \xi^2 \Delta =\frac{1}{2}eB\xi^2 \Delta \sim \frac{B}{B_{c2}}W_N \sim \frac{B_{c1}}{B} W_S
        \end{gather*}
        \begin{itemize}
            \item 当$B_{c1}<B<B_{c2}$时,$W_V<W_N,W_V<W_S$,系统处于涡旋态
            \item 当$B>B_{c2}$时,$W_V<W_S,W_V>W_N$,系统处于正常态
        \end{itemize}
    \end{itemize}
\end{itemize}

考虑在$x_3$方向上具有对称性的涡旋解,只考虑第二类超导体,
\begin{gather*}
    \intertext{静态场位形能量}
    H=\int \d^2 x \left[\frac{1}{2} B^2+\displaystyle \sum_{i=1}^{2} |\D_i \phi|^2 + \frac{g}{2} (|\phi|^2-u)^2\right]
\end{gather*}
\begin{align*}
    |\D_1 \phi|^2+|\D_2 \phi|^2&=|(\D_1 \pm \i \D_2)\phi|^2 \mp \i \overline{\D_1 \phi} \D_2 \phi \pm \i \overline{\D_2 \phi} \D_1 \phi
    \\
    &=|(\D_1 \pm \i \D_2)\phi|^2 \mp \i \epsilon^{ij} \overline{\D_i \phi} \D_j \phi
    \\
    \mp \i \epsilon^{ij} \overline{\D_i \phi} \D_j \phi &=\mp \i \epsilon^{ij} (\partial_i \overline{\phi})\D_j \phi \pm \epsilon^{ij} e A_i \overline{\phi} \D_j \phi
    \\
    &=\mp \i \epsilon^{ij} \partial_i (\overline{\phi}\D_j \phi) \pm \i \epsilon^{ij} \overline{\phi} (\partial_i \D_j \phi) \pm \epsilon^{ij} e A_i \overline{\phi} \D_j \phi
    \\
    &=\mp \i \epsilon^{ij} \partial_i (\overline{\phi}\D_j \phi) \pm \i \epsilon^{ij} \overline{\phi} (\D_i \D_j \phi)
    \\
    \epsilon^{ij} \D_i \D_j&=\frac{1}{2} \epsilon^{ij} [\D_i,\D_j]=-\i e B
\end{align*}
\begin{align*}
    H=&\int \d^2 x \left[|(\D_1 \pm \i \D_2)\phi|^2 \pm eB(|\phi|^2-u)+\frac{1}{2}B^2+\frac{g}{2}(|\phi|^2-u)^2\right]
    \\
    &+\int \d^2 x \left[\pm euB \mp \i \epsilon^{ij} \partial_i(\overline{\phi} \D_j \phi)\right]
    \\
    =&\int \d^2 x \left[|(\D_1 \pm \i \D_2)\phi|^2+\frac{1}{2}\left(B\pm e(|\phi^2-u)^2\right)^2+\frac{g-e^2}{2}(|\phi|^2-u)^2\right]
    \\
    &+\int \d^2 x \left[\pm euB \mp \i \epsilon^{ij} \partial_i(\overline{\phi} \D_j \phi)\right]
\end{align*}
\begin{gather*}
    \int \d^2 x \epsilon^{ij} \partial_i(\overline{\phi} \D_j \phi)=\int \partial_i(\overline{\phi} \D_j \phi) \d x^i \wedge \d x^j=\int \d (\overline{\phi} \D_j \phi \d x^j)=\oint \overline{\phi} \D_j \phi \d x^j
    \\
    \intertext{无穷远处,$|\phi| \to \sqrt{u},\D_i \phi \to 0$}
    \int \d^2 x \epsilon^{ij} \partial_i(\overline{\phi} \D_j \phi)=0
    \\
    H=\int \d^2 x \left[|(\D_1 \pm \i \D_2)\phi|^2+\frac{1}{2}\left(B\pm e(|\phi^2-u)^2\right)^2+\frac{g-e^2}{2}(|\phi|^2-u)^2\right] \pm 2\pi N u
    \\
    \intertext{Bogomolny能限}
    H \geq 2 \pi |N| u
\end{gather*}

对于临界超导体$g=e^2$
\begin{gather*}
    H=\int \d^2 x \left[|(\D_1 \pm \i \D_2)\phi|^2+\frac{1}{2}\left(B\pm e(|\phi^2-u)^2\right)^2\right] \pm 2\pi N u
    \\
    \intertext{能量最小时满足BPS方程}
    (\D_1 + \i \D_2) \phi=0,\quad B+e(|\phi|^2-u)=0
\end{gather*}
引入极坐标$(r,\alpha)$,在无穷远处$\theta$变化$N$圈,可取$\theta(x)=N\alpha$,假设$|\phi|$只与$r$有关,得到
\begin{align*}
    A_i&=\frac{1}{e}\left[\epsilon_{ij} \partial_j \ln |\phi|(r)+\partial_i \theta\right]
    \\
    B&=\epsilon_{ij} \partial_i A_j
    \\
    &=\frac{1}{e}\epsilon_{ij} \epsilon_{jk} \partial_i \partial_k \ln |\phi|+\frac{1}{e} \epsilon_{ij} \partial_i \partial_j \theta
    \\
    &=-\frac{1}{e} \partial_i^2 \ln |\phi|+\frac{1}{e} \epsilon_{ij} \partial_i \partial_j \theta
    \\
    \intertext{由$\epsilon_{ij} \partial_i \partial_j \alpha=2\pi \delta^2(x)$得}
    B&=-\frac{1}{e} \partial_i^2 \ln |\phi|+\frac{1}{e} 2\pi N \delta^2(x)
\end{align*}
\begin{gather*}
    \intertext{代入另一个方程得到}
    \nabla^2 \ln |\phi|(r)=2\pi N \delta^2(x)+e^2(|\phi|^2-u)
\end{gather*}
\begin{itemize}
    \item $r \to \infty$
    \begin{gather*}
        |\phi| \to \sqrt{u},\quad A_i \to \frac{1}{e} \partial_i \theta
    \end{gather*}
    \item $r \to 0$
    \begin{gather*}
        \nabla^2 \ln |\phi|(r)=2\pi N \delta^2(x) \implies |\phi| \to A r^N
        \\
        B \to eu
    \end{gather*}
\end{itemize}