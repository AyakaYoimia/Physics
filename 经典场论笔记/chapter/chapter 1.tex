\chapter{初识经典场论}

狭义相对论相互作用有上限$\implies$近距作用$\implies$场

\section{从粒子到场}

\begin{gather*}
    \intertext{对于$n$自由度粒子}
    \begin{cases}
        S(q_i(t),p_i(t))=\int \d t \left[\displaystyle \sum_{i=1}^{n} p_i \dot{q}_i-H(q_i,p_i)\right]
        \\
        \dot{q}_i=\pt{H}{p_i},\quad \dot{p}_i=-\pt{H}{q_i}
        \\
        [A,B]=\displaystyle \sum_{i=1}^{n} \left(\pt{A}{q_i} \pt{B}{p_i} - \pt{A}{p_i} \pt{B}{q_i}\right)
    \end{cases}
    \\
    \intertext{连续化指标,$S$和$H$成为泛函}
    \begin{cases}
        S[q_\sigma(t),p_\sigma(t)]=\int \d t \left[\int \d \sigma p_\sigma \dot{q}_\sigma-H[q_\sigma,p_\sigma]\right]
        \\
        \dot{q}_\sigma=\frac{\delta H}{\delta p_\sigma},\quad \dot{p}_\sigma=-\frac{\delta H}{\delta q_\sigma}
        \\
        [A,B]=\int \d \sigma \left(\frac{\delta A}{\delta q_\sigma} \frac{\delta B}{\delta p_\sigma} - \frac{\delta A}{\delta p_\sigma} \frac{\delta B}{\delta q_\sigma}\right)
    \end{cases}
    \\
    \intertext{改写得到标量场}
    \phi(\sigma,t)=q_\sigma(t),\quad \pi(\sigma,t)=p_\sigma(t)
    \\
    \begin{cases}
        S[\phi(\sigma,t),\pi(\sigma,t)]=\int \d t \d \sigma \pi(\sigma,t) \dot{\phi}(\sigma,t)-\int \d t H[\phi(\sigma),\pi(\sigma)]
        \\
        \dot{\phi}(\sigma,t)=\frac{\delta H}{\delta \pi(\sigma)},\quad \dot{\pi}(\sigma,t)=-\frac{\delta H}{\delta \phi(\sigma)}
        \\
        [A,B]=\int \d \sigma \left(\frac{\delta A}{\delta \phi(\sigma)} \frac{\delta B}{\delta \pi(\sigma)} - \frac{\delta A}{\delta \pi(\sigma)} \frac{\delta B}{\delta \phi(\sigma)}\right)
    \end{cases}
\end{gather*}
将$\sigma$取为空间位置$\bm{x}$

局域性要求Hamiltonian能够写成如下形式
\begin{gather*}
    H=\int \d^3 \bm{x} \mathscr{H}(\pi,\phi,\nabla \phi)
    \\
    S[\phi(x),\pi(x)]=\int \d^4 x \left[\pi(x) \dot{\phi}(x)-\mathscr{H}(\pi,\phi,\nabla \phi)\right]
    \\
    \intertext{从$\dot{\phi}(x)=\pt{\mathscr{H}}{\pi}$反解出$\phi(x)$,代入$S[\phi(x),\pi(x)]$}
    S[\phi(x)]=\int \d^4 x \mathscr{L}(\phi,\dot{\phi},\nabla \phi)
\end{gather*}

根据最小作用量原理,
\begin{gather*}
    \delta S[\phi(x)]=0
    \\
    \intertext{认为无穷远时间和无穷远处$\delta \phi=0$,得到场方程}
    \pt{}{t}\left(\pt{\mathscr{L}}{\dot{\phi}}\right) + \nabla \cdot \left(\pt{\mathscr{L}}{\nabla\phi}\right)=\pt{\mathcal{L}}{\phi}
\end{gather*}

\begin{example}
    场的相互作用势能$\mathscr{U}(\phi)$
    \begin{gather*}
        \mathscr{H}=\frac{1}{2} \left[\pi^2+(\nabla \phi)^2\right]+\mathscr{U}(\phi)
        \\
        \mathscr{L}=\frac{1}{2} \left[(\partial_t \phi)^2-(\nabla \phi)^2\right]-\mathscr{U}(\phi)
        \\
        \intertext{场方程}
        \partial_t^2 \phi - \nabla^2 \phi=-\pt{\mathscr{U}}{\phi}
    \end{gather*}
\end{example}

\section{粒子作为场方程的解}

考虑$1+1$维时空$(t,\sigma)$
\begin{gather*}
    \partial_t^2 \phi - \partial_\sigma^2 \phi=-\pt{\mathscr{U}}{\phi}
    \end{gather*}
    假设$\mathscr{U}(\phi)$有最小值且最小值为$0$,定义真空场位形
    \begin{gather*}
        \Omega=\left\{\phi \vert \partial_t \phi=\partial_\sigma \phi = 0,\quad \mathscr{U}(\phi)=0\right\}
    \end{gather*}
    总能量有限,要求无穷远处为真空场位形
    \begin{gather*}
        \phi_{\pm}= \lim_{\sigma \to \pm \infty} \phi(\sigma) \in \Omega
    \end{gather*}
    \begin{itemize}
        \item 若$\phi_+=\phi_-$,场位形拓扑上等价于真空场位形
        \item 若$\phi_+ \neq \phi_-$,场位形不能连续变形成真空场位形,拓扑上不等价于真空场位形,每对$(\phi_-,\phi_+)$给出场位形的一个拓扑等价类,这样的场位形称为扭结,场方程的扭结解是一种孤立子,对应粒子
    \end{itemize}

    静态场下能量
    \begin{align*}
        E&=V[\phi(\sigma)]
        \\
        &=\int_{-\infty}^{+\infty} \d \sigma \left[\frac{1}{2}(\partial_\sigma \phi)^2+\mathscr{U}(\phi)\right]
        \\
        &\geq \int_{-\infty}^{+\infty} \sqrt{2\mathscr{U}(\phi)} \partial_\sigma \phi \d \sigma
        \\
        \implies E &\geq \left| \int_{\phi_-}^{\phi_+} \sqrt{2\mathscr{U}(\phi)} \d \phi \right|
        \\
        \intertext{引入$W(\phi)$使}
        \mathscr{U}&(\phi)=\frac{1}{2} \left(\dt{W}{\phi}\right)^2
        \\
        E &\geq \left|W(\phi_+)-W(\phi_-)\right|
    \end{align*}
    该能量下界称为Bogomolny能限,对应静态场位形满足Bogomolny方程
    \begin{gather*}
        \partial_{\sigma} \phi=\pm \sqrt{2\mathscr{U}(\phi)}
    \end{gather*}

\begin{example}[\quad sine-Gordon模型]
    \begin{gather*}
        \mathscr{U}(\phi)=1-\cos \phi
    \end{gather*}
    真空场位形
    \begin{gather*}
        \phi=2\pi n,\quad n \in \mathbb{Z}
    \end{gather*}
    每个拓扑等价类由拓扑荷$N$标记
    \begin{gather*}
        N=\frac{\phi_+-\phi_-}{2\pi} \in \mathbb{Z}
    \end{gather*}
    Bogomolny能限
    \begin{gather*}
        E \geq 8|N|
    \end{gather*}
    Bogomolny方程
    \begin{gather*}
        \begin{cases}
            \partial_\sigma \phi=2\sin\frac{\phi}{2} \implies \phi(\sigma)=4\arctan\e^{\sigma-a},\quad N=1
            \\
            \partial_\sigma \phi=-2\sin\frac{\phi}{2} \implies \phi(\sigma)=4\arctan\e^{-\sigma-a},\quad N=-1
            \\
            \text{特解}\phi=2\pi n,\quad n \in \mathbb{Z},\quad N=0
        \end{cases}
    \end{gather*}
    对于$N$为其他值的解,由于扭结之间有相互作用能,无法达到Bogomolny能限,解是不稳定的。$1+1$维sine-Gordon模型是可积场论,可以精确得到所有的多扭结解
\end{example}