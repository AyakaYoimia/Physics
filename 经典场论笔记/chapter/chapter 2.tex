\chapter{相对性原理与场论}

由于Lorentz协变性,作用量是标量;由于局域性,作用量可以写成
\begin{gather*}
    S=\int \d^4 x \mathscr{L}
\end{gather*}
$\mathscr{L}$是标量

考虑标量场论,场变量$\phi$,$\mathscr{L}=\mathscr{L}(\phi,\partial_{\mu}\phi)$,场方程
\begin{gather*}
    \partial_{\mu} \left(\pt{\mathscr{L}}{(\partial_\mu \phi)}\right)=\pt{\mathscr{L}}{\phi}
\end{gather*}

最简单Lagrangian密度
\begin{gather*}
    \mathscr{L}=-\frac{1}{2}\partial_\mu \phi \partial^\mu \phi-\mathscr{U}(\phi)
    \\
    \intertext{对于复标量场}
    \mathscr{L}=-\partial_\mu \overline{\phi} \partial^\mu \phi-\mathscr{U}(\overline{\phi}\phi)
    \\
    \intertext{满足$U(1)$对称性,在如下变换下不变}
    \phi \to \e^{\i \theta} \phi,\quad \overline{\phi} \to \e^{-\i \theta} \overline{\phi}
\end{gather*}