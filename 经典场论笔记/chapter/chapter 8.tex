\chapter{介质中的电磁场与偶极辐射}

\section{偶极耦合}

考虑质量$m$、电荷$+q$、位于$x_1^{\mu}$的粒子,绕着电荷$-q$、位于$x_2^{\mu}$的核运动
\begin{gather*}
    \intertext{作用量}
    S_I=-q \int \bm{A}_{\mu} \d x_2^{\mu}+q \int \bm{A}_{\mu} \d x_1^{\mu}=\int \d t \left[q \phi(x_2)-q \bm{A}(x_2) \cdot \dot{\bm{x}}_2 \right]+\int \d t \left[-q \phi(x_1)+q \bm{A}(x_1) \cdot \dot{\bm{x}}_1 \right]
\end{gather*}
假设核和粒子只在空间原点附近微小运动,在空间原点处做Taylor展开
\begin{gather*}
    S_I \approx \int \d t \left[q \bm{A}(t,0) \cdot (\dot{\bm{x}}_1-\dot{\bm{x}}_2)\right]+\int \d t \left[-q \nabla \phi(t,0) \cdot (\bm{x}_1-\bm{x}_2)\right]+\int \d t \left[q \partial_i A_j(t,0)(x_1^i \dot{x}_1^j-x_2^i \dot{x}_2^j)\right]
\end{gather*}
对第一项分部积分,得到
\begin{gather*}
    \int \bm{E}(t,0) \cdot \bm{p}
\end{gather*}
对第三项进行规范变换,
\begin{gather*}
    A_j \to A_j-\partial_j \varepsilon
    \\
    \varepsilon=\frac{1}{2} A_i(t,\bm{x}) x^i
    \\
    \int \d t \partial_i A_j(t,0) x^i \dot{x}^j \to \frac{1}{2} \int \d t F_{ij}(t,0) x^i \dot{x}^j=\int \d t \frac{1}{2} \epsilon_{ijk} B^k(t,0) x^i \dot{x}^j
    \\
    \intertext{磁矩}
    m_k=\frac{1}{2} q \epsilon_{ijk} (x_1^i \dot{x}_1^j-x_2^i \dot{x}_2^j)
\end{gather*}
总作用量
\begin{gather*}
    S_I \approx \int \d t \left[\bm{p} \cdot \bm{E}(t,0)+\bm{m} \cdot \bm{B}(t,0)\right]
\end{gather*}
容易推广到多粒子体系
\begin{gather*}
    \intertext{定义极化-磁化张量}
    M^{\mu \nu} \to \begin{bmatrix}
        0 & -cP_1 & -cP_2 & -cP_3 \\
        cP_1 & 0 & M_3 & -M_2 \\
        cP_2 & -M_3 & 0 & M_1 \\
        cP_3 & M_2 & -M_1 & 0
    \end{bmatrix}
    \\
    S_I=\int \d^4 x \left[\bm{P} \cdot \bm{E}+\bm{M} \cdot \bm{B}\right]=\frac{1}{2} \int \d^4 x M_{\mu \nu} F^{\mu \nu}
\end{gather*}

\section{介质中的电磁场}

\begin{gather*}
    S_I=\frac{1}{2} \int \d^4 x M_{\mu \nu} F^{\mu \nu}
    \\
    \intertext{对$A_{\mu}$进行变分并分部积分得到}
    \delta S_I=-\int \d^4 x (\partial_{\mu} M^{\mu \nu}) \delta A_{\nu}
    \\
    \intertext{诱导出电流}
    J_I^{\nu}=-\partial_{\mu} M^{\mu \nu}
    \\
    \rho_I=-\nabla \cdot \bm{P},\quad \bm{J}_I=\pt{\bm{P}}{t}+\nabla \times \bm{M}
\end{gather*}
自由电流$J_f^{\mu}$与电磁场耦合作用量$S_f$,电磁场自身作用量$S_{em}=-\frac{1}{4\mu_0} \int \d^4 x F_{\mu \nu} F^{\mu \nu}$,整个系统作用量
\begin{gather*}
    S=S_{em}+S_I+S_f
    \\
    \intertext{对$A_{\mu}$变分得到Maxwell方程}
    -\partial_{\mu} F^{\mu \nu}=\mu_0(J_f^{\nu}+J_I^{\nu})
    \\
    \intertext{定义}
    H^{\mu \nu}=\frac{1}{\mu_0} F^{\mu \nu}-M^{\mu \nu}
    \\
    \intertext{得到}
    -\partial_{\mu} H^{\mu \nu}=J_f^{\nu}
    \\
    \intertext{加上Bianchi恒等式就得到Maxwell方程组}
    \partial_{\mu} F_{\nu \rho} + \partial_{\nu} F_{\rho \mu} + \partial_{\rho} F_{\mu \nu}=0
    \\
    \intertext{定义电位移矢量和磁场强度}
    H^{0i}=cD_i,\quad H^{ij}=\epsilon^{ijk} H_k
    \\
    \bm{D}=\varepsilon_0 \bm{E}+\bm{P},\quad \bm{H}=\frac{1}{\mu_0} \bm{B}-\bm{M}
    \\
    \intertext{矢量形式Maxwell方程组}
    \begin{cases}
        \nabla \cdot \bm{D}=\rho_f
        \\
        \nabla \times \bm{H}=\pt{\bm{D}}{t} + \bm{J}_f
        \\
        \nabla \cdot \bm{B}=0
        \\
        \nabla \times \bm{E}=-\pt{\bm{B}}{t}
    \end{cases}
    \\
    \intertext{本构关系}
    \bm{D}=\bm{D}(\bm{E},\bm{B}),\quad \bm{H}=\bm{H}(\bm{E},\bm{B})
\end{gather*}
\begin{itemize}
    \item 对于均匀各向同性线性介质,在介质参考系内
    \begin{gather*}
        \bm{P}=\varepsilon_0 \chi \bm{E},\quad \bm{D}=\varepsilon \bm{E}
        \\
        \bm{M}=\kappa \bm{H},\quad \bm{B}=\mu \bm{H}
    \end{gather*}
    \item 对于导电介质,还有欧姆定律
    \begin{gather*}
        \bm{J}_f=\sigma \bm{E}
    \end{gather*}
\end{itemize}

\subsection{线性介质中电磁场的能量和动量}

\begin{gather*}
    \intertext{能量-动量张量}
    T_{\text{em}}^{\mu \nu}=\frac{1}{\mu_0} \left[{F^{\mu}}_{\rho} F^{\nu \rho}-\frac{1}{4} \eta^{\mu \nu} F_{\rho \sigma} F^{\rho \sigma}\right]
    \\
    \partial_{\mu} T_{\text{em}}^{\mu \nu}=-{F^{\nu}}_{\rho} J^{\rho}
    \\
    \intertext{假设没有自由电流}
    \partial_{\mu} T_{\text{em}}^{\mu \nu}=-{F^{\nu}}_{\rho} J_I^{\rho}
\end{gather*}
\begin{align*}
    \partial_{\mu} T_{\text{em}}^{\mu \nu}&={F^{\nu}}_{\rho} \partial_{\mu} M^{\mu \rho}
    \\
    &=\partial_{\mu} ({F^{\nu}}_{\rho} M^{\mu \rho})-(\partial^{\mu} F^{\nu \rho}) M_{\mu \rho}
    \\
    &=\partial_{\mu} ({F^{\nu}}_{\rho} M^{\mu \rho})-\frac{1}{2} (\partial^{\mu} F^{\nu \rho}-\partial^{\rho} F^{\nu \mu})M_{\mu \rho}
    \\
    &=\partial_{\mu} ({F^{\nu}}_{\rho} M^{\mu \rho})=\frac{1}{2} (\partial^{\mu} F^{\rho \nu}+\partial^{\rho} F^{\nu \mu})M_{\mu \rho}
\end{align*}
\begin{gather*}
    \partial_{\mu} \left[T_{\text{em}}-{F^{\nu}}_{\rho} M^{\mu \rho}\right]=-\frac{1}{2}(\partial^{\nu} F^{\mu \rho}) M_{\mu \rho}=-\frac{1}{4} \partial^{\nu} (F^{\mu \rho} M_{\mu \rho})+\frac{1}{4}F^{\mu \rho} (\partial^{\nu} M_{\mu \rho})-\frac{1}{4}(\partial^{\nu} F^{\mu \rho}) M_{\mu \rho}
    \\
    \partial_{\mu} \left[T_{\text{em}}-{F^{\nu}}_{\rho} M^{\mu \rho}\right]+\frac{1}{4} \partial^{\nu} (F^{\mu \rho} M_{\mu \rho})=\frac{1}{4}\left[F^{\mu \rho} (\partial^{\nu} M_{\mu \rho})-(\partial^{\nu} F^{\mu \rho}) M_{\mu \rho}\right]
\end{gather*}

\section{偶极子的场}

\subsection{静态偶极子的场}

假设偶极子位于原点
\begin{align*}
    L&=\int \d^3 \bm{x} \left[\frac{1}{2} \varepsilon_0 \bm{E}^2-\frac{1}{2\mu_0} \bm{B}^2+(\bm{p} \cdot \bm{E}) \delta^3(\bm{x})+(\bm{m} \cdot \bm{B}) \delta^3(\bm{x})\right]
    \\
    &=\int \d^3 \bm{x} \left[\frac{1}{2} \varepsilon_0 (\nabla \phi)^2-\frac{1}{2\mu_0} (\nabla \times \bm{A})^2-(\bm{p} \cdot \nabla \phi) \delta^3(\bm{x})+(\bm{m} \cdot (\nabla \times \bm{A})) \delta^3(\bm{x})\right]
\end{align*}
对$\phi$和$\bm{A}$变分,得到
\begin{gather*}
    \varepsilon_0 \nabla^2 \phi=\bm{p} \cdot \nabla \delta^3(\bm{x})
    \\
    -\frac{1}{\mu_0} \nabla \times (\nabla \times \bm{A})=\bm{m} \times \nabla \delta^3(\bm{x})
    \\
    \intertext{取Coulomb规范}
    \frac{1}{\mu_0} \nabla^2 \bm{A}=\bm{m} \times \nabla \delta^3(\bm{x})
    \\
    \intertext{注意到}
    \delta^3(\bm{x})=-\nabla^2 \left(\frac{1}{4\pi |\bm{x}|}\right)
    \\
    \intertext{得到}
    \phi=\frac{\bm{p} \cdot \bm{x}}{4\pi \varepsilon_0 |\bm{x}|^3}
    \\
    \bm{A}=\frac{\mu_0}{4\pi} \frac{\bm{m} \times \bm{x}}{|\bm{x}|^3}
    \\
    \bm{E}=\frac{1}{4\pi \varepsilon_0} \frac{3(\bm{p} \cdot \bm{n})\bm{n}-\bm{p}}{|\bm{x}|^3}
    \\
    \bm{B}=\frac{\mu_0}{4\pi} \frac{3(\bm{m} \cdot \bm{n})\bm{n}-\bm{m}}{|\bm{x}|^3}
\end{gather*}

\subsection{偶极辐射}

假设电偶极矩与磁偶极矩随时间变化,诱导电流
\begin{gather*}
    \rho_I(\bm{x},t)=-\bm{p}(t) \cdot \nabla \delta^3(\bm{x})
    \\
    \bm{J}_I(\bm{x},t)=\dot{\bm{p}}(t) \delta^3(\bm{x})-\bm{m}(t) \times \nabla \delta^3(\bm{x})
\end{gather*}
代入推迟势公式
\begin{align*}
    \phi(\bm{x},t)&=\frac{1}{4\pi \varepsilon_0} \int \d^3 \bm{x'} \frac{-\bm{p}(t-\frac{|\bm{x}-\bm{x'}|}{c}) \cdot \nabla' \delta^3(\bm{x'})}{|\bm{x}-\bm{x'}|}
    \\
    &=\frac{1}{4\pi \varepsilon_0} \int \d^3 \bm{x'} \delta^3(\bm{x'}) \nabla' \cdot \frac{\bm{p}(t-\frac{|\bm{x}-\bm{x'}|}{c})}{|\bm{x}-\bm{x'}|}
    \\
    &=-\frac{1}{4\pi \varepsilon_0} \int \d^3 \bm{x'} \delta^3(\bm{x'}) \nabla \cdot \frac{\bm{p}(t-\frac{|\bm{x}-\bm{x'}|}{c})}{|\bm{x}-\bm{x'}|}
    \\
    &=-\frac{1}{4\pi \varepsilon_0} \nabla \cdot \frac{\bm{p}(t-\frac{|\bm{x}|}{c})}{|\bm{x}|}
\end{align*}
\begin{align*}
    \bm{A}(\bm{x},t)&=\frac{\mu_0}{4\pi} \int \d^3 \bm{x'} \frac{\dot{\bm{p}}(t-\frac{|\bm{x}-\bm{x'}|}{c}) \delta^3(\bm{x'})-\bm{m}(t-\frac{|\bm{x}-\bm{x'}|}{c}) \times \nabla' \delta^3(\bm{x'})}{|\bm{x}-\bm{x'}|}
    \\
    &=\frac{\mu_0}{4\pi|\bm{x}|}\dot{\bm{p}}(t-\frac{|\bm{x}|}{c})+\frac{\mu_0}{4\pi} \nabla \times \frac{\bm{m}(t-\frac{|\bm{x}|}{c})}{|\bm{x}|}
\end{align*}
\begin{gather*}
    \intertext{记$\bm{p}(t-\frac{|\bm{x}|}{c}=[\bm{p}]$)}
    \partial_i [p_j]=-[\dot{p}_j] \partial_i \frac{|\bm{x}|}{c}=-\frac{1}{c} n_i [\dot{p}_j]
    \\
    \intertext{保留到$\frac{1}{|\bm{x}|}$阶}
    \bm{E}=\frac{\mu_0}{4\pi} \frac{(\bm{n} \cdot [\ddot{\bm{p}}])\bm{n}-[\ddot{\bm{p}}]}{|\bm{x}|}
\end{gather*}