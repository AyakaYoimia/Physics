\chapter{全同粒子体系}

\begin{definition*}[全同粒子]
    如果两个粒子的一切固有性质都一样,则称它们是全同的。全同粒子可以交换。    
\end{definition*}

\begin{enumerate}
    \item 经典全同粒子:考虑两个全同粒子,
    \begin{gather*}
        \intertext{若}
        \begin{cases}
            \bm{r}_1(t_0)=\bm{r}_0
            \\
            \bm{v}_1(t_0)=\bm{v}_0
            \\
            \bm{r}_2(t_0)=\bm{r'}_0
            \\
            \bm{v}_2(t_0)=\bm{v'}_0
        \end{cases}
        \implies 
        \begin{cases}
            \bm{r}_1(t)=\bm{r}(t)
            \\
            \bm{r}_2(t)=\bm{r'}(t)
        \end{cases}
        \\
        \intertext{则}
        \begin{cases}
            \bm{r}_1(t_0)=\bm{r'}_0
            \\
            \bm{v}_1(t_0)=\bm{v'}_0
            \\
            \bm{r}_2(t_0)=\bm{r}_0
            \\
            \bm{v}_2(t_0)=\bm{v}_0
        \end{cases}
        \implies 
        \begin{cases}
            \bm{r}_1(t)=\bm{r'}(t)
            \\
            \bm{r}_2(t)=\bm{r}(t)
        \end{cases}
    \end{gather*}
    轨道是分开的,因此每个粒子在编号后可以追踪
    \item 量子力学中的全同粒子:由于波函数假定,全同粒子的波包在演化中会重叠,使粒子不可区分,同一个物理状态对应多个态矢量,即为交换简并,对每一个粒子都进行完全测量后不能确定唯一的右矢,与基本假设发生矛盾
    \begin{example}
        两个全同粒子,假设沿$z$轴自旋一个为$\frac{\hbar}{2}$,另一个为$-\frac{\hbar}{2}$,态空间的基$\{\ket{+,-},\ket{-,+}\}$,则所有右矢
        \begin{gather*}
            \ket{\psi}=\alpha \ket{+,-} + \beta \ket{-,+},\quad |\alpha|^2 + |\beta|^2 = 1
        \end{gather*}
        对应同一个物理状态,而对物理量的测量结果会与在两个基矢上的分量有关,因此需要确定右矢消除交换简并
    \end{example}
\end{enumerate}

\section{置换算符}

\subsection{双粒子体系}

考虑两个自旋均为$S$的、态空间同构的粒子(不要求全同)1和2,在粒子1的态空间中选定一组基$\{\ket{u_i}\}$,通过张量积构成态空间的基
\begin{gather*}
    \{\ket{1:u_i;2:u_j}\}
\end{gather*}
有
\begin{gather*}
    \ket{1:u_i;2:u_j}=\ket{2:u_j;1:u_i}
    \\
    \ket{1:u_i;2:u_j} \neq \ket{1:u_j;2:u_i},\quad \text{若}i \neq j
\end{gather*}

置换算符的作用为
\begin{gather*}
    P_{21} \ket{1:u_i;2:u_j}=\ket{1:u_j;2:u_i}
\end{gather*}
性质有
\begin{itemize}
    \item $P_{21}^2=1$
    \item $P_{21}$是Hermite算符
    \item $P_{21}$是幺正算符
\end{itemize}
\begin{enumerate}
    \item 对称右矢:对应$P_{21}$特征值为$1$
    \item 反对称右矢:对应$P_{21}$特征值为$-1$
\end{enumerate}
\begin{gather*}
    \intertext{设对称化算符}
    S=\frac{1+P}{2}
    \\
    \intertext{反对称化算符}
    A=\frac{1-P}{2}
    \\
    \intertext{有}
    \begin{cases}
        S^2=S
        \\
        A^2=A
    \end{cases}
    \\
    \begin{cases}
        S^{\dagger}=S
        \\
        A^{\dagger}=A
    \end{cases}
    \\
    SA=AS=0
    \\
    S+A=1
\end{gather*}
$S$和$A$的子空间构成正交补

\begin{gather*}
    \intertext{对于与子空间1,2有关的任意算符$\hat{O}(1,2)$,有}
    P_{21}\hat{O}(1,2)P_{21}^{\dagger}=\hat{O}(2,1)
    \\
    \intertext{若}
    \hat{O}_S(2,1)=\hat{O}_S(2,1)
    \\
    \implies [\hat{O}_S(1,2),P_{21}]=0
\end{gather*}

\subsection{多粒子体系}

$N$个粒子体系有$N!$个置换算符,置换算符构成置换群

位调算符:只涉及两个粒子的置换算符,位调算符是Hermite算符,是幺正算符

置换算符是幺正算符,但不一定是Hermite算符

置换算符可以表示为位调算符的乘积,且具有宇称性,对于任意$N$,奇宇称和偶宇称的置换算符个数相等

对于置换算符$P_\alpha$
\begin{enumerate}
    \item 对称右矢:对应$P_\alpha$特征值为$1$
    \item 反对称右矢:对应$P_\alpha$特征值为$\varepsilon_\alpha$
    \begin{gather*}
        \begin{cases}
            \varepsilon_\alpha=1,\text{若}P_\alpha\text{具有偶宇称}
            \\
            \varepsilon_\alpha=-1,\text{若}P_\alpha\text{具有奇宇称}
        \end{cases}
    \end{gather*}
\end{enumerate}
\begin{gather*}
    \intertext{设对称化算符}
    S=\frac{1}{N!} \displaystyle \sum_{\alpha} P_\alpha
    \\
    \intertext{反对称化算符}
    A=\frac{1}{N!} \displaystyle \sum_{\alpha} \varepsilon_\alpha P_\alpha
    \\
    \intertext{有}
    \begin{cases}
        S^2=S
        \\
        A^2=A
    \end{cases}
    \\
    \begin{cases}
        S^{\dagger}=S
        \\
        A^{\dagger}=A
    \end{cases}
    \\
    SA=AS=0
    \\
    \intertext{但}
    S+A \neq 1
\end{gather*}

\section{对称化假设}

\begin{definition*}[对称化假设]
    全同粒子体系态空间中,对应物理状态的右矢,或者是完全对称的,或者是完全反对称的
    \begin{itemize}
        \item 称这类粒子为Boson,如果物理右矢是对称的
        \item 称这类粒子为Fermion,如果物理右矢是反对称的
    \end{itemize}
    经验规律:自旋为整数的粒子是Boson,自旋为半整数的粒子是Fermion
\end{definition*}

对于右矢$\ket{u}$,$P_{\alpha} \ket{u}$也在子空间$\mathscr{E}_u$中,由于
\begin{gather*}
    S\ket{u}=SP_{\alpha} \ket{u}
    \\
    A\ket{u}=\varepsilon_\alpha AP_{\alpha} \ket{u}
\end{gather*}
$\mathscr{E}_u$中右矢在子空间$\mathscr{E}_S$和$\mathscr{E}_A$上的投影共线,从而在两个子空间中分别确定了唯一的右矢(即物理右矢)$S\ket{u}$和$A\ket{u}$

对于Fermion,在$\mathscr{E}_A$上的投影可能为$0$,这样的物理状态不会出现,即为Pauli不相容原理,两个全同Fermion不可能处于相同的单粒子态

多粒子体系确定物理右矢的方法
\begin{itemize}
    \item 对于Boson,直接对各基矢相加取平均
    \item 对于Fermion,写成Slater行列式形式
    \begin{gather*}
        A \ket{u}=\frac{1}{N!}
        \begin{vmatrix}
            \ket{1:u_1} & \ket{1:u_2} & \cdots & \ket{1:u_N} \\
            \ket{2:u_1} & \ket{2:u_2} & \cdots & \ket{2:u_N} \\
            \cdots & \cdots & \cdots & \cdots \\
            \ket{N:u_1} & \ket{N:u_2} & \cdots & \ket{N:u_N}
        \end{vmatrix}
    \end{gather*}
    若单粒子态有相同的,Slater行列式为$0$,即为Pauli不相容原理
\end{itemize}

物理态空间的基:引入占有数$n_k$,代表处于单粒子态$\ket{u_k}$的粒子个数,若两个不同右矢的各态占有数都相等,则向物理态空间中投影得到同一个右矢,为
\begin{gather*}
    \ket{n_1,n_2,\cdots,n_k,\cdots}=
    \begin{cases}
        \sqrt{\frac{N!}{n_1!n_2!\cdots n_k!\cdots}} S\ket{1:u_1;2:u_1;\cdots;n_1:u_1;n_1+1:u_2;\cdots;n_1+n_2:u_2;\cdots},\:\text{对于Boson}
        \\
        \sqrt{N!} A\ket{1:u_1;2:u_1;\cdots;n_1:u_1;n_1+1:u_2;\cdots;n_1+n_2:u_2;\cdots},\:\text{对于Fermion}
    \end{cases}
    \\
    \displaystyle \sum_{k} n_k=N
\end{gather*}
\begin{itemize}
    \item 对于Boson,$\ket{n_1,n_2,\cdots,n_k,\cdots}$构成一组正交归一基
    \item 对于Fermion,各占有数为$0$或$1$,$\ket{n_1,n_2,\cdots,n_k,\cdots}$构成一组基
\end{itemize}

\section{其他假设的应用}

\begin{enumerate}
    \item 测量概率:设测量前为态$\ket{\psi(t)}$,测量后为态$\ket{u}$,则测量结果的概率幅为$\inpro{u}{\psi(t)}$
    \item 观察算符$\hat{G}$对全同粒子体系有对称性,$[\hat{G},P_{\alpha}]=0,\forall P_{\alpha}$,子空间$\mathscr{E}_S$和$\mathscr{E}_A$是$\hat{G}$作用的不变子空间。对称化假设使$\hat{G}$的特征值数量可能减少,但不会增加
    \item $[\hat{H},P_{\alpha}]=0$,使得演化过程中$\ket{\psi(t)}$保持在子空间中
\end{enumerate}

Fermi能:全同Fermion体系基态时单粒子最大的能量

\section{全同性的干涉效应}

考虑两个全同粒子的体系,一个处于单粒子态$\ket{\varphi}$,另一个处于单粒子态$\ket{\chi}$,体系态矢量
\begin{gather*}
    \ket{\varphi;\chi}=\frac{1}{\sqrt{2}}(1+\varepsilon P_{21})\ket{1:\varphi;2:\chi}
    \\
    \varepsilon=\begin{cases}
        1,\quad \text{对于Boson}
        \\
        -1,\quad \text{对于Fermion}
    \end{cases}
    \\
    \intertext{设观测算符$\hat{B}$有离散非简并谱}
    \hat{B}\ket{u_i}=b_i \ket{u_i}
    \\
    \intertext{对于$n \neq n'$,}
    \ket{u_n;u_{n'}}=\frac{1}{\sqrt{2}}(1+\varepsilon P_{21})\ket{1:u_n;2:u_{n'}}
    \\
    \inpro{u_n;u_{n'}}{\varphi;\chi}=\inpro{u_n}{\varphi}\inpro{u_{n'}}{\chi}+\varepsilon\inpro{u_{n'}}{\varphi}\inpro{u_n}{\chi}
    \\
    P(b_n;b_{n'})=|\inpro{u_n;u_{n'}}{\varphi;\chi}=\inpro{u_n}{\varphi}\inpro{u_{n'}}{\chi}+\varepsilon\inpro{u_{n'}}{\varphi}\inpro{u_n}{\chi}|^2
\end{gather*}
编号消失,发生干涉,第一项称为直接项,第二项称为交换项

可以忽略对称化假设的情况
\begin{itemize}
    \item 处于空间中不同区域的无相互作用粒子
    \item 可以用自旋取向鉴别的粒子
\end{itemize}

\section{多电子原子}

假设原子核位置固定,
\begin{gather*}
    \hat{H}=\displaystyle \sum_{i=1}^{Z}\frac{\hat{\bm{P}_i}^2}{2m_e}-\sum_{i=1}^{Z}\frac{Ze^2}{4\pi \epsilon_0 R_i}+\sum_{i<j} \frac{e^2}{4\pi \epsilon_0 |\bm{R}_i-\bm{R}_j|}
\end{gather*}

中心场近似:对于其中一个电子,认为其他$(Z-1)$个电子的运动不受该电子影响,该电子受到其余电子的排斥力取平均,得到平均势场$V_c(R_i)$
\begin{gather*}
    \hat{H} \approx \hat{H}_0=\displaystyle \sum_{i=1}^{Z} \left[\frac{\hat{\bm{P}_i}^2}{2m_e}+V_c(R_i)\right]
    \\
    V_c(r) \approx
    \begin{cases}
        -\frac{e^2}{4\pi \epsilon_0 r},\quad r\text{很小}
        \\
        -\frac{Ze^2}{4\pi \epsilon_0 r},\quad r\text{很大}
    \end{cases}
\end{gather*}
不再有氢原子的偶然简并,能量与$n$和$l$均有关
\begin{gather*}
    E_{n',l}<E_{n,l},\quad n'< n
    \\
    E_{n,l'}<E_{n,l},\quad l'<l
\end{gather*}

\section{氦原子}

\subsection{中心场近似}

\begin{gather*}
    \hat{H}=\hat{H}_0+\hat{W}
    \\
    \hat{H}_0=\frac{\hat{\bm{P}_1}^2}{2m_e}+\frac{\hat{\bm{P}_2}^2}{2m_e}+V_c(R_1)+V_c(R_2)
    \\
    \hat{W}=-\frac{2e^2}{4\pi \epsilon_0 R_1}-\frac{2e^2}{4\pi \epsilon_0 R_2}+\frac{e^2}{4\pi \epsilon_0 |\bm{R}_1-\bm{R}_2|}-V_c(R_1)-V_c(R_2)
\end{gather*}

电子组态
\begin{gather*}
    \ket{n,l,m,\varepsilon;n',l',m',\varepsilon'}
\end{gather*}
简并度
\begin{itemize}
    \item $nl^2$组态$n=n',l=l'$简并度$(2l+1)(4l+1)$
    \item 其余情况$nl,n'l'$简并度$4(2l+1)(2l'+1)$
\end{itemize}

\subsection{微扰}

在组态对应子空间$\mathscr{E}(n,l;n';l')$中计算$\hat{W}$矩阵元,在子空间$\mathscr{E}_{n,l}(1) \otimes \mathscr{E}_{n,l}(2)$中,选取$\hat{\bm{L}}^2,\hat{L}_z,\hat{\bm{S}}^2,\hat{S}_z$共同特征向量作为基,
\begin{gather*}
    \{\ket{1:n,l;2:n',l';L,m_L} \otimes \ket{s,m_s}\}
\end{gather*}
反对称化后得到$\mathscr{E}(n,l;n';l')$的正交归一基
\begin{gather*}
    \ket{n,l;n',l';L,m_L;s,m_s}=c(1-P_{21})\ket{1:n,l;2:n',l';L,m_L} \otimes \ket{s,m_s}
    \\
    P_{21}=P_{21}^{\bm{r}} \otimes P_{21}^{S}
    \\
    P_{21}^{S} \ket{s,m_s}=(-1)^{s+1} \ket{s,m_s}
\end{gather*}
\begin{itemize}
    \item 若$n=n',l=l'$,
    \begin{gather*}
        P_{21}^{\bm{r}} \ket{1:n,l;2:n,l;L,m_L}=(-1)^L \ket{1:n,l;2:n,l;L,m_L}
        \\
        \ket{n,l;n,l;L,m_L;s,m_s}=
        \begin{cases}
            0,\quad L+s \text{为奇数}
            \\
            \ket{1:n,l;2:n,l;L,m_L} \otimes \ket{s,m_s},\quad L+s \text{为偶数}
        \end{cases}
    \end{gather*}
    \item 其余情况,$\mathscr{E}(n,l;n';l')$与$\mathscr{E}_{n,l}(1) \otimes \mathscr{E}_{n,l}(2)$维数相同,归一化系数$c=\frac{1}{\sqrt{2}}$
\end{itemize}
对称化假设要求自旋部分与轨道部分对称性相反

在这组基下,$\hat{W}$矩阵元是对角的,特征值
\begin{gather*}
    \delta(L,s)=\inpro{n,l;n',l';L,m_L;s,m_s}{\hat{W} \vert n,l;n',l';L,m_L;s,m_s}
\end{gather*}
能级$E(n,l;n'l')+\delta(L,s)$,简并度$(2L+1)(2s+1)$,这些次能级称为谱项,记为$^{2s+1}L,L=S,P,D,F \cdots$。简并性的部分消除与对称化假设无关

\subsection{精细结构}

不考虑电子自旋的相互作用,只考虑相对论效应和自旋-轨道耦合,精细结构Hamiltonian为$\hat{H}_{SF}$
\begin{gather*}
    [\hat{H}_{SF},\hat{\bm{J}}]=0
\end{gather*}
对于每一个谱项,$L$和$s$固定,基$\ket{n,l;n',l';L,m_L;s,m_s}$,由于
\begin{gather*}
    j=L+s,L+s-1,\cdots,|L-s|
\end{gather*}
$\hat{H}_{SF}$部分消除简并,能级数为$j$的可能取值数,每一个能级简并度$(2j+1)$,记为$^{2s+1}L_{j},L=S,P,D,F \cdots$

\section{电子气}

考虑有$N$个电子的体系,封闭在边长$L$的立方体中,波函数
\begin{gather*}
    \varphi_{n_x,n_y,n_z}=\left(\frac{2}{L}\right)^2 \sin\frac{n_x \pi x}{L} \sin\frac{n_y \pi y}{L} \sin\frac{n_z \pi z}{L}
    \\
    E_{n_x,n_y,n_z}=\frac{\pi^2 \hbar^2}{2m_e L^2}(n_x^2+n_y^2+n_z^2)=\frac{\hbar^2}{2m_e}\bm{k}_{n_x,n_y,n_z}^2
\end{gather*}
当$L$很大时,$\bm{k}_{n_x,n_y,n_z}$近似连续取值,能量比$E$小的电子数
\begin{gather*}
    N(E)=2 \times \frac{1}{8} \times \frac{\frac{4}{3}\pi \left(\sqrt{\frac{2m_e E}{\hbar^2}}\right)^3}{\left(\frac{\pi}{L}\right)^3}=\frac{L^3}{3\pi^2} \left(\frac{2m_e}{\hbar^2}E\right)^{\frac{3}{2}}
\end{gather*}

Fermi能$E_F$满足
\begin{gather*}
    N(E_F)=N
    \\
    E_F=\frac{\hbar^2}{2m_e}\left(\frac{3\pi^2 N}{L^3}\right)^{\frac{2}{3}}
\end{gather*}
态密度
\begin{gather*}
    \rho(E)=\dt{N(E)}{E}=\frac{L^3}{2\pi^2} \left(\frac{2m_e}{\hbar^2}\right)^{\frac{3}{2}} E^{\frac{1}{2}}
\end{gather*}

绝对零度下电子处于基态,所有电子能量不大于Fermi能。当温度升高时,由于Pauli不相容原理,能量靠近$E_F$的电子先进入到更高能级,因此电子气物理性质(如比热、磁化率、导电性)受到该部分电子影响最大

周期边界条件:实际情况下自由电子波函数为平面波,引入周期性边界条件(Born-von Kármán边界条件),限定后得到$\bm{k}$值的离散集合,满足
\begin{itemize}
    \item $\bm{k}$对应的平面波构成一组基,可以表示之前设定的箱内的函数
    \item 对应的态密度与真实态密度相等
\end{itemize}

\subsection{固体中的电子}

\begin{itemize}
    \item 导体:Fermi能级$\mu$位于容许能带中,
    \item 绝缘体:Fermi能级$\mu$位于禁止能带中,以下的所有容许能带都被填满,禁止能带间隔$\Delta E$越大,绝缘性能越好
\end{itemize}
其中深层的容许能带叫做价带,顶层部分被填充的能带叫做导带

半导体
\begin{itemize}
    \item 本征半导体:升温后电子越过禁止能带产生的导电效应
    \item 非本征半导体:掺杂产生的导电
    \begin{itemize}
        \item $n$型半导体:掺入高价杂质作为施主,禁带上方下降导致禁带变窄
        \item $p$型半导体:掺入低价杂质作为受主,禁带下方上升导致禁带变窄
    \end{itemize}
\end{itemize}