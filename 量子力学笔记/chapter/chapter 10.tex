\chapter{角动量的耦合}

\begin{enumerate}
    \item 经典力学:总角动量
    \begin{gather*}
        \bm{L}=\displaystyle \sum_{i=1}^{N} \bm{L}_i
        \\
        \bm{L}_i=\bm{r}_i \times \bm{p}_i
    \end{gather*}
    \item 量子力学:考虑自旋角动量,相对论效应导致Hamiltonian出现自旋-轨道耦合项
    \begin{gather*}
        \hat{H}_{S0}=\xi(r)\hat{\bm{L}} \cdot \hat{\bm{S}}
    \end{gather*}
    $\bm{L}$和$\bm{S}$不再是运动常量,但总角动量$\bm{J}=\bm{L}+\bm{S}$是运动常量
\end{enumerate}

\section{两个自旋$\frac{1}{2}$的耦合}

两粒子自旋算符分别为$\hat{\bm{S}}_1$和$\hat{\bm{S}}_2$

$\hat{\bm{S}}_1^2$,$\hat{\bm{S}}_2^2$,$\hat{S}_{1z}$,$\hat{S}_{2z}$构成一组ECOC,特征向量$\{\ket{\varepsilon_1,\varepsilon_2}\},\varepsilon_{1,2}=\pm$
\begin{gather*}
    \hat{\bm{S}}_1^2 \ket{\varepsilon_1,\varepsilon_2} = \hat{\bm{S}}_2^2 \ket{\varepsilon_1,\varepsilon_2} = \frac{3}{4} \hbar^2 \ket{\varepsilon_1,\varepsilon_2}
    \\
    \hat{S}_{1z} \ket{\varepsilon_1,\varepsilon_2} = \varepsilon_1 \ket{\varepsilon_1,\varepsilon_2}
    \\
    \hat{S}_{2z} \ket{\varepsilon_1,\varepsilon_2} = \varepsilon_2 \ket{\varepsilon_1,\varepsilon_2}
\end{gather*}

对易关系
\begin{gather*}
    [\hat{S}_z,\hat{\bm{S}}_1^2]=[\hat{S}_z,\hat{\bm{S}}_2^2]=0
    \\
    [\hat{\bm{S}}^2,\hat{\bm{S}}_1^2]=[\hat{\bm{S}}^2,\hat{\bm{S}}_2^2]=0
    \\
    [\hat{S}_z,\hat{S}_{1z}]=[\hat{S}_z,\hat{S}_{2z}]=0
\end{gather*}
但$\hat{\bm{S}}^2$和$\hat{\bm{S}}_1^2$、$\hat{\bm{S}}^2$和$\hat{\bm{S}}_2^2$不对易

$\hat{\bm{S}}_1^2$,$\hat{\bm{S}}_2^2$,$\hat{\bm{S}}^2$,$\hat{S}_z$构成一组ECOC,特征向量$\ket{s,m}$作为基
\begin{gather*}
    \hat{\bm{S}}_1^2 \ket{s,m} = \hat{\bm{S}}_2^2 \ket{s,m} = \frac{3}{4} \hbar^2 \ket{s,m}
    \\
    \hat{\bm{S}}^2 \ket{s,m}=s(s+1)\hbar^2 \ket{s,m}
    \\
    \hat{S}_z \ket{s,m}=m\hbar \ket{s,m}
\end{gather*}
$s$为整数或半整数,$m$取$-s$与$s$之间间隔为$1$的点列

将基$\{\ket{\varepsilon_1,\varepsilon_2}\}$转换到基$\ket{s,m}$,在基$\{\ket{\varepsilon_1,\varepsilon_2}\}$下表示算符并进行对角化
\begin{gather*}
    \hat{S}_z \ket{\varepsilon_1,\varepsilon_2}=\frac{1}{2} (\varepsilon_1+\varepsilon_2) \hbar \ket{\varepsilon_1,\varepsilon_2}
    \\
    m=\frac{1}{2} (\varepsilon_1+\varepsilon_2)
    \\
    m=1,0,-1
    \\
    \hat{S}_z \to \hbar
    \begin{bmatrix}
        1 & 0 & 0 & 0 \\
        0 & 0 & 0 & 0 \\
        0 & 0 & 0 & 0 \\
        0 & 0 & 0 & -1
    \end{bmatrix}
\end{gather*}
$m=\pm 1$非简并,$m=0$二重简并

\begin{gather*}
    \hat{\bm{S}}^2=\hat{\bm{S}}_1^2+\hat{\bm{S}}_2^2+2\hat{S}_{1z}\hat{S}_{2z}+\hat{S}_{1+}\hat{S}_{2-}+\hat{S}_{1-}\hat{S}_{2+}
    \\
    \hat{\bm{S}}^2 \to \hbar^2
    \begin{bmatrix}
        2 & 0 & 0 & 0 \\
        0 & 1 & 1 & 0 \\
        0 & 1 & 1 & 0 \\
        0 & 0 & 0 & 2
    \end{bmatrix}
    \\
    \intertext{$\hat{\bm{S}}^2$和$\hat{S}_z$的共同特征向量}
    \begin{cases}
        \ket{1,1}=\ket{+,+},\text{对应特征值}2\hbar^2\\
        \ket{1,0}=\frac{1}{\sqrt{2}}(\ket{+,-}+\ket{-,+}),\text{对应特征值}2\hbar^2\\
        \ket{0,0}=\frac{1}{\sqrt{2}}(\ket{+,-}-\ket{-,+}),\text{对应特征值}0\\
        \ket{1,-1}=\ket{-,-},\text{对应特征值}2\hbar^2
    \end{cases}
\end{gather*}
特征值$0$非简并,特征值$2\hbar^2$三重简并,$\hat{\bm{S}}^2$和$\hat{S}_z$构成一组ECOC

\section{两个任意角动量的耦合}

总体系由两个子体系$1$和$2$组成

对态空间$\mathscr{E}_1$,
\begin{gather*}
   \hat{\bm{J}}_1^2 \ket{k_1,j_1,m_1}=j_1(j_1+1)\hbar^2 \ket{k_1,j_1,m_1}
   \\
   \hat{J}_{1z} \ket{k_1,j_1,m_1}=m_1\hbar \ket{k_1,j_1,m_1}
   \\
   \hat{J}_{1\pm} \ket{k_1,j_1,m_1}=\hbar \sqrt{j_1(j_1+1)-m_1(m_1\pm1)} \ket{k_1,j_1,m_1 \pm 1}
\end{gather*}

对态空间$\mathscr{E}_2$,
\begin{gather*}
   \hat{\bm{J}}_2^2 \ket{k_2,j_2,m_2}=j_2(j_2+1)\hbar^2 \ket{k_2,j_2,m_2}
   \\
   \hat{J}_{2z} \ket{k_2,j_2,m_2}=m_2\hbar \ket{k_2,j_2,m_2}
   \\
   \hat{J}_{2\pm} \ket{k_2,j_2,m_2}=\hbar \sqrt{j_2(j_2+1)-m_2(m_2\pm1)} \ket{k_2,j_2,m_2 \pm 1}
\end{gather*}

总体系态空间
\begin{gather*}
    \mathscr{E}=\mathscr{E}_1 \otimes \mathscr{E}_2
    \\
    \intertext{基矢}
    \ket{k_1,k_2;j_1,j_2;m_1,m_2}=\ket{k_1,j_1,m_1} \otimes \ket{k_2,j_2,m_2}
    \\
    \mathscr{E}_1=\displaystyle \sum_{\oplus} \mathscr{E}_1(k_1,j_1)
    \\
    \mathscr{E}_2=\displaystyle \sum_{\oplus} \mathscr{E}_2(k_2,j_2)
\end{gather*}

总角动量
\begin{equation*}
    \hat{\bm{J}}=\hat{\bm{J}}_1+\hat{\bm{J}}_2
\end{equation*}
对易关系类似

基变换:原基矢是$\hat{\bm{J}}_1^2$,$\hat{\bm{J}}_2^2$,$\hat{J}_{1z}$,$\hat{J}_{2z}$的共同特征向量,变换到以$\hat{\bm{J}}_1^2$,$\hat{\bm{J}}_2^2$,$\hat{\bm{J}}^2$,$\hat{J}_z$的共同特征向量为基。在子空间$\mathscr{E}(k_1,k_2;j_1,j_2)$中进行对角化,因为对于相同的$j_1$、$j_2$,结果与$k_1$、$k_2$无关,略去$k$指标
\begin{gather*}
    \mathscr{E}(j_1,j_2)=\displaystyle \sum_{\oplus} \mathscr{E}(j)
\end{gather*}
$\mathscr{E}(j_1,j_2)$维数$(2j_1+1)(2j_2+1)$,$\mathscr{E}(j)$两两正交,且是$\hat{\bm{J}}^2$、$\hat{J}_z$、$\hat{J}_{\pm}$作用的不变子空间

\subsection{特征值}

对于$\hat{J}_z$,
\begin{gather*}
    m=m_1+m_2
\end{gather*}
简并度$g_{j_1 j_2}(m)$通过画图求出,在$(m_1,m_2)$坐标内画出斜率$-1$的直线,简并度为矩形$(j_1,j_2)$内落在直线上的格点数

对于$\hat{\bm{J}}^2$,
\begin{gather*}
    j=j_1+j_2,j_1+j_2-1,\cdots,|j_1-j_2|
\end{gather*}
简并度$2j+1$

\subsection{特征向量}

空间$\mathscr{E}(j_1,j_2)$中,$\hat{\bm{J}}^2$和$\hat{J}_z$的共同特征向量记为$\ket{j,m}$
\begin{gather*}
    \hat{\bm{J}}^2 \ket{j,m}=j(j+1) \hbar^2 \ket{j,m}
    \\
    \hat{J}_z \ket{j,m}=m\hbar \ket{j,m}
    \\
    \ket{j,m}=\displaystyle \sum_{m_1=-j_1}^{j_1} \sum_{m_2=-j_2}^{j_2} \ket{j_1,j_2;m_1,m_2} \inpro{j_1,j_2;m_1,m_2}{j,m}
\end{gather*}
$\inpro{j_1,j_2;m_1,m_2}{j,m}$称为Clebsch–Gordan系数

求解步骤
\begin{enumerate}
    \item 先求解$j=j_1+j_2$,
    \begin{gather*}
        \ket{j_1+j_2,j_1+j_2}=\ket{j_1,j_2;j_1,j_2}
    \end{gather*}
    \item 依次作用算符$\hat{J}_-=\hat{J}_{1-}+\hat{J}_{2-}$,得到
    \begin{gather*}
        \ket{j_1+j_2,j_1+j_2-1},\ket{j_1+j_2,j_1+j_2-2},\cdots,\ket{j_1+j_2,-j_1-j_2}
    \end{gather*}
    \item 再求解$j=j_1+j_2-1$,$\ket{j_1+j_2-1,j_1+j_2-1}$应当与$\ket{j_1+j_2,j_1+j_2-1}$正交,且选定相位使得
    \begin{gather*}
        \inpro{j_1,j_2;j_1,j-j_1}{j,j} \text{是正实数}
    \end{gather*}
    \item 依次作用算符$\hat{J}_-=\hat{J}_{1-}+\hat{J}_{2-}$,得到
    \begin{gather*}
        \ket{j_1+j_2-1,j_1+j_2-2},\ket{j_1+j_2-1,j_1+j_2-3},\cdots,\ket{j_1+j_2-1,-j_1-j_2+1}
    \end{gather*}
    \item 重复上述步骤,依次得到
    \begin{gather*}
        \ket{j,j},\ket{j,j-1},\cdots,\ket{j,-j}
    \end{gather*}
    共$(2j+1)$个向量
\end{enumerate}

Clebsch–Gordan系数性质
\begin{itemize}
    \item $\inpro{j_1,j_2;m_1,m_2}{j_1+j_2,m} \geq 0$
    \item $\inpro{j_1,j_2;j_1,m-j_1}{j,m} \geq 0$
    \item $\inpro{j_1,j_2;m+j_2,-j_2}{j,m} \geq 0$
    \item $\inpro{j_1,j_2;m_1,j-m_1}{j,j}$符号为$(-1)^{j_1-m_1}$
    \item $\inpro{j_2,j_1;m_1,m_2}{j,m}=(-1)^{j_1+j_2-j}\inpro{j_1,j_2;m_1,m_2}{j,m}$
    \item $\inpro{j_1,j_2;-m_1,-m_2}{j,-m}=(-1)^{j_1+j_2-j}\inpro{j_1,j_2;m_1,m_2}{j,m}$
    \item $\inpro{j,j;m,-m}{0,0}=\frac{(-1)^{j-m}}{\sqrt{2j+1}}$
\end{itemize}

\subsection{球谐函数加法公式}

\begin{gather*}
    Y_{l_1 m_1}(\Omega)Y_{l_2 m_2}(\Omega)=\displaystyle \sum_{l=|l_1-l_2|}^{l_1+l_2} \sqrt{\frac{(2l_1+1)(2l_2+1)}{4\pi (2l+1)}} \inpro{l_1,l_2;0,0}{l,0} \inpro{l_1,l_2;m_1,m_2}{l,m_1+m_2} Y_{l,(m_1+m_2)}(\Omega)
\end{gather*}

\section{矢量算符}

矢量算符$\hat{\bm{V}}$满足
\begin{gather*}
    [\hat{J}_x,V_x]=0
    \\
    [\hat{J}_x,\hat{V}_y]=\i \hbar \hat{V}_z
    \\
    [\hat{J}_x,\hat{V}_z]=-\i \hbar \hat{V}_y
\end{gather*}
定义
\begin{gather*}
    \hat{V}_{\pm}=\hat{V}_x \pm \i \hat{V}_y
    \\
    \hat{J}_{\pm}=\hat{J}_x \pm \i \hat{J}_y
\end{gather*}
得到
\begin{gather*}
    [\hat{J}_+,\hat{V}_+]=0
    \\
    [\hat{J}_+,\hat{V}_-]=2\hbar \hat{V}_z
    \\
    [\hat{J}_-,\hat{V}_+]=-2\hbar \hat{V}_z
    \\
    [\hat{J}_-,\hat{V}_-]=0
\end{gather*}
矩阵元$\inpro{k,j,m}{\hat{V}_z \vert k',j',m'}$$\inpro{k,j,m}{\hat{V}_{\pm} \vert k',j',m'}$
\begin{gather*}
    \hat{V}_z \implies m-m'=0
    \\
    \hat{V}_+ \implies m-m'=1
    \\
    \hat{V}_- \implies m-m'=-1
\end{gather*}
矩阵元$\inpro{k,j,m}{\hat{V}_{\pm} \vert k',j',m'}$
\begin{gather*}
    \inpro{k,j,m}{\hat{\bm{V}} \vert k',j',m'}=\alpha(k,j) \inpro{k,j,m}{\hat{\bm{J}} \vert k',j',m'}
\end{gather*}
Wigner-Eckart定理
\begin{gather*}
    \hat{P}(k,j) \hat{\bm{V}} \hat{P}(k,j)=\alpha(k,j) \hat{P}(k,j) \hat{\bm{J}} \hat{P}(k,j)
\end{gather*}

投影定理:在子空间$\mathscr{E}(k,j)$中,考虑算符$\hat{\bm{J}} \cdot \hat{\bm{V}}$
\begin{gather*}
    \hat{\bm{V}}=\frac{\langle \bm{J} \cdot \bm{V} \rangle_{k,j}}{\langle \bm{J}^2 \rangle_{k,j}} \hat{\bm{J}}
\end{gather*}

Landég因子$g_J$
\begin{gather*}
    \hat{\bm{J}}=\hat{\bm{L}}+\hat{\bm{S}}
\end{gather*}
$\hat{H}_0$、$\hat{\bm{L}}^2$、$\hat{\bm{S}}^2$、$\hat{\bm{J}}^2$、$\hat{J}_z$构成一组ECOC,共同特征向量$\ket{E_0,l,s,j,m}$
\begin{gather*}
    \hat{J}_{\pm} \ket{E_0,l,s,j,m}=\sqrt{j(j+1)-m(m \pm 1)} \hbar \ket{E_0,l,s,j,m \pm 1}
\end{gather*}
$E_0$是$(2j+1)$重简并的,$\{\ket{E_0,l,s,j,m}\}_{m=-j}^{j}$张成子空间$\mathscr{E}(E_0,l,s,j)$ \\
加入磁场,
\begin{gather*}
    \hat{H}=\hat{H}_0+\hat{H}_1
    \\
    \hat{H}_1=\omega_L (\hat{L}_z+2\hat{S}_z)
    \\
    \omega_L=-\frac{eB}{2m}
\end{gather*}
子空间$\mathscr{E}(E_0,l,s,j)$内,
\begin{gather*}
    \hat{\bm{L}}=\frac{\langle \bm{J} \cdot \bm{L} \rangle_{k,j}}{\langle \bm{J}^2 \rangle_{k,j}} \hat{\bm{J}}
    \\
    \hat{\bm{S}}=\frac{\langle \bm{J} \cdot \bm{S} \rangle_{k,j}}{\langle \bm{J}^2 \rangle_{k,j}} \hat{\bm{J}}
    \\
    \hat{H}_1=g_J \omega_L \hat{J}_z
    \\
    g_J=\frac{3}{2}+\frac{s(s+1)-l(l+1)}{2j(j+1)}
\end{gather*}

\section{耦合角动量的演变}

耦合项
\begin{gather*}
    \hat{W}=a \hat{\bm{J}}_1 \cdot \hat{\bm{J}}_2
\end{gather*}

\begin{enumerate}
    \item 经典
    \begin{gather*}
        \intertext{耦合能量}
        W=a \bm{L}_1 \cdot \bm{L}_2=aL_1 L_2 \cos\theta \ll H_0
        \\
        \intertext{旋转轴$\bm{u}$}
        \dt{\bm{L}_1}{t}=-a \bm{L}_1 \times \bm{L}_2
        \\
        \dt{\bm{L}_2}{t}=-a \bm{L}_2 \times \bm{L}_1
        \\
        \intertext{守恒}
        \dt{\bm{L}}{t}=0
        \\
        \bm{L}_1 \cdot \dt{\bm{L}_1}{t}=\bm{L}_2 \cdot \dt{\bm{L}_2}{t}=0
        \\
        \dt{}{t}(\bm{L}_1 \cdot \bm{L}_2)=0
        \\
        \begin{cases}
            \dt{\bm{L}_1}{t}=a\bm{L} \times \bm{L}_1
            \\
            \dt{\bm{L}_2}{t}=a\bm{L} \times \bm{L}_2
        \end{cases}
    \end{gather*}
    $\bm{L}_1$和$\bm{L}_2$绕$\bm{L}$以角速度$a \bm{L}$旋转
    \item 量子:由Ehrenfest定理,
    \begin{gather*}
        \dt{}{t} \langle \bm{J}_1 \rangle=-a \langle \bm{J}_1 \times \bm{J}_2 \rangle
        \\
        \dt{}{t} \langle \bm{J}_2 \rangle=-a \langle \bm{J}_2 \times \bm{J}_1 \rangle
    \end{gather*}
    一般情况下,
    \begin{gather*}
        \langle \bm{J}_1 \times \bm{J}_2 \rangle \neq \langle \bm{J}_1 \rangle \times \langle \bm{J}_2 \rangle
    \end{gather*}
\end{enumerate}

两个自旋$\frac{1}{2}$之间的碰撞模型:
\begin{gather*}
    \intertext{耦合常数}
    a(t)=\begin{cases}
        a,\quad 0 \leq t \leq T
        \\
        0,\quad \text{else}
    \end{cases}
    \\
    \ket{\psi(0)}=\ket{\psi(-\infty)}=\ket{+,-}
    \\
    \ket{\psi(+\infty)}=\ket{\psi(T)}=\cos\frac{\Omega T}{2} \ket{+,-}-\i \sin \frac{\Omega T}{2} \ket{-,+}
    \\
    E_1-E_0=\hbar \Omega
\end{gather*}
自旋的相互作用在两者间引入了相关性(EPR佯谬)