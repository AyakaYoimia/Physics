\chapter{谐振子}

Hamilton算符
\begin{gather*}
   \hat{H}=\frac{\hat{P}^2}{2m}+\frac{1}{2} m\omega^2 \hat{X}^2
\end{gather*}
特征方程
\begin{gather*}
   \hat{H} \ket{\varphi_\nu^i}=E_\nu \ket{\varphi_\nu^i}=(\nu+\frac{1}{2}) \hbar \omega \ket{\varphi_\nu^i}
\end{gather*}
$\nu$标记不同特征值,$i$标记一个特征子空间中不同正交特征向量

\section{特征值}

\subsection{代数法}

\begin{definition*}
   湮没算符与产生算符
\begin{gather*}
   \hat{a}=\sqrt{\frac{m \omega}{2\hbar}}\hat{X}+\i \frac{1}{\sqrt{2m \hbar\omega}}\hat{P}
   \\
   \hat{a}^{\dagger}=\sqrt{\frac{m \omega}{2\hbar}}\hat{X}-\i \frac{1}{\sqrt{2m \hbar\omega}}\hat{P}
   \\
   [\hat{a},\hat{a}^{\dagger}]=1
\end{gather*}
\end{definition*}
则有
\begin{align*}
   \hat{H}&=\hbar \omega (\hat{a}^{\dagger} \hat{a}+\frac{1}{2})
   \\
   &=\hbar \omega (\hat{a} \hat{a}^{\dagger}-\frac{1}{2})
\end{align*}
$\hat{a}^{\dagger} \hat{a}$与$\hat{H}$所有特征向量相同,对应特征值为$\nu$
\begin{gather*}
   ||\hat{a} \ket{\varphi_\nu^i}||^2=\inpro{\varphi_\nu^i}{\hat{a}^{\dagger} \hat{a} \vert \varphi_\nu^i}=\nu \inpro{\varphi_\nu^i}{\varphi_\nu^i}=\nu ||\ket{\varphi_\nu^i}||^2
   \\
   \implies \nu \geq 0
\end{gather*}
湮没产生算符
的作用
\begin{align*}
   \nu = 0 & \implies  \hat{a} \ket{\varphi_\nu^i}=0
   \\
   \hat{a}^{\dagger} \ket{\varphi_\nu^i} &\neq 0,\quad \forall \nu
   \\
   \hat{H} (\hat{a} \ket{\varphi_\nu^i})&=\hbar \omega (\hat{a} \hat{a}^{\dagger}-\frac{1}{2})\hat{a} \ket{\varphi_\nu^i}
   \\
   &=\hbar \omega \hat{a} (\hat{a}^{\dagger}\hat{a} -\frac{1}{2})\ket{\varphi_\nu^i}
   \\
   &=\hat{a} (\hat{H}-1)\ket{\varphi_\nu^i}
   \\
   &=(\nu-\frac{1}{2}) \hbar \omega \hat{a} \ket{\varphi_\nu^i},\quad \nu > 0
\end{align*}
\begin{align*}
   \hat{H} (\hat{a}^{\dagger} \ket{\varphi_\nu^i})&=\hbar \omega (\hat{a}^{\dagger} \hat{a}+\frac{1}{2}) \hat{a}^{\dagger} \ket{\varphi_\nu^i}
   \\
   &=\hbar \omega \hat{a}^{\dagger} (\hat{a} \hat{a}^{\dagger} + \frac{1}{2})\ket{\varphi_\nu^i}
   \\
   &=\hat{a}^{\dagger} (\hat{H}+1)\ket{\varphi_\nu^i}
   \\
   &=(\nu+\frac{3}{2}) \hbar \omega \hat{a} \ket{\varphi_\nu^i}
\end{align*}
\begin{proposition*}
   $\nu$为非负整数$n$
\end{proposition*}
\begin{proof}
   采用反证法,假设$\exists \nu$不是整数,则$\exists n \geq 0$,使得
   \begin{gather*}
      n<\nu<n+1
   \end{gather*}
   $\hat{a}^n \ket{\varphi_\nu^i}$是特征向量,且特征值$\nu-n>0$,则$\hat{a}^{n+1} \ket{\varphi_\nu^i}$也是特征向量,但$\hat{a}^{n+1} \ket{\varphi_\nu^i}$对应$\hat{a}^{\dagger}\hat{a}$的特征值$\nu-n-1 < 0$,矛盾\\
   则$\nu$为非负整数$n$
\end{proof}
有
\begin{gather*}
   E_n=(n+\frac{1}{2}) \hbar \omega
\end{gather*}
所有能级都是非简并的
\begin{enumerate}
   \item 基态
   \begin{gather*}
      a{\ket{\varphi_0^i}}=0
      \\
      \intertext{在坐标表象下,}
      \dt{}{x} \varphi_0^i(x)+\frac{m\omega}{\hbar} x\varphi_0^i(x)=0
      \\
      \varphi_0^i(x)=\sqrt[4]{\frac{m\omega}{\pi \hbar}} \e^{-\frac{1}{2} \frac{m\omega}{\hbar} x^2}
   \end{gather*}
   不是简并的
   \item 由递推,若某个能级是非简并的,则更高的能级也是非简并的,故所有能级是非简并的
\end{enumerate}

\subsection{多项式法}

Schrödinger方程
\begin{equation*}
   (-\frac{\hbar^2}{2m}\dt{^2}{x^2}+\frac{1}{2}m\omega^2 x^2)\varphi(x)=E\varphi(x)
\end{equation*}
无量纲化,
\begin{gather*}
   \begin{cases}
      \xi=\sqrt{\frac{m \omega}{\hbar}}x
      \\
      \epsilon=\frac{E}{\hbar \omega}
      \\
      \phi(\xi)=\sqrt[4]{\frac{\hbar}{m \omega}} \varphi(\sqrt{\frac{\hbar}{m \omega}} \xi)
   \end{cases}
\end{gather*}
得到
\begin{equation*}
   \dt{^2 \phi}{\xi^2}-(\xi^2-2\epsilon)\phi=0
\end{equation*}
当$x \to \infty$时,解有渐近形式$\e^{\pm \frac{\xi^2}{2}}$,平方可积性要求只能取$\e^{-\frac{\xi^2}{2}}$
\begin{gather*}
   \intertext{设}
   \phi(\xi)=\e^{-\frac{\xi^2}{2}}h(\xi)
   \\
   \intertext{则有}
   \dt{^2 h}{\xi^2}-2\xi \dt{h}{\xi}+(2\epsilon-1)h=0
   \\
   \intertext{设}
   h(\xi)=\xi^p \langle \displaystyle \sum_{m=0}^{+\infty} a_{2m} \xi^{2m} \rangle
   \\
   \intertext{得到}
   \begin{cases}
      a_{2m+2}=\frac{(4m+2p-2\epsilon+1)}{(2m+p+2)(2m+p+1)}a_{2m}
      \\
      p(p-1)a_0=0
   \end{cases}
\end{gather*}
$p=0$或$1$分别对应两种幂级数解

由平方可积性,$h(\xi)$需要截断为多项式,即
\begin{gather*}
   \exists m_0,\:\text{设}2m_0+p=n,
   \\
   \text{有}\epsilon=n+\frac{1}{2}
   \\
   \intertext{即}
   E_n=(n+\frac{1}{2})\hbar \omega
\end{gather*}

\subsection{$\{\ket{p}\}$表象}

动量空间波函数$\bar{\varphi}(p)$,无量纲化
\begin{gather*}
   \begin{cases}
      \zeta = \frac{p}{\sqrt{m\omega \hbar}} \\
      \epsilon=\frac{E}{\hbar \omega} \\
      \bar{\phi}(\zeta)=\sqrt[4]{m \omega \hbar}\bar{\varphi}(\sqrt{m\omega \hbar} \zeta)
   \end{cases}
   \\
   \intertext{对Schrödinger方程进行Fourier变换得到}
   \dt{^2 \bar{\phi}}{\zeta^2}-(\zeta^2-2\epsilon)\bar{\phi}=0
\end{gather*}
$\bar{\psi}$与$\psi$满足相同的方程,二者成比例,
\begin{gather*}
   \bar{\phi}_n(\zeta)=(-\i)^n \phi_n(\xi)
   \\
   \bar{\varphi}_n(p)=\frac{(-\i)^n}{\sqrt{m \omega}}\varphi_n(\frac{p}{m \omega})
\end{gather*}

\section{特征态}

$\hat{H}$与$\hat{a} \hat{a}^{\dagger}$均为观察算符
\begin{enumerate}
   \item $\{\ket{\varphi_n}\}$表象
   \begin{align*}
      \ket{\varphi_n}&=c_n \hat{a}^{\dagger} \ket{\varphi_{n-1}}
      \\
      \inpro{\varphi_n}{\varphi_n}&=|c_n|^2 \inpro{\varphi_{n-1}}{\hat{a} \hat{a}^{\dagger} \vert \varphi_{n-1}}
      \\
      &=|c_n|^2 \inpro{\varphi_{n-1}}{\hat{a}^{\dagger} \hat{a} +1\vert \varphi_{n-1}}
      \\
      &=n|c_n|^2=1
      \\
      c_n=\frac{1}{\sqrt{n}}
      \\
      \ket{\varphi_n}&=\frac{1}{\sqrt{n!}} (\hat{a}^{\dagger})^n \ket{\varphi_0}
   \end{align*}
   产生湮没算符的作用
   \begin{gather*}
      \hat{a}^{\dagger} \ket{\varphi_n}=\sqrt{n+1} \ket{\varphi_{n+1}}
      \\
      \hat{a} \ket{\varphi_n}=\sqrt{n} \ket{\varphi_{n-1}}
   \end{gather*}
   \item $\{\ket{x}\}$表象
   \begin{align*}
      \varphi_0(x)&=\sqrt[4]{\frac{m\omega}{\pi \hbar}} \e^{-\frac{1}{2} \frac{m\omega}{\hbar} x^2}
      \\
      \varphi_n(x)&=\frac{1}{\sqrt{2^n n!}} [\sqrt{\frac{m\omega}{\hbar}}x-\sqrt{\frac{\hbar}{m\omega}} \dt{}{x}]^n \varphi_0(x)
      \\
      &=\frac{1}{\sqrt{2^n n!}} [\sqrt{\frac{m\omega}{\hbar}}x-\sqrt{\frac{\hbar}{m\omega}} \dt{}{x}]^n \sqrt[4]{\frac{m\omega}{\pi \hbar}} \e^{-\frac{1}{2} \frac{m\omega}{\hbar} x^2}
   \end{align*}
   考虑生成函数
   \begin{align*}
      K(\lambda,x)&=\displaystyle \sum_{n=0}^{+\infty} \frac{1}{\sqrt{n!}} \lambda^n \varphi_n(x)
      \\
      &=\sqrt[4]{\frac{m\omega}{\pi \hbar}} \e^{-\frac{m \omega x^2}{2\hbar}+\sqrt{2} \sqrt{\frac{m \omega}{\hbar}} \lambda x -\frac{\lambda^2}{2}}
   \end{align*}
   则有
   \begin{gather*}
      \varphi_n(x)=\sqrt[4]{\frac{m\omega}{\pi \hbar}} \frac{1}{\sqrt{2^n n!}} \e^{-\frac{1}{2} \frac{m\omega}{\hbar} x^2} H_n(\sqrt{\frac{m \omega}{\hbar}} x)
   \end{gather*}
\end{enumerate}

\section{三维各向同性谐振子}

\begin{gather*}
   E_n=(n_x+n_y+n_z+\frac{3}{2})\hbar \omega=(n+\frac{3}{2})\hbar \omega
   \\
   \ket{\psi_{n_x n_y n_z}}=\frac{1}{n_x! n_y! n_z!} (a_x^{\dagger})^{n_x} (a_y^{\dagger})^{n_y} (a_z^{\dagger})^{n_z} \ket{\psi_{000}}
   \\
   \intertext{简并度}
   g_n=\frac{(n+1)(n+2)}{2}
\end{gather*}

\section{匀强电场中的带电谐振子}

\begin{gather*}
   \hat{H}'=\frac{1}{2m}\hat{P}^2+\frac{1}{2}m\omega^2 \hat{X}^2-q E_0 \hat{X}
   \\
   \intertext{$\ket{x}$表象下Schrödinger方程}
   (-\frac{\hbar^2}{2m}\dt{^2}{x^2}+\frac{1}{2}m\omega^2 x^2-qE_0 x)\varphi'(x)=E'\varphi'(x)
   \\
   (-\frac{\hbar^2}{2m}\dt{^2}{x^2}+\frac{1}{2}m\omega^2 (x-\frac{qE_0}{m \omega^2})^2)\varphi'(x)=(E'+\frac{q^2 E_0^2}{2m \omega^2})\varphi'(x)
   \\
   E'_n=(n+\frac{1}{2})\hbar \omega-\frac{q^2 E_0^2}{2m\omega^2}
   \\
   \varphi'(x)=\varphi(x-\frac{qE_0}{m\omega})
\end{gather*}

束缚电子可以近似为谐振子,电偶极矩
\begin{gather*}
   \langle D' \rangle=q\inpro{\varphi'_n}{\hat{X} \vert \varphi'_n}=\frac{q^2 E_0}{m \omega^2}
   \\
   \chi=\frac{q^2}{m\omega^2}
\end{gather*}

\section{准经典态}

\begin{enumerate}
   \item 经典力学
   \begin{gather*}
      \begin{cases}
         \dt{x}{t}=\frac{p}{m} \\
         \dt{p}{t}=-m \omega^2 x
      \end{cases}
      \\
      \intertext{设}
      \alpha=\sqrt{\frac{m\omega}{2\hbar}} x + \frac{\i}{\sqrt{2m\hbar \omega}} p
      \\
      \intertext{有}
      \alpha=\alpha_0 \e^{-\i \omega t}
      \\
      H=\hbar \omega |\alpha_0|^2
      \\
      \intertext{经典条件}
      |\alpha_0| \gg 1
   \end{gather*}
   \item 量子力学
   \begin{gather*}
      \intertext{由Ehrenfest定理}
      \i \hbar \dt{}{t} \langle a \rangle (t)=\langle [\hat{a},\hat{H}] \rangle (t)
      \\
      \langle a \rangle (t)=\langle a \rangle (0) \e^{-\i \omega t}
      \\
      \intertext{准经典条件}
      \begin{cases}
         \langle a \rangle (0)=\alpha_0 \\
         \langle H \rangle=H \iff \langle a^{\dagger}a \rangle = |\alpha_0|^2
      \end{cases}
      \\
      \intertext{引入算符}
      \hat{b}=\hat{a}-\alpha_0
      \\
      |\hat{b} \ket{\psi(0)}|^2=0
      \\
      \implies \hat{a} \ket{\psi(0)}=\alpha_0 \ket{\psi(0)}
      \\
      \intertext{$\alpha_0$是$\hat{a}$的特征值,改记为}
      \hat{a} \ket{\alpha}=\alpha \ket{\alpha}
      \\
      \intertext{在$\ket{\varphi_n}$表象中,}
      \ket{\alpha}=\e^{-\frac{|\alpha|^2}{2}} \displaystyle \sum_{n=0}^{+\infty} \frac{\alpha^n}{\sqrt{n!}} \ket{\varphi_n}
   \end{gather*}
\end{enumerate}

\section{耦合谐振子}

\subsection{有限个谐振子}

\begin{enumerate}
   \item 经典力学 \\
   平衡位置$\pm a$
   \begin{gather*}
      U_0=\frac{1}{2}m\omega^2 (x_1-a)^2+\frac{1}{2}m\omega^2 (x_2+a)^2
      \\
      V=\lambda m \omega^2 (x_1-x_2)^2
      \\
      \intertext{势能}
      U=U_0+V
      \\
      \intertext{简正模}
      \begin{cases}
         x_G=\frac{m_1 x_1 + m_2 x_2}{m_1+m_2} \\
         p_G=p_1+p_2 \\
         \mu_G=m_1+m_2
      \end{cases}
      \\
      \begin{cases}
         x_R=x_1-x_2 \\
         p_R=\frac{m_2 p_1 - m_1 p_2}{m_1+m_2} \\
         \mu_R=\frac{m_1 m_2}{m_1 + m_2}
      \end{cases}
      \\
      H=\frac{p_G^2}{2\mu_G}+\frac{1}{2}\mu_G \omega_G^2 x_G^2+\frac{p_R^2}{2\mu_R}+\frac{1}{2}\mu_R \omega_R^2 (x_R-\frac{2a}{1+4\lambda})^2+m\omega^2 a^2 \frac{4\lambda}{1+4\lambda}
      \\
      \begin{cases}
         \omega_G=\omega \\
         \omega_R=\omega \sqrt{1+4\lambda} 
      \end{cases}
   \end{gather*}
   \item 量子力学 \\
   \begin{gather*}
      [\hat{X}_G,\hat{P}_G]=[\hat{X}_R,\hat{P}_R]=\i \hbar
      \\
      \hat{H}=\hat{H}_G+\hat{H}_R+m\omega^2 a^2 \frac{4\lambda}{1+4\lambda}
      \\
      =\frac{\hat{P}_G^2}{2\mu_G}+\frac{1}{2}\mu_G \omega_G^2 \hat{X}_G^2+\frac{\hat{P}_R^2}{2\mu_R}+\frac{1}{2}\mu_R \omega_R^2 (\hat{X}_R-\frac{2a}{1+4\lambda})^2+m\omega^2 a^2 \frac{4\lambda}{1+4\lambda}
      \\
      \ket{\varphi}=\ket{\varphi_G}\ket{\varphi_R}
      \\
      E_{n,p}=E_n^G + E_p^R + m\omega^2 a^2 \frac{4\lambda}{1+4\lambda}
      \\
      =(n+\frac{1}{2})\hbar \omega_G+(p+\frac{1}{2})\hbar \omega_R+m\omega^2 a^2 \frac{4\lambda}{1+4\lambda}
      \\
      \intertext{Bohr频率}
      \pm \omega_G,\pm \omega_R
   \end{gather*}
\end{enumerate}

\subsection{无限长离散谐振子链}

\begin{enumerate}
   \item 经典力学 \\
   粒子$M_q$位于$x=ql$处
   \begin{gather*}
      \intertext{势能}
      U=\displaystyle \sum_{q=-\infty}^{+\infty} \frac{1}{2} m \omega^2 x_q^2 + \frac{1}{2}m \omega_1^2 (x_q-x_{q+1})^2
      \\
      \intertext{角频率$\Omega$的行波解,色散关系}
      \Omega=\sqrt{\omega^2+4\omega_1^2 \sin^2\frac{kl}{2}}
      \\
      \intertext{取第一Brillouin区}
      -\frac{\pi}{l} \leq k \leq \frac{\pi}{l}
      \\
      \intertext{容许能带}
      \omega \leq \Omega \leq \sqrt{\omega^2+4\omega_1^2}
      \\
      \intertext{Fourier变换得到简正坐标}
      \begin{cases}
         \xi(k,t)=\displaystyle \sum_{q=-\infty}^{+\infty} x_q(t) \e^{-\i qkl} \\
         x_q(t)=\frac{l}{2\pi} \int_{-\frac{\pi}{l}}^{\frac{\pi}{l}} \xi(k,t) \e^{\i kql} \d k
      \end{cases}
      \\
      \begin{cases}
         \pi(k,t)=\displaystyle \sum_{q=-\infty}^{+\infty} p_q(t) \e^{-\i qkl} \\
         p_q(t)=\frac{l}{2\pi} \int_{-\frac{\pi}{l}}^{\frac{\pi}{l}} \pi(k,t) \e^{\i kql} \d k
      \end{cases}
      \\
      \intertext{波矢空间的振子实际上也不独立,要求}
      \begin{cases}
         \xi^*(k,t)=\xi(-k,t) \\
         \pi^*(k,t)=\pi(-k,t)
      \end{cases}
      \\
      \intertext{有}
      \begin{cases}
         m\pt{}{t} \xi(k,t)=\pi(k,t) \\
         \pi(k,t)=-m\Omega^2 \xi(k,t)
      \end{cases}
      \\
      \intertext{变量$\alpha$是独立的,}
      \alpha(k,t)=\sqrt{\frac{m\Omega}{2\hbar}} \xi(k,t) + \frac{\i}{\sqrt{2m\hbar \Omega}} \pi(k,t)
      \\
      \intertext{波矢空间}
      h(k)=\frac{1}{2}m\Omega^2(k) |\xi(k,t)|^2+\frac{1}{2m}|\pi(k,t)|^2
      \\
      H=\frac{l}{2\pi} \int_{-\frac{\pi}{l}}^{\frac{\pi}{l}} h(k) \d k
   \end{gather*}
   代换得到
   \begin{align*}
      h'(k)&=\hbar \Omega(k) \alpha^*(k,t) \alpha(k,t)
      \\
      &=\hbar \Omega(k) \alpha^*(k,0) \alpha(k,0)
      \\
      H&=\frac{l}{2\pi} \int_{-\frac{\pi}{l}}^{\frac{\pi}{l}} h'(k) \d k
   \end{align*}
   \item 量子力学
   \begin{gather*}
      \hat{H}=\displaystyle \sum_{q=-\infty}^{+\infty} \frac{1}{2} m \omega^2 \hat{X}_q^2 + \frac{1}{2m} \hat{P}_q^2+\frac{1}{2}m \omega_1^2 (\hat{X}_q-\hat{X}_{q+1})^2
      \\
      \intertext{Fourier变换}
      \begin{cases}
         \hat{\Xi}(k)=\displaystyle \sum_{q=-\infty}^{+\infty} \hat{X}_q \e^{-\i qkl} \\
         \hat{\Pi}(k)=\displaystyle \sum_{q=-\infty}^{+\infty} \hat{P}_q \e^{-\i qkl}
      \end{cases}
      \\
      \intertext{要求}
      \begin{cases}
         \hat{\Xi}^{\dagger}(k)=\hat{\Xi}(-k) \\
         \hat{\Pi}^{\dagger}(k)=\hat{\Xi}(-k)
      \end{cases}
      \\
      \intertext{对易关系}
      [\hat{\Xi}(k),\hat{\Pi}^{\dagger}(k')]=\i \hbar \frac{2\pi}{l} \delta(k-k')
      \\
      \intertext{引入算符}
      \hat{a}(k)=\sqrt{\frac{m\Omega}{2\hbar}} \hat{\Xi}(k,t) + \frac{\i}{\sqrt{2m\hbar \Omega}} \hat{\Pi}(k,t)
      \\
      \intertext{对易关系}
      \begin{cases}
         [\hat{a}(k),\hat{a}(k')]=[\hat{a}^{\dagger}(k),\hat{a}^{\dagger}(k')]=0 \\
         [\hat{a}(k),\hat{a}^{\dagger}(k')]=\frac{2\pi}{l}\delta
      \end{cases}
      \\
      \hat{h}'(k)=\frac{1}{2}\hbar \Omega(k)(\hat{a}(k)\hat{a}^{\dagger}(k)+\hat{a}^{\dagger}(k)\hat{a}(k))
      \\
      [\hat{h}'(k),\hat{h}'(k')]=0
   \end{gather*}
   \item 声子 \\
   Born-Oppenheimer近似,设电子基态能量$E_0$,认为原子只与相邻的原子有相互作用,原子自身不受回复力,$\omega=0$
   \begin{gather*}
      \Omega(k)=2\omega_1 \left| \sin\frac{kl}{2} \right|
      \\
      \intertext{长波极限下,声速}
      v_s=\omega_1 l
   \end{gather*}
   每一种简正模式$k$对应一种声子
\end{enumerate}

\subsection{无限长连续体系}

\begin{enumerate}
   \item 经典力学 \\
   一维弦,质量线密度$\lambda$,拉力$F$,波速$v=\sqrt{\frac{F}{\lambda}}$
   \begin{gather*}
      (\frac{1}{v^2}\pt{^2 }{t^2}-\pt{^2}{x^2})u(x,t)=0
      \\
      \intertext{引入正交归一基}
      f_n(x)=\sqrt{\frac{2}{L}}\sin\frac{k\pi x}{L}
      \\
      u(x,t)=\displaystyle \sum_{k=1}^{+\infty} q_k(t) f_k(x)
      \\
      \intertext{简正坐标}
      q_k(t)=\int_{0}^{L} u(x,t) f_k(x) \d x
      \\
      \intertext{满足}
      (\dt{^2}{t^2}+\omega_n^2)q_k(t)=0
      \\
      \intertext{其中}
      \omega_k=\frac{k\pi v}{L}
      \\
      q_k=A_k \cos(\omega_k t+\phi_k)
      \\
      \intertext{Hamiltonian}
      H=\displaystyle \sum_{k=1}^{+\infty} (\frac{p_k^2}{2\lambda}+\frac{1}{2}\lambda \omega_k^2 q_k^2)
   \end{gather*}
   \item 量子力学
   \begin{gather*}
      \hat{H}_k=\frac{1}{2} \lambda \omega_k^2 \hat{Q}^2+\frac{1}{2m} \hat{P}^2
      \\
      \hat{H}=\displaystyle \sum_{k=1}^{+\infty} \hat{H}_k 
      \\
      E_{n_1,n_2,\cdots,n_k,\cdots}=\displaystyle \sum_{k=1}^{+\infty} n_k \hbar \omega_k
      \\
      u(x,t)\to \text{位移算符}\hat{U}(x)=\displaystyle \sum_{k=1}^{+\infty} f_k(x) \hat{Q}(k)
   \end{gather*}
   $\hat{Q}_k$与$\hat{H}_k$不对易
   \item 光子
   \begin{enumerate}
      \item 经典 \\
      标量场$U(\bm{r},t)$,三棱长分别为$L_1,L_2,L_3$的长方体导体空腔中,
      \begin{gather*}
         (\frac{1}{c^2}\pt{^2}{t^2}-\nabla^2)U(\bm{r},t)=0
         \\
         \intertext{引入正交归一基}
         f_{klm}(x,y,z)=\sqrt{\frac{8}{L_1 L_2 L_3}} \sin\frac{k\pi x}{L} \sin\frac{l\pi y}{L} \sin\frac{m\pi z}{L}
         \\
         U(x,y,z,t)=\displaystyle \sum_{k,l,m} q_{klm}(t) f_{klm}(x,y,z)
         \\
         \intertext{简正坐标}
         q_{klm}(t)=\int_{0}^{L_1} \d x \int_{0}^{L_2} \d y \int_{0}^{L_3} \d z f_{klm}(x,y,z)U(x,y,z,t)
         \\
         \intertext{得到}
         q_{klm}(t)=A_{klm}\cos(\omega_{klm}t+\phi_{klm})
         \\
         \omega_{klm}^2=\pi^2 c^2(\frac{k^2}{L_1^2}+\frac{l^2}{L_2^2}+\frac{m^2}{L_3^2})
      \end{gather*}
      \item 量子力学
      以真空态能量为零点,得到
      \begin{gather*}
         E_{n_{111},\cdots,n_{klm},\cdots}=\displaystyle \sum_{k,l,m}n_{klm} \hbar \omega_{klm}
      \end{gather*}
      每一个$\hbar \omega_{klm}$视为一个光子
      \begin{gather*}
         \intertext{场算符}
         \hat{S}(\bm{r})=\displaystyle \sum_{k,l,m}f_{klm}(\bm{r}) \hat{Q}_{klm}
      \end{gather*}
      $\hat{S}(\bm{r})$与$\hat{Q}$不可对易
      \begin{gather*}
         \intertext{真空涨落}
         \inpro{0}{\hat{S}(\bm{r}) \vert 0}=\frac{1}{2} \displaystyle \sum_{k,l,m} \frac{\hbar}{m \omega_{klm}} f_{klm}^2(\bm{r})
      \end{gather*}
      真空涨落引起原子自发发射和光谱的Lamb移位,量子电动力学对真空能量
      \begin{equation*}
         E_0=\displaystyle \sum_{k,l,m} \frac{1}{2}\hbar \omega_{klm}
      \end{equation*}
      进行重整化
   \end{enumerate}
\end{enumerate}

\section{谐振子的实例}

\subsection{双原子分子中核的振动} \label{sec:5.7.1}

设两核之间距离为$r$

Born-Oppenheimer近似: \\
先固定$r$,求出电子的定态$E_n(r)$,考虑基态,认为电子浸渐地随着$r$变化,相互作用势能
\begin{equation*}
   V(r)=E_1(r)+\frac{Z_1 Z_2 q^2}{4\pi \epsilon_0 r}
\end{equation*}

转动惯量与$r$有关,平衡位置$r=r_e$,在小振动下振动与转动自由度分离,条件是
\begin{equation*}
   \sqrt{\frac{\hbar}{2m\omega}} \ll r_e
\end{equation*}

\begin{itemize}
   \item 对于异极分子,偶极矩$D$与$r$有关,$\inpro{\psi_n}{D \vert \psi_{n'}}$在$n-n'=\pm 1$时才不为$0$,只有一个Bohr频率$\frac{\omega}{2\pi}$
   \item 入射电磁波角频率$\Omega$,电子受迫振动发生Rayleigh散射 \\
   分子振动,振幅受$\sin(\omega t+\varphi_1)$缓变因子调制,振动角频率$\Omega$,叠加得到角频率$\Omega-\omega$、$\Omega$、$\Omega+\omega$的谱线 \\
   解释:电磁波入射到处于$\ket{\varphi_n}$态的分子上,发生Raman散射
   \begin{itemize}
      \item 弹性散射:能量守恒,得到$\Omega$谱线
      \item 非弹性散射:分子吸收$\hbar \omega$能量,得到$\Omega-\omega$谱线,或分子释放能量,得到$\Omega+\omega$谱线
   \end{itemize}
\end{itemize}

\subsection{晶体中核的振动}

Einstein模型:第$q$个原子平衡位置$x_q^0=qd$
\begin{gather*}
   \intertext{势能}
   U \approx U_0 + \displaystyle \sum_{q} \frac{1}{2} U_0^{''}(x_q-qd)^2
\end{gather*}
第$q$个核运动时,认为其余核不动,其余核与第$q$个核的相互作用势能只与$x_q$有关

\subsection{分子的扭转振荡}

乙烯保持键角不变,一侧的$CH_2$原子团绕$C$——$C$轴旋转$\alpha$,由$\alpha=0,\pi$均为平衡位置,势函数可以近似为
\begin{equation*}
   V(\alpha)=\frac{V_0}{2}(1-\cos2\alpha)
\end{equation*}

\begin{enumerate}
   \item 经典力学 \\
   设两个$CH_2$原子团角坐标分别为$\alpha_1$、$\alpha_2$,在小振动下,Lagrange函数
   \begin{align*}
      \mathcal{L}&=\frac{1}{2}I(\alpha_1^2+\alpha_2^2)+V_0 (\alpha_1-\alpha_2)^2
      \\
      &=\frac{1}{4}I((\alpha_1+\alpha_2)^2+(\alpha_1-\alpha_2)^2)+V_0 (\alpha_1-\alpha_2)^2
   \end{align*}
   可知$\alpha_1+\alpha_2$是循环坐标,$\alpha=\alpha_1-\alpha_2$以角频率$\sqrt{\frac{4V_0}{I}}$振动
   \item 量子力学 \\
   在小振动下,不考虑隧道效应,每个能级$E_n=(n+\frac{1}{2})\hbar \omega$都是二重简并的,对应$\ket{\varphi_n}$和$\ket{\varphi'_n}$,前者以$\alpha=0$为中心,后者以$\alpha=\pi$为中心 \\
   考虑隧道效应,能级简并消除,得到两个定态$\ket{\psi_+^n}$和$\ket{\psi_-^n}$,能级差$\hbar \delta_n \ll \hbar \omega$,则$\langle \alpha \rangle$在以角频率$\omega$振荡基础上叠加角频率$\delta_n$的缓慢振荡
\end{enumerate}

\subsection{重$\mu$原子}

$\mu$原子:$\mu_-$子代替电子与原子核作用形成的体系。对于$\mu$原子,若按照Bohr模型计算,波函数展延范围很小,因此不能将原子核视为点

将原子核视为正电荷均匀分布的球体,半径$r_0$,势能
\begin{equation*}
   V(r)=
   \begin{cases}
      \frac{1}{2} \frac{Z q^2}{4 \pi \epsilon_0 r_0^3} r^2 - \frac{3}{2} \frac{Z q^2}{4\pi \epsilon_0 r_0},\quad 0<r<r_0 \\
      \frac{Z q^2}{4 \pi \epsilon_0 r},\quad r>r_0
   \end{cases}
\end{equation*}
可视为谐振子

\section{热力学平衡的谐振子}

温度恒为$T$
\begin{align*}
   \intertext{密度算符}
   \hat{\rho}&=Z^{-1} \e^{-\frac{\hat{H}}{kT}}
   \\
   \intertext{配分函数}
   Z&=\Tr(\e^{-\frac{\hat{H}}{kT}})
   \\
   &=\frac{\e^{-\frac{\hbar \omega}{2kT}}}{1-\e^{-\frac{\hbar \omega}{kT}}}
\end{align*}
平均能量
\begin{gather*}
   \langle H \rangle=\frac{1}{2}\hbar \omega + \frac{\hbar \omega}{\e^{\frac{\hbar \omega}{kT}}-1}
   \\
   \intertext{经典振子}
   \langle H \rangle_{cl}=kT
\end{gather*}

应用
\begin{enumerate}
   \item 黑体辐射
   \begin{gather*}
      \intertext{能量密度}
      u(\nu)=\frac{8\pi h \nu^3}{c^3} \frac{1}{\e^{\frac{h \nu}{kT}}-1}
   \end{gather*}
   \item Bose-Einstein分布
   \begin{gather*}
      \intertext{简并度}
      a_n=\frac{1}{\e^{\frac{\epsilon-\mu}{kT}}-1}
   \end{gather*}
   \item 固体定容比热:设有$N$个原子,小振动频率$\omega_E$
   \begin{gather*}
      U=3N\langle H \rangle
      \\
      c_V=3Nk \frac{(\frac{\hbar \omega_E}{kT})^2 \e^{\frac{\hbar \omega_E}{kT}}}{(\e^{\frac{\hbar \omega_E}{kT}}-1)^2}
   \end{gather*}
\end{enumerate}

\begin{align*}
   \intertext{概率密度}
   \rho(x)&=\inpro{x}{\hat{\rho} \vert x}
   \\
   &=\frac{1}{\xi \sqrt{\pi}} \e^{-\frac{x^2}{\xi^2}}
   \\
   \xi&=\sqrt{\frac{\hbar}{m\omega} \coth \frac{\hbar \omega}{2kT}}
\end{align*}
当$\sigma_X$约为$\eta d$时,固体熔化,熔点$T_f$近似满足
\begin{gather*}
   \frac{\xi^2}{2d^2} \approx \eta^2
   \\
   \intertext{渐近展开,近似有}
   T_f \propto \omega^2
\end{gather*}

Bolch定理
\begin{gather*}
   G(\hat{X},\hat{P})=\lambda \hat{X}+\mu \hat{P}
   \\
   \langle \e^{-\i q G} \rangle = \e^{-\frac{q^2}{2} \langle G^2 \rangle}
\end{gather*}