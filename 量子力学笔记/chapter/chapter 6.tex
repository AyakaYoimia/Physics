\chapter{角动量}

\begin{equation*}
   \text{总角动量}\bm{J}
   \begin{cases}
      \text{轨道角动量}\bm{L} \\
      \text{内禀角动量}\bm{S}
   \end{cases}
\end{equation*}

\section{对易关系}

\begin{equation*}
   \begin{cases}
      \text{轨道角动量:正则对易关系+量子化规则} \\
      \text{内禀角动量:将对易关系作为定义(几何)}
   \end{cases}
\end{equation*}

量子化
\begin{gather*}
   \hat{L}_x=\hat{Y} \hat{P}_z - \hat{Z} \hat{P}_y
   \\
   \hat{\bm{L}}=\hat{\bm{R}} \times \hat{\bm{P}}
\end{gather*}

对易关系
\begin{gather*}
   \begin{cases}
      [\hat{L}_x,\hat{L}_y]=\i \hbar \hat{L}_z
      \\
      [\hat{L}_y,\hat{L}_z]=\i \hbar \hat{L}_x
      \\
      [\hat{L}_z,\hat{L}_x]=\i \hbar \hat{L}_y
   \end{cases}
\end{gather*}

无自旋体系
\begin{gather*}
   \hat{\bm{L}}=\displaystyle \sum_{i=1}^{N} \hat{\bm{L}}_i
\end{gather*}

推广:将对易关系作为定义,称满足对易关系的物理量为角动量$\bm{J}$,则有
\begin{gather*}
   [\hat{\bm{J}}^2,\hat{\bm{J}}]=0
\end{gather*}

\section{角动量的谱}

寻找$\hat{\bm{J}}^2$和$\hat{J}_z$的共同特征向量
\begin{gather*}
   \intertext{定义}
   \begin{cases}
      \hat{J}_+=\hat{J}_x+\i \hat{J}_y
      \\
      \hat{J}_-=\hat{J}_x-\i \hat{J}_y
   \end{cases}
   \\
   \intertext{对易关系}
   \begin{cases}
      [\hat{J}_z,\hat{J}_+]=\hbar \hat{J}_+
      \\
      [\hat{J}_z,\hat{J}_-]=-\hbar \hat{J}_-
      \\
      [\hat{J}_+,\hat{J}_-]=2\hbar \hat{J}_z
   \end{cases}
   \\
   \hat{\bm{J}}^2=\frac{1}{2}(\hat{J}_+ \hat{J}_- + \hat{J}_- \hat{J}_+)+\hat{J}_z^2
\end{gather*}
$\hat{\bm{J}}^2$半正定,特征值非负

$\hat{\bm{J}}^2$与$\hat{J}_z$不构成ECOC,引入新的指标$k$,特征方程
\begin{gather*}
   \hat{\bm{J}}^2 \ket{k,j,m}=j(j+1)\hbar^2 \ket{k,j,m}
   \\
   \hat{J}_z \ket{k,j,m}=m\hbar \ket{k,j,m}
\end{gather*}

性质
\begin{gather*}
   -j \leq m \leq j
   \\
   \begin{cases}
      m=-j \iff \hat{J}_- \ket{k,j,-j}=0
      \\
      m>-j \implies \text{$\hat{J}_- \ket{k,j,m}$是对应特征值$j(j+1)\hbar$和$(m-1)\hbar$的特征向量}
   \end{cases}
   \\
   \begin{cases}
      m=j \iff \hat{J}_+ \ket{k,j,j}=0
      \\
      m<j \implies \text{$\hat{J}_+ \ket{k,j,m}$是对应特征值$j(j+1)\hbar$和$(m+1)\hbar$的特征向量}
   \end{cases}
   \\
   \begin{cases}
      \text{$j$只能取正整数,正半整数或零} \\
      \text{对于固定$j$,$m$有$2j+1$个取值,}m=-j,-j+1,\cdots,j-1,j
   \end{cases}
\end{gather*}

不同$m$,子空间$\mathscr{E}(j,m)$的维数都相同,$\ket{k,j,m},k=1,2,\cdots,g(j)$构成正交归一基
\begin{gather*}
   \inpro{k,j,m}{k',j',m'}=\delta_{kk'} \delta_{jj'} \delta_{mm'}
   \\
   \displaystyle \sum_{j} \sum_{m=-j}^{j} \sum_{k=1}^{g(j)} \ket{k,j,m} \bra{k,j,m}=1
\end{gather*}
设$\hat{A}$与$\hat{\bm{J}}^2$、$\hat{J}_z$构成ECOC,
\begin{gather*}
   \hat{A} \ket{k,j,m}=a_{k,j} \ket{k,j,m}
\end{gather*}
特征值$a_{k,j}$与$m$无关

将态空间分解为$\mathscr{E}(k,j)$的直和,基$\{\ket{k,j,m}\}_{m=-j}^{j}$
\begin{enumerate}
   \item $\mathscr{E}(k,j)$维数为$(2j+1)$
   \item $\mathscr{E}(k,j)$是$\hat{\bm{J}}$的不变子空间
\end{enumerate}
$\hat{\bm{J}}$在这组基下表示矩阵是分块对角的,且矩阵元只与$j,m$有关 \\
若满足下列条件,称为标准基
\begin{gather*}
   \begin{cases}
      \hat{J}_+ \ket{k,j,m}=\hbar \sqrt{j(j+1)-m(m+1)} \ket{k,j,m+1}
      \\
      \hat{J}_- \ket{k,j,m}=\hbar \sqrt{j(j+1)-m(m-1)} \ket{k,j,m-1}
   \end{cases}
\end{gather*}

\section{轨道角动量}

选择$\ket{\bm{r}}$表象,球坐标下
\begin{gather*}
   \hat{L}_x=\i \hbar (\sin \phi \pt{}{\theta} + \frac{\cos\phi}{\tan \theta} \pt{}{\phi})
   \\
   \hat{L}_y=\i \hbar (-\cos \phi \pt{}{\theta} + \frac{\sin\phi}{\tan \theta} \pt{}{\phi})
   \\
   \hat{L}_z=\frac{\hbar}{\i} \pt{}{\phi}
   \\
   \hat{L}_+=\hbar \e^{\i \phi}(\pt{}{\theta}+\i \cot\theta \pt{}{\phi})
   \\
   \hat{L}_-=\hbar \e^{-\i \phi}(-\pt{}{\theta}+\i \cot\theta \pt{}{\phi})
   \\
   \hat{\bm{L}}^2=-\hbar^2(\pt{^2}{\theta^2}+\frac{1}{\tan\theta}\pt{}{\theta}+\frac{1}{\sin^2 \theta}\pt{^2}{\phi^2})
\end{gather*}
$\hat{\bm{L}}^2$和$\hat{L}_z$的共同特征函数$Y_{lm}(\theta,\phi)$
\begin{gather*}
   \hat{\bm{L}}^2 Y_{lm}(\theta,\phi)=l(l+1)\hbar^2 Y_{lm}(\theta,\phi)
   \\
   \hat{L}_z Y_{lm}(\theta,\phi)=m\hbar Y_{lm}(\theta,\phi)
\end{gather*}
由周期性边界条件,$m$是整数,$l$也是整数

无自旋粒子的波函数空间的基
\begin{gather*}
   \psi_{klm}(\bm{r})=R_{kl}(r) Y_{lm}(\theta,\varphi)
   \\
   \int_{0}^{+\infty} R_{kl}^*(r) R_{k'l}(r) r^2 \d r=\delta_{kk'}
\end{gather*}

\section{角动量与旋转}

几何旋转操作$R$,旋转算符$\hat{R}$

旋转轴方向单位向量$\bm{u}$,旋转角$\alpha$,几何旋转$R_{\bm{u}}(\alpha)$可以用向量
\begin{equation*}
   \bm{\alpha}=\alpha \bm{u}
\end{equation*}
表示,旋转操作构成一个群,且不是Abel的,
\begin{gather*}
   R_{\bm{u}}(\alpha) R_{\bm{u'}}(\alpha') \neq R_{\bm{u'}}(\alpha') R_{\bm{u}}(\alpha)
   \\
   \intertext{但}
   R_{\bm{u}}(\alpha) R_{\bm{u}}(\alpha') = R_{\bm{u}}(\alpha') R_{\bm{u}}(\alpha)=R_{\bm{u}}(\alpha+\alpha')
\end{gather*}

无穷小旋转
\begin{gather*}
   R_{\bm{u}}(\d \alpha) \bm{OM}=\bm{OM}+\d \alpha \bm{u} \times \bm{OM}
   \\
   R_{\bm{u}}(\alpha+\d \alpha)=R_{\bm{u}}(\alpha)R_{\bm{u}}(\d \alpha)
   \\
   R_{\bm{e}_y}(-\alpha') R_{\bm{e}_x}(\d \alpha) R_{\bm{e}_y}(\d \alpha') R_{-\bm{e}_x}(\d \alpha)=R_{\bm{e}_z}(\d \alpha \d \alpha')
\end{gather*}

无自旋粒子的旋转算符:设旋转操作$R$使$\bm{r} \to \bm{r}'$
\begin{gather*}
   \bm{r}'=R \bm{r}
   \\
   \psi'(\bm{r}')=\psi(\bm{r})=\psi(R^{-1} \bm{r}')
   \\
   \intertext{定义旋转算符$\hat{R}$}
   \ket{\psi'}=\hat{R} \ket{\psi}
   \\
   \intertext{得到}
   \inpro{\bm{r}}{\hat{R} \vert \psi}=\inpro{R^{-1} \bm{r}}{\psi}
\end{gather*}
性质
\begin{itemize}
   \item 线性算符
   \item 幺正算符
   \item 旋转算符集合构成一个旋转群的表示
\end{itemize}
旋转算符与角动量算符间的联系
\begin{gather*}
   \hat{R}_{\bm{u}}(\d \alpha)=\hat{I}-\frac{\i}{\hbar} \d \alpha \hat{\bm{L}} \cdot \bm{u}
   \\
   \hat{R}_{\bm{u}}(\alpha)=\e^{-\frac{\i}{\hbar} \alpha \hat{\bm{L}} \cdot \bm{u}}
\end{gather*}
代入无穷小旋转的非对易关系式,可以看出轨道角动量的对易关系是几何旋转群非对易结构的结果。体系的旋转周期性使得$\hat{L}_z$的特征值为整数而非半整数。对于无自旋多粒子体系,上述讨论可以推广。 

对于有自旋粒子,仍满足
\begin{gather*}
   \hat{R}_{\bm{u}}(\d \alpha)=\hat{I}-\frac{\i}{\hbar} \d \alpha \hat{\bm{L}} \cdot \bm{u}
   \\
   \hat{R}_{\bm{u}}(\d \alpha)=\e^{-\frac{\i}{\hbar} \alpha \hat{\bm{L}} \cdot \bm{u}}
\end{gather*}
但无法保持旋转的周期性

对可观察量的旋转
\begin{gather*}
   \hat{A}'=\hat{R} \hat{A} \hat{R}^{\dagger}
\end{gather*}
\begin{enumerate}
   \item 标量算符:$[\hat{A},\bm{J}]=0$
   \item 矢量算符
\end{enumerate}

旋转不变性:在整体旋转下,孤立体系的物理定律不变
\begin{itemize}
   \item 可观察量$\hat{A}'$与$\hat{A}$有相同的谱$\implies \hat{R}$是线性幺正或共轭线性幺正的
   \item 体系演变不受旋转影响,$\hat{R}\ket{\psi(t)}$也是体系的一个解$\iff \hat{H}$是标量算符,方程具有对称性
\end{itemize}
可以导出角动量守恒
\begin{gather*}
   [\hat{H},\hat{\bm{J}}]=0
   \\
   \begin{cases}
      H \ket{k,j,m}=E \ket{k,j,m}
      \\
      \hat{\bm{J}}^2 \ket{k,j,m}=j(j+1)\hbar^2 \ket{k,j,m}
      \\
      \hat{J}_z \ket{k,j,m}=m\hbar \ket{k,j,m}
   \end{cases}
\end{gather*}
对于固定的$k$和$j$,$E$的简并度是$2j+1$

\section{双原子分子的转动} \label{sec:6.5}

两分子质量$m_1$和$m_2$,仅考虑转动自由度,到质心距离$r_1$和$r_2$保持不变,,约化质量$\mu=\frac{m_1 m_2}{m_1 + m_2}$,$r_e=r_1+r_2$
\begin{enumerate}
   \item 经典力学
   \begin{gather*}
      \intertext{转动惯量}
      I=\mu r_e^2
      \\
      |\bm{L}|=\mu r_e^2 \omega
      \\
      H=\frac{\bm{L}^2}{2\mu r_e^2}
   \end{gather*}
   \item 量子力学
   \begin{gather*}
      \hat{H}=\frac{\hat{\bm{L}}^2}{2\mu r_e^2}
      \\
      \intertext{特征向量}
      \ket{l,m} \leftrightarrow Y_{lm}(\theta,\varphi)
      \\
      \hat{H} \ket{l,m}=\frac{l(l+1)\hbar^2}{2\mu r_e^2} \ket{l,m}
      \\
      \intertext{正交归一完备性}
      \inpro{l,m}{l',m'}=\delta_{ll'} \delta_{mm'}
      \\
      \displaystyle \sum_{l=0}^{+\infty} \sum_{m=-l}^{l} \ket{l,m} \bra{l,m}=\hat{I}
   \end{gather*}
   $\inpro{l',m'}{\hat{Z} \vert l,m}$只有$l'-l=\pm1$时才不为$0$,得到Bohr频率
   \begin{gather*}
      \nu_{l,l-1}=\frac{l\hbar}{2\pi \mu r_e^2}
   \end{gather*}
\end{enumerate}

实验
\begin{enumerate}
   \item 异极分子:有固有电偶极矩,纯转动谱由一系列等间隔谱线构成,
   \begin{itemize}
      \item 谱线间隔远小于纯振动谱,波长位于远红外或微波波段,但振动-转动光谱(振动谱的精细结构)不在微波波段
   \end{itemize}
   应用
   \begin{enumerate}
      \item 求分子转动惯量
      \item 由不同同位素构成的分子求同位素质量比
      \item 在热力学平衡下,通过大量分子纯转动谱的谱线强度,求出介质温度
   \end{enumerate}
   \item 同极分子:没有固有电偶极矩 \\
   Raman散射
   \begin{enumerate}
      \item 经典 \\
      设入射波电场沿$z$轴,
      \begin{gather*}
         \bm{D}=(\chi_{\parallel} \bm{E}_{\parallel}+\chi_{\perp} \bm{E}_{\perp}) \cos\Omega t
         \\
         D_z=(\chi_{\perp}+(\chi_{\parallel}-\chi_{\perp})\cos^2\theta)E \cos\Omega t
         \\
         \intertext{分子转动$\omega$使得$\cos\theta$以相同频率振荡}
         \cos\theta=\alpha \cos(\omega t + \beta)
      \end{gather*}
      $\alpha$和$\beta$为待定常数,由此可以得到散射波的三个角频率$\Omega$,$\Omega \pm 2\omega$
      \item 量子
      \begin{gather*}
         \inpro{l',m'}{\hat{D}_z \vert l,m} \propto \int \d \Omega Y_{l'm'}^*(\theta,\varphi) (\chi_{\perp}+(\chi_{\parallel}-\chi_{\perp})\cos^2\theta) Y_{lm}(\theta,\varphi)
      \end{gather*}
      由球谐函数性质,只有在$l'-l=0,\pm2$时上式才不为$0$
   \end{enumerate}
\end{enumerate}

\section{二维谐振子}

势能
\begin{equation*}
   V(x,y)=\frac{m}{2} \omega^2(x^2+y^2)
\end{equation*}

\begin{enumerate}
   \item 经典力学
   \begin{gather*}
      H=\frac{p_x^2+p_y^2}{2m} + \frac{1}{2}m\omega^2(x^2+y^2) + \frac{p_z^2}{2m}
      \\
      \begin{cases}
         x=x_M \cos(\omega t+\varphi_x)
         \\
         y=y_M \cos(\omega t+\varphi_y)
         \\
         z=\frac{p_0}{m}t+z_0
      \end{cases}
      \\
      \intertext{角动量}
      L_z=m\omega x_M y_M \sin(\varphi_x-\varphi_y)
   \end{gather*}
   $xOy$平面运动的运动常量:$H_{xy}$、$H_x$、$H_y$、$L_z$
   \item 量子力学
   \begin{gather*}
      \hat{H}=\hat{H}_{xy}+\hat{H}_z
      \\
      E=E_{xy}+E_z
      \\
      \hat{H} \ket{\varphi}=E \ket{\varphi}
      \\
      \ket{\varphi}=\ket{\varphi_{xy}} \otimes \ket{\varphi_z}
      \\
      \inpro{z}{\varphi_z}=\frac{1}{\sqrt{2\pi\hbar}}\e^{\frac{\i}{\hbar} p_z z}
      \\
      E_z=\frac{p_z^2}{2m}
      \\
      E_{xy}=(n+1)\hbar\omega
   \end{gather*}
   对于$\hat{H}_{xy}$的同一个量子数$n$,$E_{xy}$简并度为$n+1$,$\hat{H}_{xy}$不能单独构成ECOC
   \begin{enumerate}
      \item $\hat{H}_x$与$\hat{H}_y$构成一组ECOC,在$x$方向和$y$方向按谐振子处理
      \item $[\hat{H}_{xy},\hat{L}_z]=0$,可以证明$\hat{H}_{xy}$和$\hat{L}_z$构成一组ECOC
      \begin{gather*}
         \intertext{定义右旋圆量子和左旋圆量子的湮没算符}
         \begin{cases}
            \hat{a}_d=\frac{1}{\sqrt{2}}(\hat{a}_x-\i \hat{a}_y)
            \\
            \hat{a}_g=\frac{1}{\sqrt{2}}(\hat{a}_x+\i \hat{a}_y)
         \end{cases}
         \\
         \intertext{对易关系}
         [\hat{a}_d,\hat{a}_d^{\dagger}]=[\hat{a}_g,\hat{a}_g^{\dagger}]=1
         \\
         \hat{H}_{xy}=(\hat{a}_d^{\dagger} \hat{a}_d + \hat{a}_g^{\dagger} \hat{a}_g + 1) \hbar \omega
         \\
         \hat{L}_z=\hbar(\hat{a}_d^{\dagger} \hat{a}_d-\hat{a}_g^{\dagger} \hat{a}_g)
         \\
         \intertext{设$\hat{a}_d^{\dagger} \hat{a}_d$特征值$n_d$,$\hat{a}_d^{\dagger} \hat{a}_d$特征值$n_g$,则有}
         n=n_d+n_g
         \\
         m=n_d-n_g=n,n-2,\cdots,-n
      \end{gather*}
   \end{enumerate}
\end{enumerate}

\section{Landau能级}

\begin{enumerate}
   \item 经典
   \begin{gather*}
      \bm{f}=q\bm{v} \times \bm{B}(\bm{r})
      \\
      \begin{cases}
         x=x_0+R \cos(\omega_c t+\varphi_0)
         \\
         y=y_0+R \sin(\omega_c t+\varphi_0)
         \\
         z=z_0+v_{0z}t
      \end{cases}
      \\
      \intertext{回旋频率}
      \omega_c=-\frac{qB}{m}
      \\
      \bm{p}=m\bm{v}+q\bm{A}(\bm{r})
   \end{gather*}
   运动常量:$H_{\perp}$、$H_{\parallel}$、$R^2=\frac{2}{m\omega_c^2}H_{\perp}$、相对圆心的角动量 \\
   角动量$L_z$在匀强磁场$\bm{A}=-\frac{1}{2} \bm{r} \times \bm{B}$下是运动常量
   \item 量子
   \begin{gather*}
      \hat{H}=\frac{1}{2m}(\hat{\bm{P}}-q\hat{\bm{A}}(\hat{\bm{R}}))^2
      \\
      \intertext{速度算符}
      \hat{\bm{V}}=\frac{1}{m} (\hat{\bm{P}}-q\hat{\bm{A}}(\hat{\bm{R}}))
      \\
      \intertext{对易关系}
      [\hat{X},\hat{P}_x]=[\hat{Y},\hat{P}_y]=[\hat{Z},\hat{P}_z]=\i\hbar
      \\
      \begin{cases}
         [\hat{V}_x,\hat{V}_y]=\frac{\i q\hbar}{m^2} \hat{B}_z(\hat{\bm{R}})
         \\
         [\hat{V}_y,\hat{V}_z]=\frac{\i q\hbar}{m^2} \hat{B}_x(\hat{\bm{R}})
         \\
         [\hat{V}_z,\hat{V}_x]=\frac{\i q\hbar}{m^2} \hat{B}_y(\hat{\bm{R}})
      \end{cases}
      \\
      [\hat{X},\hat{V}_x]=[\hat{Y},\hat{V}_y]=[\hat{Z},\hat{V}_z]=\frac{\i \hbar}{m}
      \\
      \text{机械角动量:}[\hat{\Lambda}_x,\hat{\Lambda}_y]=\i \hbar(\hat{\Lambda}_z+q\hat{Z} \hat{\bm{R}} \cdot \hat{\bm{B}}(\hat{\bm{R}}))
   \end{gather*}
   各可观察量的演变
   \begin{gather*}
      \intertext{由Ehrenfest定理}
      \dt{}{t} \langle \hat{\bm{R}} \rangle=\langle \hat{\bm{V}} \rangle
      \\
      \intertext{定义}
      \hat{\bm{F}}=\frac{q}{2}(\hat{\bm{V}} \times \hat{\bm{B}}-\hat{\bm{B}} \times \hat{\bm{V}})
      \\
      m \dt{}{t} \langle \hat{\bm{V}} \rangle=\langle \hat{\bm{F}} \rangle
      \\
      \dt{}{t} \langle \hat{\bm{\Lambda}} \rangle=\frac{1}{2}(\hat{\bm{R}} \times \hat{\bm{F}}-\hat{\bm{F}} \times \hat{\bm{R}})
   \end{gather*}
   \item 均匀磁场:将磁场方向取为$z$轴
   \begin{gather*}
      \begin{cases}
         [\hat{V}_x,\hat{V}_y]=-\i \frac{\hbar \omega_c}{m^2}
         \\
         [\hat{V}_y,\hat{V}_z]=[\hat{V}_z,\hat{V}_x]=0
      \end{cases}
      \\
      \hat{H}=\hat{H}_{\parallel}+\hat{H}_{\perp}
      \\
      [\hat{H}_{\parallel},\hat{H}_{\perp}]=0
      \\
      E=E_{\parallel}+E_{\perp}
   \end{gather*}
   \begin{itemize}
      \item $H_{\parallel}$特征值$E_{\parallel}=\frac{m}{2} v_z^2$是连续谱
      \item $H_{\perp}$特征值
      \begin{gather*}
         \intertext{归一化}
         \begin{cases}
            \hat{Q}=\sqrt{\frac{m}{\hbar \omega_c}} \hat{V}_y
            \\
            \hat{S}=\sqrt{\frac{m}{\hbar \omega_c}} \hat{V}_x
         \end{cases}
         \\
         [\hat{Q},\hat{S}]=\i
         \\
         \hat{H}_{\perp}=\frac{1}{2}\hbar \omega_c(\hat{Q}^2+\hat{S}^2)
      \end{gather*}
      与一维谐振子形式相同
      \begin{gather*}
         E_{\perp}=(n+\frac{1}{2})\hbar \omega_c
      \end{gather*}
   \end{itemize}
   Landau能级
   \begin{equation*}
      E(n,v_z)=(n+\frac{1}{2})\hbar \omega_c+\frac{1}{2}mv_z^2
   \end{equation*}
   每一个$n$都是无穷多重简并的,既来自于$v_z$的任意性,也来自于基态$\hat{V}_y+\i \hat{V}_x \ket{\varphi_0}=0$是偏微分方程,解有无穷多

   在$\omega_c>0$条件下选定规范
   \begin{gather*}
      \begin{cases}
         \hat{V}_x=\frac{\hat{P}_x}{m}-\frac{\omega_c}{2}Y
         \\
         \hat{V}_y=\frac{\hat{P}_y}{m}+\frac{\omega_c}{2}X
         \\
         \hat{V}_z=\frac{\hat{P}_z}{m}
      \end{cases}
      \\
      \intertext{Hamiltonian}
      \begin{cases}
         \hat{H}_{\perp}=\frac{\hat{P}_x^2+\hat{P}_y^2}{2m}+\frac{\omega_c}{2}\hat{L}_z+\frac{1}{8}m\omega_c^2(\hat{X}^2+\hat{Y}^2)
         \\
         \hat{H}_{\parallel}=\frac{\hat{P}_z^2}{2m}
      \end{cases}
   \end{gather*}
   $z$方向为自由粒子波函数,考虑$xOy$平面,
   \begin{gather*}
      \hat{H}_{\perp}=\hat{H}_{xy}+\frac{\omega_c}{2} \hat{L}_z
   \end{gather*}
   $\hat{H}_{xy}$为二维谐振子的Hamiltonian,$\hat{H}_{\perp}$与$\hat{H}_{xy}$有相同特征向量,类似计算得到
   \begin{gather*}
      E_{\perp}=(n_d+\frac{1}{2})\hbar\omega_c
   \end{gather*}
   与$n_g$无关,可以看出是无穷多重简并的
\end{enumerate}