\chapter{量子力学的假设}

\begin{enumerate}
   \item 在确定时刻$t_0$的量子态由态空间$\mathscr{E}$中的右矢$\ket{\psi(t_0)}$确定
   \item 每个物理量$A$由$\mathscr{E}$中作用的算符$\hat{A}$描述,$\hat{A}$是一个观察算符
   \item 测量$A$得到的结果是$\hat{A}$的特征值之一
   \item 谱分解原理
   \begin{itemize}
      \item 离散谱:归一化态$\ket{\psi}$,测量$A$得到特征值$a_n$的概率
      \begin{equation*}
         P(a_n)=\displaystyle \sum_{i=1}^{g_n} |\inpro{u_n^i}{\psi}|^2
      \end{equation*}
      $\{\ket{u_n^i}\}$为特征子空间$\mathscr{E}_n$中的一组正交归一基
      \item 非简并连续谱:归一化态$\ket{\psi}$,测量$A$结果在$[\alpha,\alpha+\d \alpha]$中的概率
      \begin{equation*}
         \d P(\alpha)= |\inpro{v_{\alpha}}{\psi}|^2 \d \alpha
      \end{equation*}
      $\ket{v_{\alpha}}$为$\hat{A}$的属于特征值$\alpha$的特征向量
      \item[$\circ$] {\color{red} 注意:} 对特征态添加相因子不改变量子态,但在叠加态下,对不同特征向量添加不同相因子一般会改变量子态,除非相位差为$2n\pi$
   \end{itemize}
   \item 波包的坍缩:对$\ket{\psi}$态测量$A$得到$a_n$,则观测后的态为
   \begin{equation*}
      \frac{P_n \ket{\psi}}{\sqrt{\inpro{\psi}{P_n \vert \psi}}}
   \end{equation*}
   投影算符
   \begin{equation*}
      P_n=\displaystyle \sum_{i=1}^{g_n} \ket{u_n^i} \bra{u_n^i}
   \end{equation*}
   \item 体系的演变:Schrödinger方程
   \begin{equation*}
      \i \hbar \dt{}{t} \ket{\psi(t)} = \hat{H}(t) \ket{\psi(t)} 
   \end{equation*}
\end{enumerate}
量子化规则:经典物理量$A$对应算符$\hat{A}$可以在适当对称化后,将$\bm{r}$和$\bm{p}$替换为$\hat{R}$和$\hat{P}$得到

\section{统计量的物理解释}
\begin{enumerate}
   \item 期望(对不同态的平均值)
   \begin{equation*}
      \langle A \rangle_{\psi}=\inpro{\psi}{\hat{A} \vert \psi}
   \end{equation*}
   \item 标准差
   \begin{equation*}
      \sigma_A=\sqrt{\langle (A-\langle A \rangle)^2 \rangle}=\sqrt{\langle A^2 \rangle-(\langle A \rangle)^2} 
   \end{equation*}
   \item 不确定性关系
   \begin{align*}
      \sigma_A^2 \sigma_B^2 &= \langle (A-\langle A \rangle)^2 \rangle \langle (B-\langle B \rangle)^2 \rangle
      \\
      & \geq |\inpro{(\hat{A}-\langle A \rangle) \psi}{(\hat{B}-\langle B \rangle) \psi}|^2
      \\
      & \geq  |\Im(\inpro{(\hat{A}-\langle A \rangle) \psi}{(\hat{B}-\langle B \rangle) \psi})|^2
      \\
      &= |\frac{\inpro{(\hat{A}-\langle A \rangle) \psi}{(\hat{B}-\langle B \rangle) \psi}-\inpro{(\hat{A}-\langle A \rangle) \psi}{(\hat{B}-\langle B \rangle) \psi}^*}{2\i}|^2
      \\
      &= \frac{1}{4} |\inpro{\psi}{[\hat{A}-\langle A \rangle,\hat{B}-\langle B \rangle] \vert \psi}|^2
      \\
      &= \frac{1}{4} |\inpro{\psi}{[\hat{A},\hat{B}] \vert \psi}|^2
      \\
      &= \frac{1}{4} |\langle [A,B] \rangle|^2
      \\
      \intertext{综上可得}
      \sigma_A \sigma_B &\geq \frac{1}{2} |\langle [A,B] \rangle|
   \end{align*}
   特别地,有
   \begin{equation*}
      \Delta E \Delta t \geq \frac{\hbar}{2}
   \end{equation*}
\end{enumerate}

相容性
\begin{enumerate}
   \item 相容的可观察量在同一时间不同测量顺序得到相同的结果
   \item 不相容的可观察量在同一时间不同测量顺序得到不同的结果
   \item 为了使体系的态在测量后由结果可以唯一确定,所测的可观察量必须是ECOC
\end{enumerate}

\section{Schrödinger方程的物理解释}

概率守恒:
\begin{equation*}
   \dt{}{t} \inpro{\psi}{\psi}=0
\end{equation*}

概率流:考虑无自旋的单个粒子,
\begin{gather*}
   \intertext{概率密度}
   \rho(\bm{r},t)=|\psi(\bm{r},t)|^2
   \\
   \intertext{概率流密度}
   \bm{J}(\bm{r},t)=\frac{\hbar}{2m\i} (\psi^* \nabla \psi - \psi \nabla \psi^*)=\frac{1}{m} \Re(\psi^* \frac{\hbar}{\i} \nabla \psi)
   \\
   \intertext{概率流密度算符}
   \hat{K}=\frac{1}{2m} (\ket{\bm{r}}\bra{\bm{r}} \hat{P}+\hat{P}\ket{\bm{r}}\bra{\bm{r}})
\end{gather*}

可观察量的平均值的演变:\\
Ehrenfest定理:
\begin{equation*}
   \dt{}{t} \langle A \rangle = \frac{1}{\i \hbar} \langle [A,H(t)] \rangle + \langle \pt{A}{t} \rangle
\end{equation*}
对$\hat{R}$和$\hat{P}$的特例
\begin{gather*}
   \dt{}{t} \langle \bm{r} \rangle = \frac{1}{m} \langle \bm{p} \rangle
   \\
   \dt{}{t} \langle \bm{p} \rangle = - \langle \nabla V(\bm{r}) \rangle
\end{gather*}
波包中心不一定沿经典轨道,因为
\begin{gather*}
   \bm{F} = -\left. \nabla V(\bm{r}) \right|_{\bm{r}=\langle \bm{r} \rangle} \neq - \langle \nabla V(\bm{r}) \rangle
\end{gather*}
在宏观极限下,可以回归到经典力学的情况

保守体系:$\hat{H}$不显含时间,特征方程
\begin{equation*}
   \hat{H} \ket{\varphi_{n\tau}}=E_n \ket{\varphi_{n\tau}}
\end{equation*}
$\tau$为ECOC中其他参数
\begin{gather*}
   \ket{\psi(t_0)}=\displaystyle \sum_{n} \sum_{\tau} c_{n\tau}(t_0) \ket{\varphi_{n\tau}}
   \\
   \ket{\psi(t)}=\displaystyle \sum_{n} \sum_{\tau} c_{n\tau}(t_0) \e^{-\i E_n(t-t_0) /\hbar} \ket{\varphi_{n\tau}}
\end{gather*}
称$\hat{H}$的特征态为定态,定态的物理性质随时间不变

运动常量$A$满足
\begin{gather*}
   \begin{cases}
      \pt{A}{t} = 0 
      \\
      [A,H]=0
   \end{cases}
\end{gather*}
运动常量的特征值称为好量子数,定态保持为$A$的特征态,测量结果的概率不变

对于不显含时间的、与$H$不对易的可观察量$B$,
\begin{gather*}
   \langle B \rangle(t)=\inpro{\psi(t)}{\hat{B} \vert \psi(t)}=\displaystyle \sum_n \sum_\tau \sum_{n'} \sum_{\tau'} c_{n'\tau'}^*(t_0) c_{n\tau}(t_0) \inpro{\varphi_{n'\tau'}}{\hat{B} \vert \varphi_{n\tau}} \e^{\i(E_{n'}-E_n)(t-t_0)/\hbar}
   \\
   \intertext{Bohr频率}
   \nu_{nn'}=\frac{1}{2\pi} \frac{|E_{n'}-E_n|}{\hbar}
\end{gather*}
在一些情况下,$\nu_{nn'}$权重为$0$,给出选择定则

\section{对物理体系一部分的测量}

态空间$\mathscr{E}_1 \otimes \mathscr{E}_2$,测量的物理量的算符
\begin{equation*}
   \hat{A}_1 \to \tilde{A}_1 = \hat{A}_1 \otimes \hat{I}_2
\end{equation*}

$\{\ket{u_n^i}\}$是$\hat{A}$的正交归一特征向量,$\{\ket{v_k}\}$是$\mathscr{E}_2$中任意一组正交归一基
\begin{align*}
   \tilde{P}_{1,n}&=\hat{P}_{1,n} \otimes \hat{I}_2
   \\
   &=(\displaystyle \sum_{i=1}^{g_n} \ket{u_n^i}\bra{u_n^i}) \otimes \hat{I}_2
   \\
   &=\displaystyle \sum_{i=1}^{g_n} \sum_k \ket{u_n^i v_k}\bra{u_n^i v_k}
\end{align*}
测量后态矢量
\begin{align*}
   \ket{\psi'}=\frac{\tilde{P}_{1,n} \ket{\psi}}{\sqrt{\inpro{\psi}{\tilde{P}_{1,n} \vert \psi}}}
\end{align*}

\begin{enumerate}
   \item $\hat{A}$构成ECOC,不论测量前是否是乘积态,测量后一定变为乘积态
   \begin{gather*}
      \ket{\psi'}=\ket{u_n} \otimes \ket{\chi'}
      \\
      \ket{\chi'}=\frac{\displaystyle \sum_{k} \ket{v_k} \inpro{u_n v_k}{\psi}}{\sqrt{\displaystyle \sum_k |\inpro{u_n v_k}{\psi}|^2}}
   \end{gather*}
   \item 测量前的态是乘积态
   \begin{gather*}
      \ket{\psi}=\ket{\varphi} \otimes \ket{\chi}
      \\
      \intertext{则测量后的态也是张量积}
      \ket{\psi'}=\ket{\varphi'} \otimes \ket{\chi}
      \\
      \ket{\varphi'}=\frac{\hat{P}_{1,n} \ket{\varphi}}{\sqrt{\inpro{\varphi}{\hat{P}_{1,n} \vert \varphi}}}
   \end{gather*}
   代表两个子体系是独立的
   \item 测量前的态不是乘积态:两个子体系不独立,两个体系之间存在相互作用,用密度算符描述子体系$1$
\end{enumerate}

\section{密度算符}

统计混合态:对体系的初态了解并不完备,分布列为$\ket{\psi_k}$出现的概率为$p_k$\\
与$\ket{\psi}=\displaystyle \sum_{k} c_k \ket{\psi_k}$不等价,因为后者会有干涉效应

纯态:$p_k$中某一个为$1$,其余为$0$
\begin{gather*}
   \ket{\psi(t)}=\displaystyle \sum_{n} c_n(t) \ket{u_n}
   \\
   \intertext{引入密度算符}
   \hat{\rho}(t)=\ket{\psi(t)} \bra{\psi(t)}
   \\
   \intertext{概率归一化}
   \Tr \hat{\rho}(t)=1
   \\
   \langle A \rangle (t)=\Tr[\hat{\rho}(t) \hat{A}]
   \\
   \intertext{由Schrödinger方程,}
   \dt{}{t} \hat{\rho}(t)=\frac{1}{\i \hbar} [\hat{H},\hat{\rho}]
   \\
   P(a_n)=\Tr[\hat{P}_n \hat{\rho}(t)]
\end{gather*}
纯态下体系可以用态矢量描述,也可以用密度算符描述

统计混合态
\begin{gather*}
   \hat{\rho}_k = \ket{\psi_k} \bra{\psi_k}
   \\
   P_k(a_n)=\Tr[\hat{\rho}_k \hat{P_n}]
   \\
   P(a_n)=\displaystyle \sum_k p_k P_k(a_n)
   \\
   =Tr[\displaystyle \sum_k p_k \hat{\rho}_k P_n]
   \\
   \intertext{定义密度算符}
   \hat{\rho}=\displaystyle \sum_k p_k \hat{\rho}_k
   \\
   \intertext{推广后仍有}
   \Tr \hat{\rho}(t)=1
   \\
   \langle A \rangle (t)=\Tr[\hat{\rho}(t) \hat{A}]
   \\
   \intertext{假设$\hat{H}(t)$已知,则$\forall t,\ket{\psi_k(t)}$对应的概率为$p_k$}
   \dt{}{t} \hat{\rho}(t)=\frac{1}{\i \hbar} [\hat{H},\hat{\rho}]
\end{gather*}
$\hat{\rho}$为正定算符

$\hat{\rho}$的矩阵元
\begin{itemize}
   \item 对角元:布居数
   \begin{gather*}
      c_n^k=\inpro{u_n}{\psi_k}
      \\
      \hat{\rho}_{nn}=\displaystyle \sum_k p_k|c_n^k|^2
   \end{gather*}
   \item 非对角元:相干元
   \begin{gather*}
      \hat{\rho}_{np}=\displaystyle \sum_k p_k c_n^k {c_p^k}^*
   \end{gather*}
\end{itemize}

应用
\begin{enumerate}
   \item 热力学平衡体系
   \begin{gather*}
      \intertext{密度算符}
      \hat{\rho}=Z^{-1} \e^{-\frac{\hat{H}}{kT}}
      \\
      \intertext{配分函数}
      Z=\Tr(\e^{-\frac{\hat{H}}{kT}})
   \end{gather*}
   在$\hat{H}$的特征向量组成的基下,$\hat{\rho}$是对角的
   \item 对物理体系的一部分的描述 \\
   $\hat{\rho}$作用于$\mathscr{E}=\mathscr{E}_1 \otimes \mathscr{E}_2$,定义作用于$\mathscr{E}_1$的密度算符
   \begin{gather*}
      \hat{\rho}(1)=\Tr_2 \hat{\rho}
      \\
      \hat{\rho}(1)_{nn'}=\displaystyle \sum_k (\bra{u_n} \bra{v_k}) \hat{\rho} (\ket{u_{n'}} \ket{v_p})
      \\
      \langle \tilde{A} \rangle = \Tr[\hat{\rho}(1) \hat{A}]
   \end{gather*}
   对于不可分解为张量积的密度算符,
   \begin{equation*}
      \hat{\rho} \neq \hat{\rho}(1) \otimes \hat{\rho}(2)
   \end{equation*}
\end{enumerate}

\section{演变算符}

演变算符$\hat{U}(t,t_0)$
\begin{equation*}
   \ket{\psi(t)}=\hat{U}(t,t_0)\ket{\psi(t_0)}
\end{equation*}

性质
\begin{itemize}
   \item 由Schrödinger方程,
   \begin{gather*}
      \i \hbar \pt{}{t} \hat{U}(t,t_0)=\hat{H}(t)\hat{U}(t,t_0)
   \end{gather*}
   \item \begin{gather*}
      \hat{U}(t_n,t_1)=\hat{U}(t_n,t_{n-1})\cdots\hat{U}(t_3,t_2)\hat{U}(t_2,t_1)
   \end{gather*}
   \item \begin{gather*}
      \hat{U}(t',t)=\hat{U}^{-1}(t,t')
   \end{gather*}
   \item 无穷小演变算符
   \begin{gather*}
      \hat{U}(t+\d t,t)=\hat{I}-\frac{\i}{\hbar} \hat{H}(t) \d t
      \\
      \implies \hat{U}(t,t_0) \text{是幺正算符}
   \end{gather*}
\end{itemize}

\begin{enumerate}
   \item $\hat{H}$与时间无关
   \begin{equation*}
      \hat{U}(t,t_0)=\e^{-\i \hat{H} (t-t_0) /\hbar}
   \end{equation*}
   \item $\hat{H}$与时间有关:一般而言,
   \begin{equation*}
      \hat{U}(t,t_0) \neq \e^{-\frac{\i}{\hbar} \int_{t_0}^{t} \hat{H}(t') \d t'}
   \end{equation*}
\end{enumerate}

\section{绘景}

Schrödinger绘景:可观察量与时间无关,体系的演变包含在态矢量$\ket{\psi_S(t)}$中

Heisenberg绘景:通过幺正变换,使态矢量与时间无关,可观察量依赖于时间
\begin{gather*}
   \ket{\psi_H}=\ket{\psi_S(t_0)}
   \\
   \hat{A}_H(t)=\hat{U}^{\dagger}(t,t_0)\hat{A}_S(t)\hat{U}(t,t_0)
   \\
   \intertext{对于保守系,若$\hat{A}_S$与$\hat{H}_S$对易,则$\hat{A}_H$也与时间无关}
   A_H=A_S
   \\
   \intertext{一般情况}
   \i \hbar \dt{}{t} \hat{A}_H(t)=[\hat{A}_H(t),\hat{H}_H(t)]+\i \hbar \left ( \dt{}{t} \hat{A}_S(t) \right )_H
   \\
   \intertext{对于势场中的粒子,有Ehrenfest方程}
   \dt{}{t} \hat{X}_H(t)=\frac{1}{m} \hat{P}_H(t)
   \\
   \dt{}{t} \hat{P}_H(t)=-\pt{V(\hat{X},t)}{\hat{X}}
\end{gather*}
下标$H$表示变换到Heisenberg绘景 \\

\section{规范不变性}

规范选择:选定一组电磁势

规范变换
\begin{equation*}
   \begin{cases}
      \bm{A}'=\bm{A}+\nabla \Lambda
      \\
      \Phi'=\Phi - \frac{\partial \Lambda}{\partial t}
   \end{cases}
\end{equation*}

规范不变性:规范变换不改变物理结果
\begin{enumerate}
   \item 经典力学
   \begin{enumerate}
      \item Newton力学:$\bm{F}=q\bm{E}+q\bm{v} \times \bm{B}$仅涉及电磁场
      \item Hamilton力学
      \begin{gather*}
         \intertext{机械动量}
         \bm{p}_0=m\bm{v}
         \\
         \intertext{动量}
         \bm{p}=m\bm{v}+q\bm{A}
         \\
         \begin{cases}
            \bm{r'}=\bm{r}
            \\
            \bm{p'}=\bm{p}+q\nabla \Lambda
         \end{cases}
      \end{gather*}
      \begin{itemize}
         \item 真实物理量:与规范无关 \\
         在原规范下对应$g(\bm{r},\bm{p},t)$,在新规范下对应$g'(\bm{r'},\bm{p'},t)$,则真实物理量的条件为
         \begin{equation*}
            g(\bm{r}(t),\bm{p}(t),t)=g'(\bm{r}(t),\bm{p}(t)+\nabla \Lambda(\bm{r}(t),t),t)
         \end{equation*}
         例如:位矢,机械动量,机械动能,机械角动量
         \item 非物理量:在规范变换下会变化 \\
         例如:$\frac{\bm{p}^2}{2m}$,角动量$\bm{r} \times \bm{p}$,Hamiltonian
      \end{itemize}
   \end{enumerate}
   \item 量子力学
   \begin{gather*}
      \intertext{规范变换前后}
      \begin{cases}
         \hat{R}=\hat{R}'
         \\
         \hat{P}=\hat{P}'
      \end{cases}
      \\
      \intertext{幺正变换}
      \hat{T}_{\Lambda}(t)=\e^{\i \frac{q}{\hbar} \Lambda(\hat{R},t)}
      \\
      \ket{\psi'(t)}=\hat{T}_{\Lambda}(t) \ket{\psi(t)}
      \\
      \intertext{真实物理量对应的算符满足}
      \hat{G}'(t)=\hat{T}_{\Lambda}(t) G(t) \hat{T}_{\Lambda}^{\dagger}(t)
   \end{gather*}
   而非物理量不满足上式 \\
   在规范变换下,概率密度与概率流不变
\end{enumerate}

\section{Schrödinger方程的传播函数}

\begin{gather*}
   \psi(\bm{r}_2,t_2)=\int K(\bm{r_2},t_2;\bm{r_1},t_1) \psi(\bm{r}_1,t_1) \d^3 \bm{r},\quad t_2 > t_1
   \\
   \intertext{推迟传播函数}
   K(2,1)=K(\bm{r_2},t_2;\bm{r_1},t_1)=\inpro{\bm{r_2}}{\hat{U}(t_2,t_1) \vert \bm{r}_1}\theta(t_2-t_1)
\end{gather*}
$K(2,1)$表示以$\ket{\psi(t_1)}$为初态的粒子,在$t_2$时位于$\bm{r}_2$的概率幅

用保守系$\hat{H}$的特征态$\ket{\varphi_n}$表示$K(2,1)$
\begin{gather*}
   K(\bm{r_2},t_2;\bm{r_1},t_1)=\theta(t_2-t_1) \displaystyle \sum_n \varphi_n^*(\bm{r}_1) \varphi_n(\bm{r}_2) \e^{-\frac{\i}{\hbar} E_n(t_2-t_1)}
\end{gather*}
$K$满足
\begin{gather*}
   \begin{cases}
      [\i \hbar \pt{}{t_2}-\hat{H}(\bm{r}_2,\frac{\hbar}{\i}\nabla_2)] K(\bm{r_2},t_2;\bm{r_1},t_1)= \i \hbar \delta(t_2-t_1) \delta^3(\bm{r}_2-\bm{r}_1)
      \\
      K(\bm{r_2},t_2;\bm{r_1},t_1)=0,\quad \text{若}t_2<t_1
   \end{cases}
\end{gather*}

量子力学的Lagrange表述 \\
在$(t_1,t_2)$插入$N$个点$(t_{\alpha_i},\bm{r}_{\alpha_i})$,当$N \to +\infty$时确定了一条两端点确定的时空路径$\bm{r}(t)$
\begin{gather*}
   \intertext{传播函数}
   K(2,1)=\lim_{N \to +\infty} \int \d^3 \bm{r}_{\alpha_N} \int \d^3 \bm{r}_{\alpha_{N-1}} \cdots \int \d^3 \bm{r}_{\alpha_1} K(2,\alpha_N) K(\alpha_N,\alpha_{N-1}) \cdots K(\alpha_1,1)
\end{gather*}

Feymann假设
\begin{itemize}
   \item $1,2$之间每一条路径都对应一个概率幅,$K(2,1)$为概率幅总和
   \item 对于路径$\Gamma$,经典作用量
   \begin{gather*}
      S_{\Gamma}=\int_{\Gamma} L(\bm{r},\bm{p},t) \d t
      \\
      K_{\Gamma}(2,1)=C \e^{\frac{\i}{\hbar} S_{\Gamma}}
   \end{gather*}
   在$\Delta S_{\Gamma} \ll \hbar$情况下,大部分路径相位变化迅速,对$K(2,1)$贡献抵消,但对于$\delta S_{\Gamma}=0$的路径,$K_{\Gamma}(2,1)$相位稳定,即为最小作用量原理确定的轨道
\end{itemize}

\section{不稳定性}

氢原子激发态不稳定性来源于原子与电磁场合起来构成孤立体系,由于耦合很弱(为精细结构常数$\alpha$量级),在求解定态时可以忽略电磁场,但在稳定性问题上需要考虑电磁场

初态为非稳态$\ket{\varphi_n}$,$t$时刻仍处于$\ket{\varphi_n}$的概率
\begin{gather*}
   P(t)=\e^{-\frac{t}{\tau}}
   \\
   \intertext{$N$个粒子的全同体系}
   \dt{N}{t}=-\frac{N}{\tau}
   \\
   \intertext{体系脱离非稳态的概率}
   \d P = \frac{\d t}{\tau}
\end{gather*}
$\tau$为平均停留时间,能级有自然展宽
\begin{gather*}
   \Delta E \approx \frac{\hbar}{\tau}
\end{gather*}
非稳定性唯象描述:将能量特征值替换为复数
\begin{gather*}
   E'_n=E_n-\i \frac{\hbar}{2\tau}
\end{gather*}
态矢量的模不为$1$,因为引入了非Hermite的Hamiltonian

\section{一维问题的普遍讨论}

\subsection{束缚态}

能量是离散谱

波函数的平方可积性决定特征态$E<0$,且只能取离散谱,设势阱$V(x)$最小值为$-V_0$,
\begin{align*}
   E&=\langle T \rangle + \langle V \rangle 
   \\
   &> \langle V \rangle
   \\
   &= \int_{-\infty}^{+ \infty} |\varphi(x)|^2 V(x) \d x
   \\
   &\geq \int_{-\infty}^{+ \infty} -V_0|\varphi(x)|^2 \d x
   \\
   &=-V_0
\end{align*}
综上,
\begin{gather*}
   -V_0<E<0
\end{gather*}
与经典力学不同,$E$的可能取值是离散的,且无法取到$-V_0$

\subsection{散射态}

能量是连续谱

对于$[-\frac{l}{2},\frac{l}{2}]$上的势$V(x)$,考虑特征值$E$对应的定态$\varphi_k(x)$,
\begin{gather*}
   (\dt{^2}{x^2}+\frac{2m}{\hbar^2} (E-V(x)))\varphi_k(x)=0
   \\
   \intertext{设}
   k=\sqrt{\frac{2mE}{\hbar^2}}
   \\
   \varphi_k(x)=
   \begin{cases}
      A\e^{\i kx}+A'\e^{-\i kx},\quad x<-\frac{l}{2} \\
      \tilde{A}\e^{\i kx}+\tilde{A}'\e^{-\i kx},\quad x>\frac{l}{2}
   \end{cases}
   \\
   \intertext{透射矩阵}
   M(k)=\begin{bmatrix}
      F(k) & F'(k) \\
      G(k) & G'(k)
   \end{bmatrix}
   \\
   \begin{bmatrix}
      \tilde{A} \\
      \tilde{A}'
   \end{bmatrix}
   =M(k)
   \begin{bmatrix}
      A \\
      A'
   \end{bmatrix}
   \\
   \begin{cases}
      \e^{\i kx} \to F(k)\e^{\i kx}+G(k)\e^{-\i kx} \\
      \e^{-\i kx} \to F'(k)\e^{\i kx}+G'(k)\e^{-\i kx}
   \end{cases}
   \\
   \implies \begin{cases}
      F^*(k)=G(k) \\
      G^*(k)=F(k)
   \end{cases}
   \\
   M(k)=\begin{bmatrix}
      F(k) & G^*(k) \\
      G(k) & F^*(k)
   \end{bmatrix}
   \\
   \intertext{由概率守恒}
   \mathrm{det}M(k)=|F(k)|^2-|G(k)|^2=1
\end{gather*}
若$V(x)$是偶函数,则$G(k)$是纯虚数

引入矩阵$S(k)$,
\begin{gather*}
   \begin{bmatrix}
      \tilde{A} \\
      A'
   \end{bmatrix}
   =S(k)
   \begin{bmatrix}
      A \\
      \tilde{A}'
   \end{bmatrix}
   \\
   S(k)=\frac{1}{F(k)}
   \begin{bmatrix}
      1 & G^*(k) \\
      -G^*(k) & 1
   \end{bmatrix}
\end{gather*}
$S(k)$为幺正矩阵

无论从哪个方向入射,有
\begin{gather*}
   R=|\frac{G(k)}{F(k)}|^2
   \\
   T=\frac{1}{|F(k)|^2}
\end{gather*}

\subsection{周期势}

设有$N$个周期势,中心位于$x=0,l,\cdots,(N-1)l$处,周期长度$l$
\begin{gather*}
   \intertext{引入}
   \alpha=\sqrt{\frac{2mE}{\hbar^2}}
   \\
   (\dt{^2}{x^2}+\frac{2m}{\hbar^2} (E-V(x)))\varphi_\alpha(x)=0
   \\
   \varphi_{\alpha}=
   \begin{cases}
      A_0 \e^{\i \alpha x} + A'_0 \e^{-\i \alpha x},\quad x \leq -\frac{l}{2} \\
      A_n v_{\alpha}(x-(n-1)l) + A'_n v'_{\alpha}(x-(n-1)l),\quad (n-1)l-\frac{l}{2} \leq x \leq (n-1)l+\frac{l}{2},\quad n=1,2,\cdots,N \\
      C_0 \e^{\i \alpha (x-(N-1)l)} + C'_0 \e^{-\i \alpha (x-(N-1)l)},\quad x \leq -\frac{l}{2},\quad x \geq (N-1)l+\frac{l}{2}
   \end{cases}
   \\
   \intertext{则有}
   \begin{bmatrix}
      A_{n+1} \\
      A'_{n+1}
   \end{bmatrix}
   =Q(\alpha)
   \begin{bmatrix}
      A_{n} \\
      A'_{n}
   \end{bmatrix}
   \\
   \begin{bmatrix}
      C_0 \\
      C'_0
   \end{bmatrix}
   =M(\alpha)[Q(\alpha)]^{N-1}
   \begin{bmatrix}
      A_{0} \\
      A'_{0}
   \end{bmatrix}
   \\
   \intertext{其中}
   Q(\alpha)=\begin{bmatrix}
      \e^{\i \alpha l} & 0 \\
      0 & \e^{-\i \alpha l}
   \end{bmatrix}
   M(\alpha)
   \\
   =\begin{bmatrix}
      \e^{\i \alpha l}F(k) & \e^{\i \alpha l}G^*(k) \\
      \e^{-\i \alpha l}G(k) & \e^{-\i \alpha l}F^*(k)
   \end{bmatrix}
   \\
   \intertext{特征方程}
   \lambda^2-2X(\alpha)\lambda+1=0
   \\
   \intertext{其中}
   X(\alpha)=\Re (\e^{\i \alpha l}F(\alpha))
\end{gather*}
\begin{itemize}
   \item 若$|X(\alpha)|\leq 1$,称为容许能带
   \begin{gather*}
      X(\alpha)=\cos(k(\alpha)l),\quad 0 \leq k(\alpha) \leq \frac{\pi}{l}
      \\
      \lambda=\e^{\pm \i k(\alpha) l}
   \end{gather*}
   \item 若$|X(\alpha)|\geq 1$,称为禁止能带
   \begin{gather*}
      X(\alpha)=\epsilon \cosh(\rho(\alpha)l),\quad \rho(\alpha) \geq 0
      \\
      \epsilon=\mathrm{sgn} X(\alpha)
      \\
      \lambda=\epsilon \e^{\pm \rho(\alpha)l}
   \end{gather*}
   当$N$足够大时,总透射系数指数减小,透射波相干相消,反射波相干相长 \\
   当$F(\alpha) \to 1$时,禁止能带非常狭窄,趋于孤立能级,得到反射的Bragg条件$l=n\frac{\lambda}{2}$
\end{itemize}

考虑如下情况,当$x \to \infty,V(x) \to V_e$,
\begin{gather*}
   \intertext{按特征值分解,}
   \begin{cases}
      A_n=f_1(\alpha)\lambda_1^{n-1}+f_2(\alpha)\lambda_2^{n-1} \\
      A'_n=f'_1(\alpha)\lambda_1^{n-1}+f'_2(\alpha)\lambda_2^{n-1}
   \end{cases}
   \intertext{在左侧远处有}
   \varphi_{\alpha}(x)=B\e^{\mu(\alpha)x}
   \\
   \mu(\alpha)=\sqrt{\frac{2m}{\hbar^2}(V_e-E)}
   \\
   \intertext{由概率流守恒,}
   \frac{A_1}{A'_1}=\e^{\i \chi(\alpha)}
   \\
   \frac{A_{N+1}}{A'_{N+1}}=\e^{\i \chi'(\alpha)}
   \\
   \intertext{取定}
   \begin{cases}
      A_1=\e^{\i \frac{\chi(\alpha)}{2}}
      \\
      A_1'=\e^{\i \frac{-\chi(\alpha)}{2}}
   \end{cases}
   \\
   A'_n=A_n^*   
\end{gather*}
\begin{itemize}
   \item 容许能带
   \begin{gather*}
      \begin{cases}
         A_n=f_1(\alpha)\e^{\i (n-1)k(\alpha)l}+f_2(\alpha)\e^{-\i (n-1)k(\alpha)l} \\
         A'_n=A_n^*
      \end{cases}
      \\
      \intertext{得到}
      \frac{f_1(\alpha)\e^{2\i N k(\alpha)l}+f_2(\alpha)}{f_2^*(\alpha)\e^{2\i N k(\alpha)l}+f_1^*(\alpha)}=\e^{\i \chi'(\alpha)}
      \\
      \intertext{设}
      \theta(\alpha)=\mathrm{Arg}\frac{f_1^*(\alpha)\e^{\i \frac{\chi'(\alpha)}{2}}-f_2(\alpha)\e^{-\i \frac{\chi'(\alpha)}{2}}}{f_1(\alpha)\e^{-\i \frac{\chi'(\alpha)}{2}}-f_2^*(\alpha)\e^{\i \frac{\chi'(\alpha)}{2}}}
      \\
      \intertext{得到关于能量的方程}
      k(\alpha)=\frac{\theta(\alpha)}{2Nl}+p\frac{\pi}{Nl},\quad p=0,1,\cdots,(N-1)
   \end{gather*}
   当$N \gg 1$时,从离散能级变为连续能带,这时$\theta(\alpha)$所表征的边缘效应可忽略\\
   若$T(\alpha)=0$,$|F(\alpha)| \gg 1,|G(\alpha)| \gg 1$,容许能带变为孤立能级
   \item 禁止能带
   \begin{gather*}
      \begin{cases}
         A_n=\epsilon^{n-1}(f_1(\alpha)\e^{(n-1)\rho(\alpha)l}+f_2(\alpha)\e^{-(n-1)\rho(\alpha)l}) \\
         A'_n=A_n^*
      \end{cases}
      \\
      \intertext{得到}
      \frac{f_1(\alpha)+f_2(\alpha)\e^{-2N\rho(\alpha)l}}{f_1^*(\alpha)+f_2^*(\alpha)\e^{-2N\rho(\alpha)l}}=\e^{\i \chi'(\alpha)}
      \\
      \intertext{设}
      L(\alpha)=-\frac{f_1^*(\alpha)\e^{\i \frac{\chi'(\alpha)}{2}}-f_1(\alpha)\e^{-\i \frac{\chi'(\alpha)}{2}}}{f_2^*(\alpha)\e^{\i \frac{\chi'(\alpha)}{2}}-f_2(\alpha)\e^{-\i \frac{\chi'(\alpha)}{2}}}
   \end{gather*}
   当$N \gg 1$时,$L(\alpha)=0$,能级与边缘效应有关,$(A_1,A'_1)$与某个$\lambda$对应的特征向量成正比,波函数定域在边缘,且禁止能带中只有有限个能级,可以认为所有能级都在容许能带中
\end{itemize}