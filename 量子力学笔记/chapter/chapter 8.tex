\chapter{势场中的散射的初等量子理论}

物理实验中,通过探测碰撞后末态粒子来研究粒子的相互作用
\begin{itemize}
    \item 重排碰撞:基本粒子重新组合
    \item 相对论能量的“物质化”:产生新粒子
    \item 散射:粒子种类没有变化
    \begin{itemize}
        \item 弹性散射:粒子的内部状态没有变化
    \end{itemize}
\end{itemize}

本章讨论第一类粒子入射到作为靶的第二类粒子弹性的散射,引入简化假设
\begin{itemize}
    \item 两类粒子都没有自旋
    \item 不考虑两类粒子的内部结构
    \item 靶充分薄,忽略多重散射
    \item 忽略各个靶粒子产生散射波的相干性,即第一类粒子波包的展延度小于第二类粒子的平均间距
    \item 两类粒子的相互作用势能可以表示为$V(\bm{r}_1-\bm{r}_2)$,转化到质心系中约化粒子在势场$V(\bm{r})$中的散射
\end{itemize}

入射粒子通量$F_i$,在$(\theta,\phi)$方向单位时间探测到粒子数
\begin{gather*}
    \d n=F_i \sigma(\theta,\phi) \d \Omega
\end{gather*}
$\sigma(\theta,\phi)$称为有效微分散射截面,散射截面
\begin{gather*}
    \sigma=\int \sigma(\theta,\phi) \d \Omega
\end{gather*}

\section{散射定态}

\subsection{定义}

\begin{gather*}
    \psi(\bm{r},t)=\varphi(\bm{r})\e^{-\frac{\i}{\hbar} Et}
    \\
    \intertext{定态Schrödinger方程}
    \left[ -\frac{\hbar^2}{2\mu} \nabla^2+V(\bm{r}) \right] \varphi(\bm{r})=E\varphi(\bm{r})
    \\
    \intertext{设}
    \begin{cases}
        E=\frac{\hbar^2 k^2}{2\mu}
        \\
        V(\bm{r})=\frac{\hbar^2}{2\mu}U(\bm{r})
    \end{cases}
    \\
    \intertext{得到}
    \left[ \nabla^2+k^2 - U(\bm{r}) \right] \varphi(\bm{r})=0
\end{gather*}
对于$k$的每一个值都是无穷重简并的,满足特定条件的特征态称为散射的定态,波函数$v_k^{(\text{diff})}(\bm{r})$
\begin{gather*}
    v_k^{(\text{diff})}(\bm{r}) \approx \e^{\i kz}+f_k(\theta,\phi)\frac{\e^{\i kr}}{r},\quad r \to +\infty
\end{gather*}

\subsection{概率流}

散射定态对应稳恒流动的概率流
\begin{gather*}
    \intertext{概率流密度}
    \bm{J}(\bm{r})\frac{1}{\mu} \Re(\varphi(\bm{r})^* \frac{\hbar}{\i} \nabla \varphi(\bm{r}))
    \\
    \intertext{入射流}
    \bm{J}_i=\frac{\hbar k}{\mu} \bm{e}_z
    \\
    \intertext{散射流}
    (\bm{J}_d)_r=\frac{\hbar k}{\mu r^2}|f_k(\theta,\phi)|^2
    \\
    (\bm{J}_d)_{\theta}=\frac{\hbar}{\mu r^3} \Re\left[\frac{1}{\i} f_k^*(\theta,\phi)\pt{}{\theta}f_k(\theta,\phi)\right]
    \\
    (\bm{J}_d)_{\phi}=\frac{\hbar}{\mu r^3 \sin\theta} \Re\left[\frac{1}{\i} f_k^*(\theta,\phi)\pt{}{\phi}f_k(\theta,\phi)\right]
\end{gather*}
由于$r \to +\infty$,可以忽略$(\bm{J}_d)_{\theta}$和$(\bm{J}_d)_{\phi}$,认为$\bm{J}_d$沿径向

\begin{gather*}
    \intertext{入射通量}
    F_i=C\frac{\hbar k}{\mu}
    \\
    \d n=C\frac{\hbar k}{\mu} |f_k(\theta,\phi)|^2 \d \Omega
    \\
    \sigma(\theta,\phi)=|f_k(\theta,\phi)|^2
\end{gather*}
除了$\theta=0$方向,入射波和散射波的干涉项不存在

\subsection{积分方程}

\begin{gather*}
    (\nabla^2+k^2) \varphi_0(\bm{r})=0
    \\
    (\nabla^2+k^2) G(\bm{r})=\delta^3(\bm{r})
    \\
    \varphi(\bm{r})=\varphi_0(\bm{r})+\int G_+(\bm{r}-\bm{r'})U(\bm{r'})\varphi(\bm{r'}) \d V'
    \\
    \intertext{Green函数}
    G_{\pm}(\bm{r})=-\frac{1}{4\pi} \frac{\e^{\pm \i kr}}{r}
    \\
    v_k^{(\text{diff})}(\bm{r})=\e^{\i kz}+\int G_+(\bm{r}-\bm{r'})U(\bm{r'})v_k^{(\text{diff})}(\bm{r'}) \d V'
\end{gather*}
对于远场区域,设势场作用范围$L$,有
\begin{gather*}
    r\gg L,\quad r' \leq L
    \\
    |\bm{r}-\bm{r'}| \approx r-\bm{n} \cdot \bm{r'}
\end{gather*}
得到
\begin{gather*}
    v_k^{(\text{diff})}(\bm{r}) \approx \e^{\i kz}-\frac{1}{4\pi} \frac{\e^{\i kr}}{r} \int \e^{-\i k \bm{n} \cdot \bm{r'}}U(\bm{r'})v_k^{(\text{diff})}(\bm{r'}) \d V'
    \\
    f_k(\theta,\phi)=-\frac{1}{4\pi} \int \e^{-\i k \bm{n} \cdot \bm{r'}}U(\bm{r'})v_k^{(\text{diff})}(\bm{r'}) \d V' 
\end{gather*}

\subsection{Born近似}

\begin{gather*}
    \intertext{定义入射波波矢$\bm{k}_i$,散射波波矢}
    \bm{k}_d=k\bm{n}
    \\
    \intertext{转移波矢}
    \bm{K}=\bm{k}_d-\bm{k}_i
    \\
    v_k^{(\text{diff})}(\bm{r})=\e^{\i \bm{k}_i \cdot \bm{r}}+\int G_+(\bm{r}-\bm{r'})U(\bm{r'})v_k^{(\text{diff})}(\bm{r'}) \d V'
    \\
    v_k^{(\text{diff})}(\bm{r'})=\e^{\i \bm{k}_i \cdot \bm{r'}}+\int G_+(\bm{r'}-\bm{r''})U(\bm{r''})v_k^{(\text{diff})}(\bm{r''}) \d V''
\end{gather*}
类似可以不断迭代展开,取最低次项,
\begin{gather*}
    v_k^{(\text{diff})}(\bm{r})=\e^{\i \bm{k}_i \cdot \bm{r}}+\int G_+(\bm{r}-\bm{r'})U(\bm{r'})\e^{\i \bm{k}_i \cdot \bm{r'}}\d V'
    \\
    f_k^{(B)}=-\frac{1}{4\pi}\int \e^{-\i \bm{K} \cdot \bm{r'}}U(\bm{r'}) \d V'
    \\
    \sigma_k^{(B)}=\frac{\mu^2}{4\pi^2 \hbar^4}\left|\int \e^{-\i \bm{K} \cdot \bm{r}}V(\bm{r}) \d V\right|^2
\end{gather*}

\section{中心势场:分波法}

分波:$\hat{H}$、$\hat{\bm{L}}^2$、$\hat{L}_z$的共同特征态对应的波函数$\varphi(\bm{r})$

\subsection{自由粒子}

\begin{gather*}
    \intertext{自由粒子Hamiltonian}
    \hat{H}_0=\frac{1}{2\mu} \hat{\bm{P}}^2
\end{gather*}

\begin{enumerate}
    \item 平面波:选取ECOC$\hat{H}_0$和$\hat{\bm{P}}$确定的态空间
    \begin{gather*}
        \begin{cases}
            \bm{P} \ket{\bm{p}}=\bm{p} \ket{\bm{p}}
            \\
            \hat{H}_0 \ket{\bm{p}}=\frac{\bm{p}^2}{2\mu} \ket{\bm{p}}
            \\
            \inpro{\bm{r}}{\bm{p}}=(\frac{1}{2\pi\hbar})^{\frac{3}{2}} \e^{\frac{\i}{\hbar} \bm{p} \cdot \bm{r}}
        \end{cases}
        \\
        \begin{cases}
            \bm{P} \ket{\bm{k}}=\hbar\bm{k} \ket{\bm{k}}
            \\
            \hat{H}_0 \ket{\bm{k}}=\frac{\hbar^2 \bm{k}^2}{2\mu} \ket{\bm{k}}
            \\
            \inpro{\bm{r}}{\bm{k}}=(\frac{1}{2\pi})^{\frac{3}{2}} \e^{\i \bm{k} \cdot \bm{r}}
        \end{cases}
    \end{gather*}
    \item 球面波:选取ECOC$\hat{H}$、$\hat{\bm{L}}^2$和$\hat{L}_z$确定的态空间
    \begin{gather*}
        \varphi_{klm}^{(0)}(\bm{r})=\sqrt{\frac{2k^2}{\pi}} j_l(kr)Y_{lm}(\theta,\phi)
        \\
        \inpro{\varphi_{klm}^{(0)}}{\varphi_{k'l'm'}^{(0)}}=\delta(k-k')\delta_{ll'}\delta_{mm'}
        \\
        \int_{0}^{+\infty} \displaystyle \sum_{l=0}^{+\infty} \sum_{m=-l}^{l} \ket{\varphi_{klm}^{(0)}} \bra{\varphi_{klm}^{(0)}} \d k=\hat{I}
    \end{gather*}
    \begin{enumerate}
        \item 原点附近
        \begin{gather*}
            j_l(\rho) \approx \frac{\rho^l}{(2l+1)!!},\quad \rho \to 0
        \end{gather*}
        当$r<\frac{1}{k}\sqrt{l(l+1)}$时,概率接近$0$,也就是定态下粒子对半径$b_k=\frac{1}{k}\sqrt{l(l+1)}$内的过程无反应,经典意义下$b_k=\frac{L}{p}$为碰撞参数
        \item 渐近行为
        \begin{gather*}
            \varphi_{klm}^{(0)}(r,\theta,\phi) \approx -\sqrt{\frac{2k^2}{\pi}} Y_{lm}(\theta,\phi)\frac{\e^{-\i kr}\e^{\i l\frac{\pi}{2}}-\e^{\i kr}\e^{-\i l\frac{\pi}{2}}}{2\i kr},\quad r \to +\infty
        \end{gather*}
        为向内球面波与向外球面波的叠加,相位差$l\pi$
    \end{enumerate}
    \item 平面波按球面波展开
    \begin{align*}
        \e^{\i kz}&=\displaystyle \sum_{l=0}^{+\infty} \i^l \sqrt{4\pi(2l+1)} j_l(kr)Y_{l,0}(\theta,\phi)
        \\
        &=\displaystyle \sum_{l=0}^{+\infty} \i^l (2l+1) j_l(kr) P_l(\cos\theta)
    \end{align*}
\end{enumerate}

\subsection{势场中的分波}

\begin{gather*}
    \varphi_{klm}(\bm{r})=R_{kl}(r)Y_{lm}(\theta,\phi)=\frac{1}{r} u_{kl}(r)Y_{lm}(\theta,\phi)
    \\
    \intertext{$u_{kl}(r)$满足}
    \begin{cases}
        (-\frac{\hbar^2}{2m}\dt{^2}{r^2}+\frac{l(l+1)\hbar^2}{2mr^2} + V(r))u_{kl}(r)=E_{kl} u_{kl}(r)
        \\
        u_{kl}(0)=0
    \end{cases}
\end{gather*}
$r \to +\infty$,方程化为
\begin{gather*}
    \left(\dt{^2}{r^2}+k^2\right)u_{kl}(r) =0
    \\
    u_{kl}(r)=C\sin\left(kr-l\frac{\pi}{2}+\delta_l\right)
    \\
    \varphi_{klm}(\bm{r}) \approx -C Y_{lm}(\theta,\phi)\frac{\e^{-\i kr}\e^{\i (l\frac{\pi}{2}+\delta_l)}-\e^{\i kr}\e^{-\i (l\frac{\pi}{2}-\delta_l)}}{2\i kr}
    \\
    \intertext{改写为}
    \tilde{\varphi}_{klm}(\bm{r}) \approx -Y_{lm}(\theta,\phi)\frac{\e^{-\i kr}\e^{\i l\frac{\pi}{2}}-\e^{\i kr}\e^{-\i l\frac{\pi}{2}}\e^{2\i \delta_l}}{2\i kr}
\end{gather*}

\subsection{散射定态}

由于中心势场下关于$z$轴对称,只包含$m=0$的分波,
\begin{gather*}
    v_k^{(\text{diff})}(\bm{r})=\displaystyle \sum_{l=0}^{+\infty} \i^l \sqrt{4\pi(2l+1)} \tilde{\varphi}_{k,l,0}(\bm{r})
\end{gather*}
\begin{proof}
    \begin{align*}
        \e^{2\i \delta_l}&=1+2\i \e^{\i \delta_l} \sin\delta_l
        \\
        \displaystyle \sum_{l=0}^{+\infty} \i^l \sqrt{4\pi(2l+1)} \tilde{\varphi}_{k,l,0}(\bm{r})&=-\displaystyle \sum_{l=0}^{+\infty} \i^l \sqrt{4\pi(2l+1)} \left[\frac{\e^{-\i kr}\e^{\i l\frac{\pi}{2}}-\e^{\i kr}\e^{-\i l\frac{\pi}{2}}}{2\i kr}-\frac{\e^{\i kr}}{r} \frac{1}{k} \e^{-\i l\frac{\pi}{2}}\e^{\i \delta_l}\sin\delta_l\right]
        \\
        &=\e^{\i kz}+f_k(\theta)\frac{\e^{\i kr}}{r}
        \\
        f_k(\theta)&=\frac{1}{k} \displaystyle \sum_{l=0}^{+\infty} \i^l \sqrt{4\pi(2l+1)} \e^{\i \delta_l} \sin\delta_l Y_{l,0}(\theta)
    \end{align*}
\end{proof}
\begin{gather*}
    \intertext{有效微分散射截面}
    \sigma(\theta)=\frac{1}{k^2} \left|\displaystyle \sum_{l=0}^{+\infty} \sqrt{4\pi(2l+1)} \e^{\i \delta_l} \sin\delta_l Y_{l,0}(\theta)\right|^2
    \\
    \intertext{散射截面}
    \sigma=\frac{4\pi}{k^2} \displaystyle \sum_{l=0}^{+\infty} (2l+1) \sin^2 \delta_l
\end{gather*}