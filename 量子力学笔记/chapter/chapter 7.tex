\chapter{中心势场中的粒子}

\section{定态} \label{sec:7.1}

类比经典力学,在$\{\ket{\bm{r}}\}$表象下,定态Schrödinger方程可以写成
\begin{gather*}
   -\frac{\hbar^2}{2mr}\pt{^2}{r^2}(r\varphi)+\frac{1}{2mr^2} \hat{\bm{L}}^2 \varphi + V(r)\varphi=E \varphi
\end{gather*}
类比角动量守恒,
\begin{gather*}
   [\hat{H},\hat{\bm{L}}]=0
\end{gather*}
$\hat{H}$、$\hat{\bm{L}}^2$、$\hat{L}_z$构成一组ECOC,再引入一个指标$k$,由第六章讨论,$\hat{H}$仅与$n,l$有关。选取共同特征函数构成态空间的一组正交归一基,待求解的特征方程组
\begin{gather*}
   \begin{cases}
      H \varphi(\bm{r})=E \varphi(\bm{r})
      \\
      \hat{\bm{L}}^2 \varphi(\bm{r})=l(l+1)\hbar^2 \varphi(\bm{r})
      \\
      \hat{L}_z \varphi(\bm{r})=m\hbar \varphi(\bm{r})
   \end{cases}
   \\
   \varphi_{klm}(\bm{r})=R_{kl}(r)Y_{lm}(\theta,\phi)=\frac{1}{r} u_{kl}(r)Y_{lm}(\theta,\phi)
\end{gather*}
径向方程化简为
\begin{gather*}
   (-\frac{\hbar^2}{2m}\dt{^2}{r^2}+\frac{l(l+1)\hbar^2}{2mr^2} + V(r))u_{kl}(r)=E_{kl} u_{kl}(r)
\end{gather*}
为了使$r \to 0$,$\varphi(\bm{r}) \to 0$,有
\begin{gather*}
   R_{kl}(r) \propto r^{l}
   \\
   R_{kl}(0)=0
\end{gather*}
这使得对于每个$E_{kl}$,径向方程最多只有一个合理解 \\
平方可积性
\begin{gather*}
   \int_{0}^{+\infty} |R_{kl}(r)|^2 r^2 \d r=1
   \\
   \intertext{或对于连续谱,}
   \int_{0}^{+\infty} R^*_{k'l'}(r) R_{kl}(r)r^2 \d r=\delta(k'-k)
\end{gather*}
简并度
\begin{itemize}
   \item 实质性简并:同一个$E_{kl}$,$m$可以取$2l+1$个不同的值
   \item 偶然性简并:对于某些势场,存在$E_{kl}=E_{k'l'},k' \neq k,l' \neq l$
\end{itemize}

\section{二体运动}

两个粒子构成的孤立体系,相互作用势能$V(|\bm{r}_1-\bm{r}_2|)$只与两粒子距离有关

\begin{enumerate}
   \item 经典力学 \\
   引入两个假想粒子,质量、位置、动量分别为
   \begin{gather*}
      \begin{cases}
         M=m_1+m_2
         \\
         \bm{r}_C=\frac{m_1 \bm{r}_1 + m_2 \bm{r}_2}{m_1+m_2}
         \\
         \bm{p}_C=\bm{p}_1+\bm{p}_2
      \end{cases}
      \\
      \begin{cases}
         \mu=\frac{m_1 m_2}{m_1+m_2}
         \\
         \bm{r}=\bm{r}_1-\bm{r}_2
         \\
         \bm{p}=\frac{m_2 \bm{p}_1 - m_1 \bm{p}_2}{m_1+m_2}
      \end{cases}
   \end{gather*}
   则Hamiltonian化为
   \begin{gather*}
      H=\frac{\bm{p}_C^2}{2M}+\frac{\bm{p}^2}{2\mu}+V(r)
      \\
      \intertext{动力学方程}
      \begin{cases}
         \dot{\bm{p}_C}=0
         \\
         \dot{\bm{p}}=-\nabla V(r)
      \end{cases}
   \end{gather*}
   在质心系中,只考虑约化粒子的运动即可
   \item 量子力学 \\
   类比经典力学,可以将Hamiltonian分为两部分,态空间表示为两个假想粒子态空间的张量积
   \begin{gather*}
      \hat{H}=\hat{H}_C+\hat{H}_r
      \\
      \begin{cases}
         \hat{H}_C=\frac{\hat{\bm{P}}_C^2}{2M}
         \\
         \hat{H}_r=\frac{\hat{\bm{P}}^2}{2\mu}+\hat{V}(\hat{R})
      \end{cases}
      [\hat{H}_C,\hat{H}_r]=0
      \\
      \begin{cases}
         \hat{H}_C \ket{\chi_C}=E_C \ket{\chi_C}
         \\
         \hat{H}_r \ket{\chi_r}=E_r \ket{\chi_r}
      \end{cases}
      \\
      \intertext{在$\{\ket{\bm{r}_C}\}$和$\{\ket{\bm{r}}\}$表象下写出方程}
      \begin{cases}
         -\frac{\hbar^2}{2M} \nabla_C^2 \chi_C(\bm{r}_C)=E_C \chi_C(\bm{r}_C)
         \\
         (-\frac{\hbar^2}{2\mu} \nabla^2 + V(r))\chi_r(\bm{r})=E_r \chi_r(\bm{r})
      \end{cases}
   \end{gather*}
   质心的波函数为平面波,
   \begin{gather*}
      \chi_C(\bm{r}_C)=\frac{1}{(2\pi \hbar)^{\frac{3}{2}}} \e^{\frac{\i}{\hbar} \bm{p}_C \cdot \bm{r}_C}
      \\
      E_C=\frac{\bm{p}_C^2}{2M}
   \end{gather*}
   相对运动方程与中心势场中的粒子相同
\end{enumerate}

\section{氢原子}

\begin{gather*}
   V(r)=-\frac{e^2}{4\pi \epsilon_0 r}
   \\
   \mu=\frac{m_p m_e}{m_p + m_e} \approx m_e
\end{gather*}

\subsection{Bohr模型}

\begin{gather*}
   E=\frac{1}{2}\mu v^2-\frac{e^2}{4\pi \epsilon_0 r}
   \\
   \mu \frac{v^2}{r}=\frac{e^2}{4\pi \epsilon_0 r^2}
   \\
   \mu vr=n\hbar,\quad n \in \mathbb{Z}^+
   \\
   \intertext{设电离能}
   E_I=\frac{1}{2} \mu (\frac{e^2}{4\pi \epsilon_0 \hbar})^2
   \\
   \intertext{Bohr半径}
   a=\frac{4\pi \epsilon_0 \hbar^2}{\mu e^2}
   \\
   \intertext{解得}
   E_n=-\frac{E_I}{n^2}
   \\
   r_n=n^2 a
   \\
   v_n=\frac{1}{n} \frac{e^2}{4\pi \epsilon_0 \hbar}
\end{gather*}

\subsection{量子力学}

由\ref{sec:7.1},只需关注径向方程,
\begin{gather*}
   \begin{cases}
      (-\frac{\hbar^2}{2m}\dt{^2}{r^2}+\frac{l(l+1)\hbar^2}{2mr^2}-\frac{e^2}{4\pi \epsilon_0 r})u_{kl}(r)=E_{kl} u_{kl}(r)
      \\
      u_{kl}(0)=0
   \end{cases}
   \\
   \intertext{无量纲化}
   \begin{cases}
      \rho=\frac{r}{a}
      \\
      \lambda_{kl}=\sqrt{-\frac{E_{kl}}{E_I}}
   \end{cases}
   \\
   \dt{^2 u}{\rho^2} - \frac{l(l+1)}{\rho^2} u + \frac{2}{\rho} u -\lambda_{kl}^2 u=0
   \\
   \intertext{$\rho \to +\infty$,渐近行为}
   u_{kl}(\rho) \to \e^{\pm \lambda_{kl} \rho}
   \\
   \intertext{舍去正指数解,设}
   u_{kl}(\rho)=\e^{-\lambda_{kl} \rho} y_{kl}(\rho)
   \\
   \intertext{幂级数法,设}
   y_{kl}(\rho)=\rho^{s} \displaystyle \sum_{q=0}^{+\infty} c_q \rho^q
   \\
   \intertext{解得}
   \begin{cases}
      s=l+1
      \\
      c_q=\frac{2((q+l)\lambda_{kl}-1)}{q(q+2l+1)}c_{q-1}
   \end{cases}
   \\
   \intertext{由波函数可积性,幂级数截断为多项式}
   \exists q=k, \lambda_{kl}=\frac{1}{k+l}
   \\
   E_{kl}=-\frac{1}{2} \mu (\frac{e^2}{4\pi \epsilon_0 \hbar})^2 \frac{1}{(k+l)^2}
   \\
   c_q=(-\frac{2}{k+l})^q \frac{(k-1)!}{(k-q-1)!} \frac{(2l+1)!}{q!(q+2l+1)!} c_0
\end{gather*}

$k+l$同时出现,标记为新的量子数$n=k+l$
\begin{gather*}
   E_n=-\frac{1}{2} \mu (\frac{e^2}{4\pi \epsilon_0 \hbar})^2 \frac{1}{n^2}
   \\
   n \in \mathbb{Z}^+
   \\
   l=0,1,\cdots,n-1,
   \\
   m=-l,-l+1,\cdots,l
   \\
   \intertext{$E_n$简并度}
   g_n=n^2
   \\
   \intertext{定态波函数}
   \varphi_{nlm}=\sqrt{\left(\frac{2}{na}\right)^3\frac{(n-l-1)!}{2n(n+l)!}}\left(\frac{2r}{na}\right)^le^{-\frac{r}{na}}L_{n-l-1}^{(2l+1)}\left(\frac{2r}{na}\right)Y_{lm}(\theta,\varphi)
\end{gather*}
其中$L_n^{(\alpha)}(x)$为关联Laguerre多项式

\subsection{定态的概率流}

一般表达式
\begin{gather*}
   \psi(\bm{r})=\alpha(\bm{r}) \e^{\i \xi(\bm{r})},\quad \alpha(\bm{r}) \geq 0,0\leq \xi(\bm{r}) < 2\pi
   \\
   \rho(\bm{r})=\alpha^2(\bm{r})
   \\
   \bm{J}(\bm{r})=\bm{J}(\bm{r},t)=\frac{\hbar}{2\mu\i} (\psi^* \nabla \psi - \psi \nabla \psi^*)=\frac{\hbar}{\mu} \alpha^2(\bm{r}) \nabla \xi(\bm{r})
\end{gather*}
若已知概率密度和概率流,
\begin{gather*}
   \nabla \xi(\bm{r})=\frac{\mu}{\hbar} \frac{\bm{J}(\bm{r})}{\rho(\bm{r})}
   \\
   \intertext{只有满足下列条件时,波函数才有解}
   \nabla \times \frac{\bm{J}(\bm{r})}{\rho(\bm{r})}=0
\end{gather*}

对于氢原子的定态,
\begin{gather*}
   \bm{J}_{nlm}(\bm{r})=\frac{m\hbar}{\mu} \frac{\rho_{nlm}(\bm{r})}{r\sin\theta} \bm{e}_{\phi}
\end{gather*}
概率流密度可以类比为动量密度,角动量
\begin{gather*}
   L_z=\mu \int \bm{e}_z \cdot (\bm{r} \times \bm{J}_{nlm}(\bm{r})) \d^3 \bm{r}=m\hbar
\end{gather*}

磁场影响:加入沿$z$轴的匀强磁场$\bm{B}$,在规范$\bm{A}(\bm{r})=-\frac{1}{2} \bm{r} \times \bm{B}$下,磁场可以视为微扰,波函数的变化很小,
\begin{gather*}
   \bm{J}_{nlm}(\bm{r})=\frac{1}{\mu} \rho_{nlm}(\bm{r})(\hbar \nabla \xi_{nlm}(\bm{r}))-e \bm{A}(\bm{r})
   \\
   \bm{J}_{1,0,0}(\bm{r})=\frac{\omega_c}{2} \rho_{1,0,0}(\bm{r}) \bm{e}_z \times \bm{r}
\end{gather*}
与无磁场时不同,基态概率流密度不为$0$

\subsection{均匀磁场中的氢原子}

认为原子核是静止的,磁场是微扰

选取规范
\begin{gather*}
   \bm{A}(\bm{r})=-\frac{1}{2} \bm{r} \times \bm{B}
\end{gather*}
Hamiltonian
\begin{gather*}
   \hat{H}=\frac{1}{2m}(\hat{\bm{P}}-q\hat{\bm{A}}(\hat{\bm{R}}))^2+\hat{V}(\hat{\bm{R}})=\hat{H}_0+\hat{H}_1+\hat{H}_2
   \\
   \hat{H}_0=\frac{1}{2m}\hat{\bm{P}}^2+\hat{V}(\hat{\bm{R}})
   \\
   \hat{H}_1=-\frac{\mu_B}{\hbar} \hat{\bm{L}} \cdot \bm{B}
   \\
   \hat{H}_2=\frac{q^2 \bm{B}^2}{8m} \hat{\bm{R}}_{\perp}^2
   \\
   \intertext{Bohr磁子}
   \mu_B=\frac{q\hbar}{2m}
   \\
   \intertext{有}
   \frac{\Delta E_2}{\Delta E_1} \approx \frac{\Delta E_1}{\Delta E_0} \ll 1
\end{gather*}
顺磁项
\begin{gather*}
   \intertext{角动量对应磁矩}
   \bm{m}_1=\frac{q}{2m} \bm{L}
   \\
   \hat{H}_1=-\hat{\bm{m}}_1 \cdot \bm{B}
\end{gather*}
抗磁项
\begin{gather*}
   \intertext{磁矢势带来磁矩}
   \bm{m}_2=\frac{q}{2m}(-q \bm{r} \times \bm{A}(\bm{r}))=\frac{q^2}{4m}((\bm{r} \cdot \bm{B})\bm{r}-r^2 \bm{B})
   \\
   \intertext{对应能量}
   W_2=-\int_{0}^{B} \bm{m}(\bm{B'}) \cdot \d \bm{B'}=\frac{q^2}{8m} \bm{r}_{\perp}^2 \bm{B}^2
   \\
   \hat{H}_2 =\frac{q^2 \bm{B}^2}{8m} \hat{\bm{R}}_{\perp}^2
\end{gather*}
总磁矩与机械角动量有关,
\begin{gather*}
   \bm{m}=\frac{q}{2m} \bm{r} \times m\bm{v}=\bm{m}_1+\bm{m}_2
\end{gather*}

Zeeman效应:忽略$\hat{H}_2$,设$\hat{H}_0$特征向量$\ket{\varphi_{nlm}}$
\begin{gather*}
   \hat{H} \ket{\varphi_{nlm}}=(E_n-m\mu_B B)\ket{\varphi_{nlm}}
\end{gather*}
对于$1s$和$2p$能级之间的跃迁,
\begin{gather*}
   \hat{H}\ket{\varphi_{1,0,0}}=-E_I \ket{\varphi_{1,0,0}}
   \\
   \hat{H}\ket{\varphi_{2,1,m}}=(-E_I+\hbar (\Omega + m \omega_L))\ket{\varphi_{2,1,m}}
   \\
   \intertext{式中}
   \Omega=\frac{E_2-E_1}{\hbar}
   \\
   \intertext{Larmor角频率}
   \omega_L=-\frac{qB}{2m}
\end{gather*}
偶极辐射
\begin{gather*}
   \intertext{由对称性,}
   \inpro{\varphi_{1,0,0}}{\hat{\bm{D}} \vert \varphi_{1,0,0}}=\inpro{\varphi_{2,1,m'}}{\hat{\bm{D}} \vert \varphi_{2,1,m}}=0
   \\
   \intertext{设径向积分}
   \chi=\int_{0}^{+\infty} R_{2,1}(r)R_{1,0}(r) r^3 \d^3 r=\frac{128\sqrt{6}}{243}a
   \\
   \begin{cases}
      \inpro{\varphi_{2,1,1}}{\hat{D}_x \vert \varphi_{1,0,0}}=-\inpro{\varphi_{2,1,-1}}{\hat{D}_x \vert \varphi_{1,0,0}}=-\frac{q\chi}{\sqrt{6}}
      \\
      \inpro{\varphi_{2,1,0}}{\hat{D}_x \vert \varphi_{1,0,0}}=0
   \end{cases}
   \\
   \begin{cases}
      \inpro{\varphi_{2,1,1}}{\hat{D}_y \vert \varphi_{1,0,0}}=\inpro{\varphi_{2,1,-1}}{\hat{D}_y \vert \varphi_{1,0,0}}=-\frac{\i q\chi}{\sqrt{6}}
      \\
      \inpro{\varphi_{2,1,0}}{\hat{D}_y \vert \varphi_{1,0,0}}=0
   \end{cases}
   \\
   \begin{cases}
      \inpro{\varphi_{2,1,1}}{\hat{D}_z \vert \varphi_{1,0,0}}=\inpro{\varphi_{2,1,-1}}{\hat{D}_z \vert \varphi_{1,0,0}}=0
      \\
      \inpro{\varphi_{2,1,0}}{\hat{D}_x \vert \varphi_{1,0,0}}=\frac{q\chi}{\sqrt{3}}
   \end{cases}
   \\
   \intertext{设初态}
   \ket{\psi_m(0)}=\cos\alpha \ket{\varphi_{1,0,0}}+\sin\alpha \ket{\varphi_{2,1,m}}
   \\
   \intertext{忽略总体相位因子$\e^{\frac{\i}{\hbar} E_I t}$}
   \ket{\psi_m(t)}=\cos\alpha \ket{\varphi_{1,0,0}}+\sin\alpha \e^{-\i(\Omega+m\omega_L)t} \ket{\varphi_{2,1,m}}
   \\
   \intertext{计算}
   \langle \bm{D} \rangle_m(t)=\inpro{\psi_m(t)}{\hat{\bm{D}} \vert \psi_m(t)}
   \\
   \begin{cases}
      \langle D_x \rangle_1=-\frac{q\chi}{\sqrt{6}} \sin2\alpha \cos(\Omega+\omega_L)t
      \\
      \langle D_y \rangle_1=-\frac{q\chi}{\sqrt{6}} \sin2\alpha \sin(\Omega+\omega_L)t
      \\
      \langle D_z \rangle_1=0
   \end{cases}
   \\
   \begin{cases}
      \langle D_x \rangle_0=\langle D_y \rangle_0=0
      \\
      \langle D_z \rangle_0=\frac{q\chi}{\sqrt{3}} \sin2\alpha \cos\Omega t
   \end{cases}
   \\
   \begin{cases}
      \langle D_x \rangle_{-1}=\frac{q\chi}{\sqrt{6}} \sin2\alpha \cos(\Omega-\omega_L)t
      \\
      \langle D_y \rangle_{-1}=-\frac{q\chi}{\sqrt{6}} \sin2\alpha \sin(\Omega-\omega_L)t
      \\
      \langle D_z \rangle_{-1}=0
   \end{cases}
\end{gather*}
\begin{itemize}
   \item 若$m=1$,频率$\Omega+\omega_L$,$z$轴上是圆偏振光$\sigma_+$偏振,$xOy$平面上是线偏振光,其余方向是椭圆偏振
   \item 若$m=0$,频率$\Omega$,在$z$轴上无电磁波,其余方向是线偏振
   \item 若$m=-1$,频率$\Omega-\omega_L$,偏振方向与$m=1$时相反,$z$轴上是圆偏振光$\sigma_-$偏振,$xOy$平面上是线偏振光,其余方向是椭圆偏振
\end{itemize}

\subsection{原子轨道}

原子轨道:重新选定一组基,用实函数来表示波函数
\begin{enumerate}
   \item $ns$轨道$(l=0)$
   \begin{gather*}
      \varphi_{ns}(\bm{r})=\varphi_{n,0,0}(\bm{r})
   \end{gather*}
   \item $np_x,np_y,np_z$轨道$(l=1)$
   \begin{gather*}
      \begin{cases}
         \varphi_{np_x}(\bm{r})=-\frac{1}{\sqrt{2}}(\varphi_{n,1,1}(\bm{r})-\varphi_{n,1,-1}(\bm{r}))=\sqrt{\frac{3}{4\pi}}R_{n,1}(r)\frac{x}{r}
         \\
         \varphi_{np_y}(\bm{r})=\frac{\i}{\sqrt{2}}(\varphi_{n,1,1}(\bm{r})+\varphi_{n,1,-1}(\bm{r}))=\sqrt{\frac{3}{4\pi}}R_{n,1}(r)\frac{y}{r}
         \\
         \varphi_{np_z}(\bm{r})=\varphi_{n,1,0}(\bm{r})=\sqrt{\frac{3}{4\pi}}R_{n,1}(r)\frac{z}{r}
      \end{cases}
   \end{gather*}
   对于$\bm{u}=(\cos\alpha,\cos\beta,\cos\gamma)$方向上的$np$轨道,
   \begin{gather*}
      \varphi_{np_u}(\bm{r})=\cos\alpha \varphi_{np_x}(\bm{r})+\cos\beta \varphi_{np_y}(\bm{r})+\cos\gamma \varphi_{np_z}(\bm{r})
   \end{gather*}
   \item $l$的其他值:对于$m \neq 0$,将$\varphi_{n,l,m}(\bm{r})$和$\varphi_{n,l,m}(\bm{r})$替换为
   \begin{gather*}
      \frac{1}{\sqrt{2}}(\varphi_{n,l,m}(\bm{r})+(-1)^m \varphi_{n,l,-m}(\bm{r}))
      \\
      \frac{\i}{\sqrt{2}}(\varphi_{n,l,m}(\bm{r})-(-1)^m \varphi_{n,l,-m}(\bm{r}))
   \end{gather*}
\end{enumerate}

杂化轨道:重新选定$\mathscr{E}_s \otimes \mathscr{E}_p$空间的一组基
\begin{enumerate}
   \item $sp$杂化
   \begin{gather*}
      \begin{cases}
         \varphi_{n,s,p_z}(\bm{r})=\frac{1}{\sqrt{2}}(\varphi_{ns}(\bm{r})+\varphi_{np_z}(\bm{r}))
         \\
         \varphi'_{n,s,p_z}(\bm{r})=\frac{1}{\sqrt{2}}(\varphi_{ns}(\bm{r})-\varphi_{np_z}(\bm{r}))
         \\
         \varphi_{np_x}(\bm{r})
         \\
         \varphi_{np_y}(\bm{r})
      \end{cases}
   \end{gather*}
   \item $sp^2$杂化
   \begin{gather*}
      \begin{cases}
         \varphi_{n,s,p_x,p_y}(\bm{r})=\frac{1}{\sqrt{3}}\varphi_{ns}(\bm{r})+\sqrt{\frac{2}{3}}\varphi_{np_x}(\bm{r})
         \\
         \varphi'_{n,s,p_x,p_y}(\bm{r})=\frac{1}{\sqrt{3}}\varphi_{ns}(\bm{r})-\frac{1}{\sqrt{6}}\varphi_{np_x}(\bm{r})+\frac{1}{\sqrt{2}}\varphi_{np_y}(\bm{r})
         \\
         \varphi''_{n,s,p_x,p_y}(\bm{r})=\frac{1}{\sqrt{3}}\varphi_{ns}(\bm{r})-\frac{1}{\sqrt{6}}\varphi_{np_x}(\bm{r})-\frac{1}{\sqrt{2}}\varphi_{np_y}(\bm{r})
         \\
         \varphi_{np_z}(\bm{r})
      \end{cases}
      \\
      \intertext{旋转对称}
      \begin{cases}
         \ket{\varphi'_{n,s,p_x,p_y}}=\e^{-\frac{\i}{\hbar}\frac{2\pi}{3} \hat{L}_z} \ket{\varphi_{n,s,p_x,p_y}}
         \\
         \ket{\varphi''_{n,s,p_x,p_y}}=\e^{\frac{\i}{\hbar}\frac{2\pi}{3} \hat{L}_z} \ket{\varphi_{n,s,p_x,p_y}}
      \end{cases}
   \end{gather*}
   \item $sp^3$杂化
   \begin{gather*}
      \begin{bmatrix}
         \varphi_{n,s,p_x,p_y,p_z}(\bm{r}) \\
         \varphi'_{n,s,p_x,p_y,p_z}(\bm{r}) \\
         \varphi''_{n,s,p_x,p_y,p_z}(\bm{r}) \\
         \varphi'''_{n,s,p_x,p_y,p_z}(\bm{r})
      \end{bmatrix}
      =
      \begin{bmatrix}
         \frac{1}{2} & \frac{1}{2} & \frac{1}{2} & \frac{1}{2} \\
         \frac{1}{2} & -\frac{1}{2} & -\frac{1}{2} & \frac{1}{2} \\
         \frac{1}{2} & -\frac{1}{2} & \frac{1}{2} & -\frac{1}{2} \\
         \frac{1}{2} & \frac{1}{2} & -\frac{1}{2} & -\frac{1}{2}
      \end{bmatrix}
      \begin{bmatrix}
         \varphi_{ns}(\bm{r}) \\
         \varphi_{np_x}(\bm{r}) \\
         \varphi_{np_y}(\bm{r}) \\
         \varphi_{np_z}(\bm{r})
      \end{bmatrix}
   \end{gather*}
\end{enumerate}

\subsection{类氢体系}

\begin{gather*}
   \begin{cases}
      \text{含有一个电子}
      \begin{cases}
         \text{氢的同位素}
         \\
         \text{$\mu$子素:$\mu^+$子代替质子}
         \\
         \text{电子偶素:正电子$e^+$代替质子}
         \\
         \text{固体物理}
         \begin{cases}
            \text{磷原子代替硅原子,要替换介电常数,质量替换为有效质量}
            \\
            \text{激子:外层电子吸收光子从价带激发到导带,产生空穴}
         \end{cases}
         \\
         \text{类氢离子:只有一个电子的离子}
      \end{cases}
      \\
      \text{无电子:奇特原子}
      \begin{cases}
         \text{$\mu$原子:$\mu^-$子代替电子,对强相互作用不敏感,用来探索核的内部结构}
         \\
         \text{强子原子:带负电的强子代替电子,激发态可以忽略强相互作用}
      \end{cases}
   \end{cases}
\end{gather*}

\section{各向同性三维谐振子}

径向方程
\begin{gather*}
   -\frac{\hbar^2}{2mr}\dt{^2}{r^2}(rR_{nl})+\frac{1}{2}m\omega^2 r^2 R_{kl}+\frac{l(l+1)\hbar^2}{2mr^2} R_{kl}=E_{kl} R_{kl}
   \\
   \intertext{设}
   \begin{cases}
      R_{kl}(r)=\frac{1}{r}u_{kl}(r)
      \\
      \epsilon_{kl}=\frac{2mE_{kl}}{\hbar^2}
      \\
      \beta=\sqrt{\frac{m\omega}{\hbar}}
   \end{cases}
   \\
   \begin{cases}
      (\dt{^2}{r^2}-\beta^4 r^2-\frac{l(l+1)}{r^2}+\epsilon_{kl})u_{kl}(r)=0
      \\
      u_{kl}(0)=0
   \end{cases}
   \\
   \intertext{考察渐近行为,设}
   u_{kl}(r)=\e^{-\frac{1}{2}\beta^2 r^2} y_{kl}(r)
   \\
   \intertext{幂级数法,设}
   y_{kl}(r)=r^s \displaystyle \sum_{q=0}^{+\infty} a_q r^q
   \\
   \intertext{解得奇数项为$0$,}
   s=l+1
   \\
   a_{q+2}=\frac{(2q+2l+3)\beta^2-\epsilon_{kl}}{(q+2)(q+2l+3)}a_q
   \\
   \intertext{由平方可积性,幂级数截断为多项式}
   \epsilon_{kl}=(2k+2l+3)\beta^2,\quad k\text{为偶数}
   \\
   E_{kl}=(k+l+\frac{3}{2})\hbar \omega,\quad k\text{为偶数}
   \\
   \intertext{统一标记为量子数$n=k+l$}
   E_n=(n+\frac{3}{2})\hbar \omega
   \\
   \intertext{由$k$为偶数可以推知$l$的取值,可以求得简并度}
   g_n=\frac{(n+1)(n+2)}{2}
\end{gather*}

\section{双原子分子的振动-转动}

仅考虑相对运动,定态波函数
\begin{gather*}
   \varphi_{klm}(r,\theta,\phi)=\frac{1}{r}u_{kl}(r) Y_{lm}(\theta,\phi)
   \\
   (-\frac{\hbar^2}{2m}\dt{^2}{r^2}+V(r)+\frac{l(l+1)\hbar^2}{2mr^2})u_{kl}(r)=E_{kl} u_{kl}(r)
\end{gather*}


\begin{enumerate}
   \item $l=0$ \\
   $V(r)$在最小值$r=r_e$处Taylor展开
   \begin{gather*}
      V(r)=-V_0+f(r-r_e)^2-g(r-r_e)^3+\cdots
   \end{gather*}
   略去三次及以上的项,得到谐振子方程,
   \begin{gather*}
      \omega=\sqrt{\frac{2f}{\mu}}
      \\
      E_{k,0}=-V_0+(k+\frac{1}{2})\hbar \omega,\quad k \in \mathbb{N}
      \\
      u_{k,0}(r)=\sqrt[4]{\frac{\mu\omega}{\pi \hbar}} \frac{1}{\sqrt{2^k k!}} \e^{-\frac{1}{2} \frac{\mu \omega}{\hbar} (r-r_e)^2} H_k(\sqrt{\frac{\mu \omega}{\hbar}} (r-r_e))
      \\
      \intertext{波函数展延范围}
      \Delta r_k \approx \sqrt{(k+\frac{1}{2})\frac{\hbar}{\mu \omega}}
      \\
      \intertext{条件}
      \begin{cases}
         f \gg g \Delta r_k
         \\
         \Delta r_k \ll r_e
      \end{cases}
   \end{gather*}
   只考虑小量子数$k$
   \item $l$为任意值 \\
   为了使离心势可以忽略高阶项,取为$r=r_e$处的值,应当有
   \begin{gather*}
      \frac{l(l+1)\hbar^2}{2\mu r_e^2} \ll \hbar \omega
   \end{gather*}
   只考虑小量子数$l$
   \begin{gather*}
      E_{k,l}=-V_0+(k+\frac{1}{2})\hbar \omega+\frac{l(l+1)\hbar^2}{2\mu r_e^2} \quad k,l \in \mathbb{N}
      \\
      \intertext{径向函数只与$k$有关}
      u_{kl}(r)=u_{k,0}(r)
   \end{gather*}
   振动-转动谱
   \begin{gather*}
      \inpro{\varphi_{k'l'm'}}{\hat{D} \cos\theta \vert \varphi_{klm}}
   \end{gather*}
   非零条件:
   \begin{gather*}
      \begin{cases}
         l'-l=\pm 1
         \\
         k'-k=0,\pm 1
      \end{cases}
   \end{gather*}
   \begin{itemize}
      \item $k'-k=0$构成纯转动谱,在\ref{sec:6.5}节已经讨论过
      \item $k'=k+1,l'=l+1$,谱线角频率
      \begin{gather*}
         \omega+\frac{\hbar}{\mu r_e^2} (l+1) \quad l = 0,1,\cdots
      \end{gather*}
      \item $k'=k+1,l'=l-1$,谱线角频率
      \begin{gather*}
         \omega-\frac{\hbar}{\mu r_e^2} (l'+1) \quad l' = 0,1,\cdots
      \end{gather*}
   \end{itemize}
   \ref{sec:5.7.1}节讨论的纯振动谱实际上不存在
\end{enumerate}

离心势的变化带来的修正:离心势在$r=r_e$处Taylor展开
\begin{gather*}
   \frac{l(l+1)\hbar^2}{2\mu r^2}=\frac{l(l+1)\hbar^2}{2\mu r_e^2}-\frac{l(l+1)\hbar^2}{\mu r_e^3}(r-r_e)+\frac{3l(l+1)\hbar^2}{2\mu r_e^2}(r-r_e)^2+\cdots
   \\
   V_{eff}(r)=-V_0+f(r-r_e)^2-g(r-r_e)^3+\frac{l(l+1)\hbar^2}{2\mu r_e^2}-\frac{l(l+1)\hbar^2}{\mu r_e^3}(r-r_e)+\frac{3l(l+1)\hbar^2}{2\mu r_e^2}(r-r_e)^2+\cdots
\end{gather*}
极小值点变为$r=r'_e$,忽略$\frac{3l(l+1)\hbar^2}{2\mu r_e^2}(r-r_e)^2$和$-g(r-r_e)^3$及更高阶项,
\begin{gather*}
   r'_e = r_e+\frac{l(l+1)\hbar^2}{2\mu f r_e^3}
\end{gather*}
满足
\begin{gather*}
   \frac{r'_e-r_e}{\Delta r_0} \approx \frac{\frac{l(l+1)\hbar^2}{\mu r_e^2}}{\hbar \omega} \frac{\Delta r_0}{r_e} \ll 1
   \\
   V_{eff}(r'_e)=-V_0+\frac{\hbar^2}{2\mu r_e^2}l(l+1)-\frac{\hbar^4}{4\mu^2 r_e^6 f} l^2 (l+1)^2
   \\
   \intertext{有效势重新Taylor展开为}
   V_{eff}(r)=V_{eff}(r'_e)+f'(r-r'_e)^2-g'(r-r'_e)^3+\cdots
   \\
   f'=f+\frac{3l(l+1)\hbar^2}{2\mu r_e^4}-\frac{3gl(l+1)\hbar^2}{2\mu f r_e^3}
   \\
   \omega'=\sqrt{\frac{2f'}{\mu}}=\omega-l(l+1)\frac{3\hbar^2 }{4\mu f r_e^3} \omega(\frac{g}{f}-\frac{1}{r_e})
   \\
   g' \approx g
   \\
   \intertext{忽略高阶项,有效势可以写作}
   V_{eff}(r)=V_{eff}(r'_e)+\frac{1}{2} \mu \omega'(r-r'_e)^2-g(r-r'_e)^3
   \\
   \intertext{运用微扰论,在谐振子势的基础上,求得三次方的势带来的能量修正}
   E_{kl}=-V_0+(k+\frac{1}{2})\hbar \omega'+ \frac{\hbar^2}{2\mu r_e^2}l(l+1)-\frac{\hbar^4}{4\mu^2 r_e^6 f} l^2 (l+1)^2-\frac{15g^2\hbar^2}{4\mu^3 \omega'^4}((k+\frac{1}{2})^2+\frac{7}{60})
\end{gather*}
振动-转动耦合项
\begin{gather*}
   (k+\frac{1}{2})\hbar (\omega'-\omega)=-l(l+1)(k+\frac{1}{2})\frac{3\hbar^2 }{4\mu f r_e^3} \hbar \omega(\frac{g}{f}-\frac{1}{r_e})
\end{gather*}
可以等效为振动改变了转动惯量,原因有
\begin{itemize}
   \item 第一项$\frac{g}{f}$来源于$g$引起$V(r)$非对称,$r$在$r>r_e$区间时间更长,贡献更大,能量减小
   \item 第二项来源于离心势取平均,$r<r_e$贡献更大,$\langle \frac{1}{r^2} \rangle \geq \frac{1}{\langle r^2 \rangle}$,能量增大
\end{itemize}