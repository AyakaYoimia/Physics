\chapter{量子力学入门:波粒二象性}

\section{电磁波与光子}

Plank-Einstein关系:
\begin{gather*}
   E=\hbar \omega
   \\
   \bm{p}=\hbar \bm{k}
\end{gather*}

波粒二象性:双缝干涉

谱分解定理:分解到特征向量

\section{物质粒子与物质波}

De-Broglie关系:
\begin{gather*}
   E=\hbar \omega
   \\
   \bm{p}=\hbar \bm{k}
\end{gather*}

\section{波函数与Schrödinger方程}

用波函数$\psi(\bm{r},t)$描述粒子的量子态
\begin{equation*}
   \d P(\bm{r},t)=|\psi(\bm{r},t)|^2 \d^3 \bm{r}
\end{equation*}

谱分解定理
\begin{enumerate}
   \item 测量结果在本征结果张成的空间内
   \item 本征结果对应本征态
\end{enumerate}

Schrödinger方程
\begin{equation*}
   \i \hbar \pt{}{t} \psi(\bm{r},t)=-\frac{\hbar^2}{2m}\nabla^2\psi(\bm{r},t)+V(\bm{r},t)\psi(\bm{r},t)
\end{equation*}
\begin{enumerate}
   \item 方程线性齐次$\implies$叠加原理
   \item 方程一阶$\implies$初值决定演化
\end{enumerate}

经典态:由相空间6个参数确定

量子态:由波函数确定

非相对论量子力学中物质粒子不会产生或湮灭

定态Schrödinger方程
\begin{gather*}
   \psi(\bm{r},t)=\psi(\bm{r})\e^{-\i E t/\hbar}
   \\
   -\frac{\hbar^2}{2m}\nabla^2\psi(\bm{r})+V(\bm{r})\psi(\bm{r})=E\psi(\bm{r})
   \\
   \hat{H} \psi(\bm{r})=E\psi(\bm{r})
\end{gather*}

\section{波包}

用波包描述粒子

\subsection{自由粒子}

Schrödinger方程的解为平面波
\begin{gather*}
   \psi(\bm{r},t)=A\e^{\i(\bm{k}\cdot\bm{r}-\omega t)}
   \\
   \omega=\frac{\hbar k^2}{2m}
\end{gather*}
根据叠加原理,
\begin{equation*}
   \psi(\bm{r},t)=\frac{1}{(2\pi)^{\frac{3}{2}}} \int g(k)\e^{\i(\bm{k}\cdot\bm{r}-\omega(\bm{k}) t)} \d^3 \bm{k}
\end{equation*}
一维情况下,
\begin{gather*}
   \psi(x,t)=\frac{1}{(2\pi)^{\frac{1}{2}}} \int_{-\infty}^{+\infty} g(k)\e^{\i(kx-\omega(k) t)} \d k
   \\
   g(k)=\frac{1}{(2\pi)^{\frac{1}{2}}} \int_{-\infty}^{+\infty} \psi(x,0)\e^{-\i kx} \d x
\end{gather*}

单色平面波不是平方可积函数,不能表示量子态

\subsection{波包的形状}

方法1:
\begin{gather*}
   g(k)=|g(k)|\e^{\i\alpha(k)}
   \\
   \alpha(k)=\alpha_0+\left.\dt{\alpha}{k}\right|_{k=k_0} (k-k_0)
   \\
   \psi(x,0)=\frac{\e^{\i (k_0 x+\alpha(k_0))}}{\sqrt{2\pi}} \int_{-\infty}^{+\infty} |g(k)|\e^{\i (k-k_0)(x-x_0)} \d k
   \\
   \intertext{其中}
   x_0=-\left.\dt{\alpha}{k}\right|_{k=k_0}
\end{gather*}
波包中心在$x_0$处

方法2:稳定相位\\
波包中心相位对$k$的导数为$0$,$x+\dt{\alpha}{k}=0$
\begin{equation*}
   \Delta k \Delta x \geq 1
\end{equation*}

相速度
\begin{equation*}
   v_{\phi}=\frac{\omega}{k}
\end{equation*}
群速度
\begin{equation*}
   v_g=\dt{\omega}{k}
\end{equation*}
稳定相位:$g(k)$辐角$\alpha(k)-\omega(k)t$
\begin{gather*}
   x_M(t)=\left.\dt{\omega}{k}\right|_{k=k_0} t - \left.\dt{\alpha}{k}\right|_{k=k_0}
   \\
   v_g=\left.\dt{\omega}{k}\right|_{k=k_0}
\end{gather*}

\section{Heisenberg不确定性关系}

\begin{gather*}
   \psi(x,0)=\frac{1}{\sqrt{2\pi \hbar}} \int \bar{\psi}(p) \e^{\i(px/\hbar)} \d p
   \\
   \intertext{Bessel-Parseval关系}
   \int_{-\infty}^{+\infty} |\psi(x,0)|^2 \d x=\int_{-\infty}^{+\infty} |\bar{\psi}(p)|^2 \d p
\end{gather*}

不确定性关系
\begin{equation*}
   \Delta x \Delta p \geq \frac{\hbar}{2}
\end{equation*}