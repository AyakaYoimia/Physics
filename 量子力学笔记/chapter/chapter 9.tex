\chapter{电子的自旋}

\section{实验}

\begin{itemize}
    \item 谱线的精细结构
    \item 反常Zeeman效应:原子序数为奇数的原子却得到了偶数个Zeeman次能级
    \item Stern-Gerlach实验中,银原子分布在对称的斑点上,说明银原子具有半整数角动量
\end{itemize}

\section{自旋}

电子具有固有的角动量$\bm{S}$,称为自旋角动量
\begin{gather*}
    \intertext{对应磁矩}
    \bm{M}_S=\frac{q}{m} \bm{S}
\end{gather*}
量子电动力学会对旋磁比加入修正,称为反常磁矩

对自旋角动量量子化
\begin{enumerate}
    \item 自旋算符$\hat{\bm{S}}$满足对易关系
    \begin{gather*}
        [\hat{S}_x,\hat{S}_y]=\i \hbar \hat{S}_z
    \end{gather*}
    \item 自旋算符在自旋态空间$\mathscr{E}_S$中作用,$\hat{\bm{S}}^2$和$\hat{S}_z$构成一组ECOC
    \begin{gather*}
        \begin{cases}
            \hat{\bm{S}}^2 \ket{s,m}=s(s+1)\hbar^2 \ket{s,m}
            \\
            \hat{S}_z \ket{s,m}=m\hbar \ket{s,m}
        \end{cases}
    \end{gather*}
    $s$为整数或半整数,$m$取$-s$与$s$之间间隔为$1$的点列,$\mathscr{E}_S$是$(2s+1)$维空间
    \item 粒子的态空间
    \begin{gather*}
        \mathscr{E}=\mathscr{E}_{\bm{r}} \otimes \mathscr{E}_S
    \end{gather*}
    \item 电子自旋为$s=\frac{1}{2}$
\end{enumerate}

\section{$\frac{1}{2}$角动量}

选取$\hat{\bm{S}}^2$和$\hat{S}_z$的共同特征向量$\ket{+}$和$\ket{-}$作为基,
\begin{gather*}
    \begin{cases}
       \hat{S}_z \ket{\pm}=\pm\frac{\hbar}{2} \ket{\pm} \\
       \hat{\bm{S}}^2 \ket{\pm}=\frac{3}{4} \hbar^2 \ket{\pm}
    \end{cases}
    \\
    \begin{cases}
       \inpro{+}{+}=\inpro{-}{-}=1 \\
       \inpro{+}{-}=0\\
       \ket{+}\bra{+}+\ket{-}\bra{-}=\hat{I}
    \end{cases}
    \\
    \intertext{升降阶算符}
    \hat{S}_{\pm}=\hat{S}_x \pm \i \hat{S}_y
 \end{gather*}

\section{对自旋$\frac{1}{2}$粒子的非相对论性描述}

将$\mathscr{E}_{\bm{r}}$和$\mathscr{E}_S$中的观察算符延伸到$\mathscr{E}$中,在两个子空间分别选定一组ECOC,由于$\mathscr{E}$中的所有右矢都是$\hat{\bm{S}}^2$中对应相同特征值的特征向量,可以不选择$\hat{\bm{S}}^2$
\begin{gather*}
    \ket{\bm{r},\pm}=\ket{\bm{r}} \otimes \ket{\pm}
\end{gather*}
构成正交归一完备基

在$\left\{\ket{\bm{r}}\right\}$表象下,
\begin{gather*}
    \psi_{\pm}(\bm{r})=\inpro{\bm{r},\pm}{\psi}
    \\
    \intertext{右矢表示为二分量旋量的形式}
    [\psi](\bm{r})=
    \begin{bmatrix}
        \psi_+(\bm{r})
        \\
        \psi_-(\bm{r})
    \end{bmatrix}
    \\
    \intertext{左矢表示为}
    [\psi]^{\dagger}(\bm{r})=
    \begin{bmatrix}
        \psi_+^*(\bm{r}) & \psi_-^*(\bm{r})
    \end{bmatrix}
\end{gather*}
线性算符可以用矩阵表示

\section{自旋$\frac{1}{2}$粒子的旋转算符}

围绕$\bm{u}$方向旋转角度$\alpha$,
\begin{gather*}
    \hat{R}_{\bm{u}}(\alpha)=\e^{-\frac{\i}{\hbar} \alpha \hat{\bm{J}} \cdot \bm{u}}
\end{gather*}

自旋态的旋转算符
\begin{align*}
    \hat{R}_{\bm{u}}^{(S)}(\alpha)&=\e^{-\frac{\i}{\hbar} \alpha \hat{\bm{J}} \cdot \bm{u}}
    \\
    &=\cos\frac{\alpha}{2}-\i \bm{\sigma} \cdot \bm{u} \sin\frac{\alpha}{2}
\end{align*}

二分量旋量的旋转
\begin{gather*}
    \ket{\psi'}=\hat{R} \ket{\psi}
    \\
    \psi'_{\pm}(\bm{r})=R_{\pm +}^{(s)} \psi_+(R^{-1}\bm{r}) + R_{\pm -}^{(s)} \psi_-(R^{-1}\bm{r})
    \\
    \begin{bmatrix}
        \psi'_{+}(\bm{r})
        \\
        \psi'_{-}(\bm{r})
    \end{bmatrix}
    =
    \begin{bmatrix}
        R_{++}^{(s)} & R_{+-}^{(s)} \\
        R_{-+}^{(s)} & R_{--}^{(s)}
    \end{bmatrix}
    \begin{bmatrix}
        \psi_{+}(\bm{r})
        \\
        \psi_{-}(\bm{r})
    \end{bmatrix}
\end{gather*}