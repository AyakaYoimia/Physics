\chapter{定态微扰理论}

保守Hamiltonian特征方程的近似解法

\section{方法概述}

微扰理论适用于Hamiltonian
\begin{gather*}
    \hat{H}(\lambda)=\hat{H}_0+\lambda\hat{W}
\end{gather*}
$\hat{H}_0$的特征值特征向量已知,具有离散谱,微扰$\lambda \ll 1$
\begin{gather*}
    \hat{H}_0 \ket{\varphi_p^i}=E_p^0 \ket{\varphi_p^i}
\end{gather*}
$\{\ket{\varphi_p^i}\}$构成一组正交归一基

近似求解特征方程
\begin{gather*}
    \hat{H}(\lambda) \ket{\psi(\lambda)}=E(\lambda) \ket{\psi(\lambda)}
    \\
    \intertext{幂次展开}
    E(\lambda)=\sum_{q=0}^{\infty} \lambda^q \varepsilon_q
    \\
    \ket{\psi(\lambda)}=\sum_{q=0}^{\infty} \lambda^q \ket{q}
    \\
    \intertext{代入特征方程}
    (\hat{H}_0+\lambda\hat{W}) \left[\sum_{q=0}^{\infty} \lambda^q \ket{q}\right]=\left[\sum_{q'=0}^{\infty} \lambda^{q'} \varepsilon_{q'} \right] \left[\sum_{q=0}^{\infty} \lambda^q \ket{q}\right]
    \\
    \intertext{上式对$\forall \lambda$成立,比较系数}
    \hat{H}_0 \ket{0}=\varepsilon_0 \ket{0}
    \\
    (\hat{H}_0-\varepsilon_0) \ket{1}+(\hat{W}-\varepsilon_1)\ket{0}=0
    \\
    (\hat{H}_0-\varepsilon_0) \ket{2}+(\hat{W}-\varepsilon_1)\ket{1}-\varepsilon_2 \ket{0}=0
    \\
    (\hat{H}_0-\varepsilon_0) \ket{q}+(\hat{W}-\varepsilon_1)\ket{q-1}-\varepsilon_2 \ket{q-2}-\cdots-\varepsilon_q \ket{0}=0
\end{gather*}
规定相位使$\inpro{0}{\psi(\lambda)}$为实数

\subsection{非简并能级}

考虑非简并能级$E_n^0$,特征向量$\ket{\varphi_n}$

\subsubsection{零阶修正}

\begin{gather*}
    \lambda \to 0
    \\
    \begin{cases}
        \varepsilon_0=E_n^0
        \\
        \ket{0}=\ket{\varphi_n}
    \end{cases}
\end{gather*}
假设$\lambda$足够小使得$E_n(\lambda)$保持为非简并的

\subsubsection{一阶修正}

一阶方程作用到$\bra{\varphi_n}$上
\begin{gather*}
    \varepsilon_1=\inpro{\varphi_n}{\hat{W} \vert \varphi_n}
    \\
    E_n(\lambda)=E_n^0+\lambda \inpro{\varphi_n}{\hat{W} \vert \varphi_n}+o(\lambda^2)
\end{gather*}

一阶方程作用到$\ket{\varphi_p^i}(p \neq n)$上,结合$\inpro{\varphi_n}{1}=0$,
\begin{gather*}
    \ket{1}=\sum_{p \neq n} \sum_{i} \frac{\inpro{\varphi_p^i}{\hat{W} \vert \varphi_n}}{E_n^0-E_p^0} \ket{\varphi_p^i}+o(\lambda^2)
    \\
    \ket{\psi_n(\lambda)}=\ket{\varphi_n}+\lambda \sum_{p \neq n} \sum_{i} \frac{\inpro{\varphi_p^i}{\hat{W} \vert \varphi_n}}{E_n^0-E_p^0} \ket{\varphi_p^i}+o(\lambda^2)
\end{gather*}

\subsubsection{二阶修正}

仿照上述方法
\begin{gather*}
    \varepsilon_2=\sum_{p \neq n} \sum_{i} \frac{\left| \inpro{\varphi_p^i}{\hat{W} \vert \varphi_n} \right|^2}{E_n^0-E_p^0}+o(\lambda^3)
    \\
    E_n(\lambda)=E_n^0+\lambda \inpro{\varphi_n}{\hat{W} \vert \varphi_n}+\lambda^2 \sum_{p \neq n} \sum_{i} \frac{\left| \inpro{\varphi_p^i}{\hat{W} \vert \varphi_n} \right|^2}{E_n^0-E_p^0}+o(\lambda^3)
\end{gather*}

$\varepsilon_2$上限
\begin{gather*}
    \intertext{$E_n^0$与其他能级差下限$\Delta E$}
    |E_n^0-E_p^0| \geq \Delta E
    \\
    |\varepsilon_2| \leq \frac{\sigma^2_{\hat{W}}}{\Delta E}
\end{gather*}

\subsection{简并能级}

设能级$E_n^0$简并度为$g_n$
\begin{gather*}
    \varepsilon_0=E_n^0
    \\
    \ket{0}\in \mathscr{E}_n^0
\end{gather*}

一阶方程作用到$\ket{\varphi_n^i}$上
\begin{gather*}
    \inpro{\varphi_n^i}{\hat{W} \vert 0}=\varepsilon_1 \inpro{\varphi_n^i}{0}
    \\
    \intertext{插入封闭性关系式,由$\ket{0}\in \mathscr{E}_n^0$得到}
    \sum_{i'=1}^{g_n} \inpro{\varphi_n^i}{\hat{W} \vert \varphi_n^{i'}} \inpro{\varphi_n^{i'}}{0}=\varepsilon_1 \inpro{\varphi_n^{i}}{0}
\end{gather*}
限制在子空间$\mathscr{E}_n^0$内,得到特征值方程
\begin{gather*}
    \hat{W}^{(n)} \ket{0}=\varepsilon_1 \ket{0}
\end{gather*}

在$\mathscr{E}_n^0$内将表示矩阵对角化,可以得到一阶特征值和零阶特征向量,设表示矩阵互异特征值为
\begin{gather*}
    \varepsilon_i^j \quad j=1,2,\cdots,f_n^{(1)}
\end{gather*}
简并度之和为$g_n$,则在一级修正下简并能级$E_n^0$分裂
\begin{itemize}
    \item $\varepsilon_1^j$非简并:$\ket{0}$可以确定,对应非简并特征值$E_n^0+\lambda \varepsilon_1^j$
    \item $\varepsilon_1^j$是$q$重简并的:$\ket{0}$在对应的$q$维子空间中,需要更高阶近似确定,在实质简并的情况下高阶也无法确定
\end{itemize}

\section{微扰势场中的一维谐振子}

\begin{gather*}
    \hat{H}_0=\frac{\hat{P}^2}{2m}+\frac{1}{2}m\omega^2 \hat{X}^2
    \\
    E_n^0=(n+\frac{1}{2})\hbar \omega \quad n=0,1,2,\cdots
\end{gather*}

\subsection{一次微扰}

\begin{gather*}
    \hat{W}=\lambda \hbar \omega \frac{m\omega}{\hbar} \hat{X}
    \\
    \hat{H}=\hat{H}_0+\hat{W}
    \\
    E_n=(n+\frac{1}{2})\hbar \omega-\frac{1}{2} \lambda^2 \hbar \omega
    \\
    \ket{\psi_n}=\e^{-\frac{\lambda}{\sqrt{2}}(\hat{a}^{\dagger}-\hat{a})}\ket{\varphi_n}
\end{gather*}

\subsection{二次微扰}

\begin{gather*}
    \hat{W}=\frac{1}{2} \rho m\omega^2 \hat{X}^2
    \\
    E_n=(n+\frac{1}{2})\hbar \omega \sqrt{1+\rho}
\end{gather*}

\subsection{三次微扰}

\begin{align*}
    \hat{W}=&\sigma \hbar \omega \hat{X}^3
    \\
    E_n=&(n+\frac{1}{2})\hbar \omega-\frac{15}{4} \sigma^2 (n+\frac{1}{2})^2 \hbar \omega - \frac{7}{16} \sigma^2 \hbar \omega + \cdots
    \\
    \ket{\psi_n}=&\ket{\varphi_n}-3\sigma (\frac{n+1}{2})^{\frac{3}{2}} \ket{\varphi_{n+1}}+3\sigma (\frac{n}{2})^{\frac{3}{2}} \ket{\varphi_{n}}
    \\
    &-\frac{\sigma}{3}(\frac{(n+3)(n+2)(n+1)}{8})^{\frac{1}{2}} \ket{\varphi_{n+3}}+\frac{\sigma}{3}(\frac{n(n-1)(n-2)}{8})^{\frac{1}{2}} \ket{\varphi_{n-3}}
\end{align*}

应用:双原子分子振动的非谐性\\
受到微扰势场影响,选择定则改变

\section{磁偶极子相互作用}

考虑静磁场$\bm{B}_0$中两个自旋$\frac{1}{2}$粒子的磁偶极相互作用,例如$\mathrm{CaSO_4 \cdot 2H_2O}$单晶中结晶水分子的两个质子
\begin{gather*}
    \bm{M}_1=\gamma_1 \bm{S}_1
    \\
    \bm{M}_2=\gamma_2 \bm{S}_2
    \\
    W=\frac{\mu_0}{4\pi} \gamma_1 \gamma_2 \frac{1}{r^3}[\bm{S}_1 \cdot \bm{S}_2-3(\bm{S}_1 \cdot \bm{n})(\bm{S}_2 \cdot \bm{n})]
\end{gather*}
$\bm{n}$为粒子1指向粒子2连线方向的单位矢量
\begin{gather*}
    \hat{W}=\xi(r)(T_0+T'_0+T_1+T_{-1}+T_2+T_{-2})
    \\
    \begin{cases}
        \xi(r)=-\frac{\mu_0}{4\pi} \gamma_1 \gamma_2 \frac{1}{r^3}
        \\
        T_0=(3\cos^2\theta-1)\hat{S}_{1z}\hat{S}_{2z}
        \\
        T'_0=-\frac{3\cos^2\theta-1}{4}(\hat{S}_{1+}\hat{S}_{2-}+\hat{S}_{1-}\hat{S}_{2+})
        \\
        T_1=\frac{3}{2}\sin\theta\cos\theta \e^{-\i \phi}(\hat{S}_{1z}\hat{S}_{2+}+\hat{S}_{1+}\hat{S}_{2z})
        \\
        T_{-1}=\frac{3}{2}\sin\theta\cos\theta \e^{\i \phi}(\hat{S}_{1z}\hat{S}_{2-}+\hat{S}_{1-}\hat{S}_{2z})
        \\
        T_2=\frac{3}{4}\sin^2\theta \e^{-2\i \phi} \hat{S}_{1+} \hat{S}_{2+}
        \\
        T_{-2}=\frac{3}{4}\sin^2\theta \e^{2\i \phi} \hat{S}_{1-} \hat{S}_{2-}
    \end{cases}
\end{gather*}
$T_q$前一因子为轨道项,正比于球谐函数$Y_{2,q}(\theta,\phi)$,在轨道自由度给出选择定则
\begin{gather*}
    l'=l,l\pm2
    \\
    m'=m+q
    \\
    l,l' \geq 1
\end{gather*}

若粒子1和粒子2位置未固定,例如基态氢原子,选取基
\begin{gather*}
    \{\ket{\varphi_{1,0,0} \otimes \ket{\varepsilon_1,\varepsilon_2}}\}
\end{gather*}
$\hat{W}$在这组基下矩阵元为$0$,偶极-偶极相互作用会影响到激发态的超精细结构

\section{van der Vaals力}

考虑两氢原子A、B间偶极相互作用,A、B距离$R$,从A指向B方向的单位矢量$\bm{n}$,电子A相对质子A位矢$\bm{r}_A$,电子B相对质子B位矢$\bm{r}_B$,假设
\begin{gather*}
    R \ll |\bm{r}_A|,|\bm{r}_B|
    \\
    \intertext{偶极相互作用能}
    \hat{W}=\frac{q^2}{4\pi \epsilon_0 R^3}(\hat{X}_A \hat{X}_B+\hat{Y}_A \hat{Y}_B-2\hat{Z}_A \hat{Z}_B)
\end{gather*}

\subsection{两个1s氢原子}

\begin{gather*}
    \hat{H}=\hat{H}_{0A}+\hat{H}_{0B}+\hat{W}
    \\
    \hat{H}_{0A}+\hat{H}_{0B} \ket{\varphi_{nlm}^A \varphi_{n'l'm'}^B}=(E_n+E_{n'}) \ket{\varphi_{nlm}^A \varphi_{n'l'm'}^B}
\end{gather*}

一阶修正
\begin{gather*}
    \varepsilon_1=\inpro{\varphi_{1,0,0}^A \varphi_{1,0,0}^B}{\hat{W} \vert \varphi_{1,0,0}^A \varphi_{1,0,0}^B}=0
\end{gather*}

二阶修正
\begin{gather*}
    \varepsilon_2=-\frac{C}{R^6}
    \\
    C=(\frac{e^2}{4\pi \epsilon_0})^2 \displaystyle \sum_{nlm} \sum_{n'l'm'} \frac{|\inpro{\varphi_{nlm}^A \varphi_{n'l'm'}^B}{\hat{X}_A \hat{X}_B+\hat{Y}_A \hat{Y}_B-2\hat{Z}_A \hat{Z}_B \vert \varphi_{1,0,0}^A \varphi_{1,0,0}^B}|^2}{2E_I-E_n-E_{n'}}
    \\
    \intertext{忽略激发态能量}
    C\approx (\frac{e^2}{4\pi \epsilon_0})^2 \frac{1}{2E_I} \inpro{\varphi_{1,0,0}^A \varphi_{1,0,0}^B}{(\hat{X}_A \hat{X}_B+\hat{Y}_A \hat{Y}_B-2\hat{Z}_A \hat{Z}_B)^2 \vert \varphi_{1,0,0}^A \varphi_{1,0,0}^B}
    \\
    \intertext{由1s态球对称性,}
    C\approx (\frac{e^2}{4\pi \epsilon_0})^2 \frac{1}{2E_I} \times 6|\inpro{\varphi_{1,0,0}}{\frac{\hat{\bm{R}}^2}{3} \vert \varphi_{1,0,0}}|^2=6\frac{q^2}{4\pi \epsilon_0} a_0^5
    \\
    \varepsilon_2 \approx -6\frac{e^2}{4\pi \epsilon_0 R} (\frac{a_0}{R})^5
\end{gather*}

A、B偶极矩随机涨落无关,相互作用平均值为0,没有一阶修正。但A的涨落影响电场,对B产生感应电偶极矩,反过来作用到A上,产生二阶修正。

若计入推迟效应,相互作用能以$\frac{1}{R^7}$减小

\subsection{1s氢原子和2p氢原子}

特征子空间基
\begin{gather*}
    \{\ket{\varphi_{1,0,0}^A \varphi_{2,0,0}^B},\ket{\varphi_{2,0,0}^A \varphi_{1,0,0}^B},\ket{\varphi_{1,0,0}^A \varphi_{2,1,m}^B},\ket{\varphi_{2,1,m}^A \varphi_{1,0,0}^B}\}
\end{gather*}

一级修正
\begin{itemize}
    \item 由$\hat{R}_{Ai},\hat{R}_{Bi}$是奇性的,$\hat{W}$只能联系$\ket{\varphi_{1,0,0}^A}$和$\ket{\varphi_{2,1,m}^A}$,$\ket{\varphi_{1,0,0}^B}$和$\ket{\varphi_{2,1,B}^A}$
    \item 绕连线旋转,相互作用不变,$[\hat{W},\hat{L}_A+\hat{L}_B]=0$,$\hat{W}$只能联系$\hat{L}_A+\hat{L}_B$特征值相等的态
\end{itemize}
可以得到,
\begin{gather*}
    \inpro{\varphi_{1,0,0}^A \varphi_{2,1,m}^B}{\hat{W} \vert \varphi_{2,1,m}^A \varphi_{1,0,0}^B}=\frac{k_m}{R^3}
\end{gather*}
其余矩阵元均为$0$。在对应于$m$的子空间中对角化
\begin{gather*}
    \begin{bmatrix}
        0 & \frac{k_m}{R^3} \\
        \frac{k_m}{R^3} & 0
    \end{bmatrix}
\end{gather*}
特征值$\pm \frac{k_m}{R^3}$,特征向量
\begin{gather*}
    \frac{1}{\sqrt{2}}(\ket{\varphi_{1,0,0}^A \varphi_{2,1,m}^B}+\ket{\varphi_{2,1,m}^A \varphi_{1,0,0}^B})
    \\
    \frac{1}{\sqrt{2}}(\ket{\varphi_{1,0,0}^A \varphi_{2,1,m}^B}-\ket{\varphi_{2,1,m}^A \varphi_{1,0,0}^B})
\end{gather*}

\section{体积效应}

类氢原子中,将原子核视为半径$\rho_0$、电荷$Ze$的均匀带电球体,电势
\begin{gather*}
    V(r)=
    \begin{cases}
        -\frac{Ze^2}{4\pi \epsilon_0 r} \quad r > \rho_0
        \\
        \frac{Ze^2}{8\pi \epsilon_0 \rho_0}((\frac{r}{\rho_0})^2-3) \quad 0 \leq r \leq \rho_0
    \end{cases}
\end{gather*}
微扰
\begin{gather*}
    W(r)=
    \begin{cases}
        0 \quad r > \rho_0
        \\
        \frac{Ze^2}{8\pi \epsilon_0 \rho_0}((\frac{r}{\rho_0})^2+\frac{2\rho_0}{r}-3) \quad 0 \leq r \leq \rho_0
    \end{cases}
    \intertext{假设}
    \rho_0 \ll a_0
    \\
    R_{nl}(r) \approx R_{nl}(0)
    \\
    \inpro{\varphi_{nlm}}{\hat{W} \vert \varphi_{n'l'm'}}=\frac{Ze^2}{40\pi \epsilon_0} \rho_0^2 |R_{nl}(0)|^2 \delta_{ll'} \delta_{mm'}
\end{gather*}

\section{变分法}

考虑保守体系,假设能级是离散且非简并的
\begin{gather*}
    \hat{H} \ket{\varphi_n}=E_n \ket{\varphi_n}\quad n=1,2,\cdots
    \\
    \left<H\right>=\frac{\inpro{\psi}{\hat{H} \vert \psi}}{\inpro{\psi}{\psi}} \geq E_0
\end{gather*}
当且仅当$\ket{\psi}$是$E_0$对应特征向量时,等号成立

Ritz定理:Hamiltonian的平均值在离散特征值附近是稳定的

将试探右矢限制在一个子空间中,通过求极小值得到Hamiltonian各个特征值

\subsection{一维谐振子}

\begin{gather*}
    \hat{H}=-\frac{\hbar^2}{2m} \dt{^2}{x^2} + \frac{1}{2} m\omega^2 x^2
\end{gather*}

指数试探函数
\begin{gather*}
    \psi_{\alpha}(x)=\e^{-\alpha x^2} \quad \alpha>0
    \\
    \left<H\right>(\alpha)=\frac{\hbar^2}{2m}\alpha+\frac{1}{8}m\omega^2 \frac{1}{\alpha}
    \\
    \intertext{极小值}
    \left<H\right>(\alpha_0)=E_0=\frac{1}{2}\hbar \omega
    \\
    \intertext{求$E_1$,选取}
    \psi_{\alpha}(x)=x\e^{-\alpha x^2} \quad \alpha>0
    \\
    \left<H\right>(\alpha)=\frac{3\hbar^2}{2m}\alpha+\frac{3}{8}m\omega^2 \frac{1}{\alpha}
    \\
    \intertext{极小值}
    \left<H\right>(\alpha_0)=E_1=\frac{3}{2}\hbar \omega
\end{gather*}

\subsection{固体中电子能带}

$N$个等间隔全同离子作用下,随着间距$R$减小,定态逐步取消定域,每个孤立原子的能级逐渐变为$N$个能级,展宽$\Delta$,若$N$很大,形成了容许能带

考察由等间隔无穷多正离子构成的无穷长直链,设电子与第$n$个离子构成原子的态为$\ket{v_n}$,假设$v_n(x)$的互相覆盖可以忽略,且构成正交归一基。只考虑相邻原子的耦合,$\hat{H}$的表示矩阵为
\begin{gather*}
    \hat{H} \to
    \begin{bmatrix}
        E_0 & -A & 0 & 0 & \cdots \\
        -A & E_0 & -A & 0 & \cdots \\
        0 & -A & E_0 & -A & \cdots \\
        0 & 0 & -A & E_0 & \cdots \\
        \cdots & \cdots & \cdots & \cdots & \cdots
    \end{bmatrix}
    \\
    \intertext{求特征向量,设特征向量为}
    \ket{\varphi}=\sum_{q=-\infty}^{+\infty} c_q \ket{v_q}
    \\
    c_q=\e^{\i kql}
    \\
    E(k)=E_0-2A\cos kl \quad k \in [-\frac{\pi}{l},\frac{\pi}{l}]
    \\
    \intertext{周期性}
    v_q(x)=v_0(x-ql)
    \\
    \intertext{得到}
    \varphi_k(x+l)=\e^{\i kl} \varphi_k(x)
\end{gather*}
这样的波函数称为Bloch函数

对于有限的原子数,忽略边界效应,引入周期性边界条件Born-von Kármán边界条件
\begin{gather*}
    \e^{\i kL}=1
    \\
    k_n=n\frac{2\pi}{L}
\end{gather*}
可以证明,在这样边界条件下求出的态密度$\rho(E)$与真实态密度相同
\begin{gather*}
    \intertext{Fourier变换求出定态}
    \chi(x,t)=\frac{1}{\sqrt{2\pi}} \int g(k)\e^{\i(kx-\frac{E(k)}{\hbar}t)} \d k
    \\
    \intertext{群速度}
    v_g=\frac{1}{\hbar} \left.\dt{E}{k} \right|_{k=k_0}=\frac{2Al}{\hbar}\sin k_0l
\end{gather*}
$k_0=\pm \frac{\pi}{l}$,光学类比Bragg反射

\subsection{$\mathrm{H_2^+}$离子的化学键}

采用Born-Oppenheimer近似,认为两质子固定,设质子距离$R$,电子与两质子距离$r_1$、$r_2$,换到椭球坐标$(\mu,\nu,\phi)$
\begin{gather*}
    \mu=\frac{r_1+r_2}{R}
    \\
    \nu=\frac{r_1-r_2}{R}
    \\
    \d V=\frac{R^3}{8}(\mu^2-\nu^2)\d \mu \d \nu \d \phi
\end{gather*}

变分法计算能量:原子轨道线性组合法
\begin{gather*}
    \intertext{Hamiltonian}
    \hat{H}=\frac{\hat{\bm{P}}^2}{2m}-\frac{e^2}{4\pi \epsilon_0 r_1}-\frac{e^2}{4\pi \epsilon_0 r_2}+\frac{e^2}{4\pi \epsilon_0 R}
    \\
    \intertext{两个1s态右矢组成的子空间}
    \inpro{\bm{r}}{\varphi_1}=\frac{1}{\sqrt{\pi a_0^3}} \e^{-\frac{r_1}{a_0}}
    \\
    \inpro{\bm{r}}{\varphi_2}=\frac{1}{\sqrt{\pi a_0^3}} \e^{-\frac{r_2}{a_0}}
    \\
    \intertext{试探右矢}
    \ket{\psi}=c_1\ket{\varphi_1}+c_2\ket{\varphi_2}
    \\
    \intertext{特征方程}
    \inpro{\varphi_i}{\hat{H} \vert \psi}=E \inpro{\varphi_i}{\psi}
    \\
    \intertext{设}
    S_{ij}=\inpro{\varphi_i}{\varphi_j}
    \\
    \begin{bmatrix}
        H_{11}-ES_{11} & H_{12}-ES_{12} \\
        H_{21}-ES_{21} & H_{22}-ES_{22}
    \end{bmatrix}
    \begin{bmatrix}
        c_1 \\ c_2
    \end{bmatrix}
    =0
    \\
    S_{11}=S_{22}=1
    \\
    S_{21}=S_{12}=S=\e^{-\frac{R}{a_0}}(1+\frac{R}{a_0}+\frac{1}{3}(\frac{R}{a_0})^2)
    \\
    H_{11}=H_{22}=-E_I+\frac{e^2}{4\pi \epsilon_0 R}-C
    \\
    \intertext{Coulomb积分}
    C=\frac{2a_0}{R}E_I(1-\e^{-\frac{2R}{a_0}}(1+\frac{R}{a_0}))
    \\
    H_{12}=H_{21}=-E_I S+\frac{e^2}{4\pi \epsilon_0 R}S-A
    \\
    \intertext{共振积分}
    A=2E_I \e^{-\frac{R}{a_0}}(1+\frac{R}{a_0})
    \\
    \intertext{特征值}
    E_+=-E_I+\frac{2a_0}{R}E_I+\frac{A-C}{1-S}
    \\
    E_-=-E_I+\frac{2a_0}{R}E_I-\frac{A+C}{1+S}
    \\
    \intertext{$E_+$对应特征态为反键合态}
    \ket{\psi_+}=\frac{1}{\sqrt{2(1-S)}}(\ket{\varphi_1}-\ket{\varphi_2})
    \\
    \ket{\psi_-}=\frac{1}{\sqrt{2(1+S)}}(\ket{\varphi_1}+\ket{\varphi_2})
\end{gather*}