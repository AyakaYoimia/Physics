\documentclass[lang=cn,newtx,12pt,scheme=chinese]{elegantbook}

\title{理论力学笔记}

\author{温晨煜}
\institute{清华大学致理书院}
\version{1.0}

\setcounter{tocdepth}{3}

\logo{THU logo.jpg}
\cover{cover.jpg}

% 本文档命令
\usepackage{array}
\newcommand{\ccr}[1]{\makecell{{\color{#1}\rule{1cm}{1cm}}}}
\def\d{\mathrm{d}}
\def\e{\mathrm{e}}
\def\i{\mathrm{i}}
\newcommand{\pt}[2]{\frac{\partial #1}{\partial #2}}
\newcommand{\dt}[2]{\frac{\d #1}{\d #2}}

% 修改标题页的橙色带
\definecolor{customcolor}{RGB}{32,178,170}
\colorlet{coverlinecolor}{customcolor}
\usepackage{cprotect}

\addbibresource[location=local]{reference.bib} % 参考文献,不要删除

\begin{document}

\maketitle
\frontmatter

\tableofcontents

\mainmatter

\chapter*{前言}
\markboth{前言}{前言}

本笔记记录了我在大学阶段再次学习理论力学时的一些想法,试图用微分几何和群论的观点,来重新审视理论力学。

\chapter{Hamilton力学}

\section{Hamilton正则方程}

Lagrangian微分
\begin{align}
   \d L = \pt{L}{q_i} \d q_i + \pt{L}{\dot{q_i}} \d \dot{q_i} + \pt{L}{t} \d t
\end{align}
对Lagrangian进行Legendre变换,定义Hamiltonian:
\begin{align}
   H(p,q,t) = p_i \dot{q_i} - L
\end{align}
Hamiltonian的微分
\begin{align}
   \d H = \dot{q_i} \d p_i - \dot{p_i} \d q_i - \pt{L}{t} \d t
\end{align}
得到
\begin{align}
   \dot{q_i} = \pt{H}{p_i},&\quad \dot{p_i}=-\pt{H}{q_i}
   \\
   \dt{H}{t} = &\pt{H}{t} = -\pt{L}{t}
\end{align}

\section{Poisson括号}

$f(p,q,t)$随时间的变化率:
\begin{align}
   \dt{f}{t} = \pt{f}{t} + \pt{f}{q_i} \dot{q_i} + \pt{f}{p_i} \dot{p_i} 
\end{align}
定义Poisson括号:
\begin{align}
   [f,H]=\pt{f}{q_i} \pt{H}{p_i} - \pt{f}{p_i} \pt{H}{q_i}
\end{align}
代入正则方程,则有
\begin{align}
   \dt{f}{t} = \pt{f}{t} + [f,H]
\end{align}

Poisson括号的性质
\begin{enumerate}
   \item 双线性
   \item 乘积性质$[f_1 f_2,g]=f_1[f_2,g]+f_2[f_1,g]$
   \item Jacobi等式$[f,[g,h]]+[g,[h,f]]+[h,[f,g]]=0$
\end{enumerate}

\section{作用量为端点的函数}

\subsection{Hamilton-Jacobi方程}

在真实轨道下,考虑端点的变化

$\eta_1(t)$和$\eta_2(t)$均为真实轨道,固定起始端点$(t_0,q(t_0))$,考察另一端点$(t,q(t)=\eta_1(t))$与$(t+\d t,q(t)+\d q(t)\eta_2(t+\d t))$
\begin{gather}
   \d S=p_i \d q_i - H \d t
   \\
   \pt{S}{q_i} = p_i,\quad \pt{S}{t}=-H
   \\
   \pt{S}{t}+H=0
\end{gather}

\subsection{Hamilton正则方程}

固定两端点:
\begin{gather}
   S=\int_{t_1}^{t_2} (p_i \d q_i - H \d t)
   \\
   \delta S = \int_{t_1}^{t_2} \delta (p_i \dot{q_i} - H )\d t \nonumber
   \\
   = \int_{t_1}^{t_2} [(\dot{q_i}-\pt{H}{p_i})-(\dot{p_i}+\pt{H}{q_i})]\d t
   \\
   \dot{q_i}=\pt{H}{p_i},\quad \dot{p_i}=-\pt{H}{q_i}
\end{gather}

\subsection{Maupertuis原理}

起点广义坐标$q(t_0)$和终点广义坐标$q(t_2)$固定,除起点外每一个时刻$t'$,时间可变$\delta t'$,考虑路径$\eta_1$变为$\eta_2$带来的变分。记等时变分$\Delta f(t') =f_2(t')-f_1(t')$,非等时变分$\delta f(t) = f_2(t+\delta t) - f_1(t)$ 
\begin{align}
   \delta q(t)= \eta_2(t+\delta t)-\eta_1(t)=0
\end{align}
作用量的变分
\begin{align}
   \delta S &= \int_{t_0}^{t+\delta t} L_2(t') \d t' - \int_{t_0}^{t} L_1(t') \d t' \nonumber
   \\
   &=\int_{t}^{t+\delta t} L_2(t') \d t' + \int_{t_0}^{t} \Delta L(t') \d t' \nonumber
   \\
   &= L(t) \delta t + \left . (\pt{L}{\dot{q_i}} \Delta q_i(t')) \right |_{t_0}^{t} + \int_{t_0}^{t} (\pt{L}{q_i}-\dt{}{t'}\pt{L}{\dot{q_i}}) \Delta q_i(t') \d t'
\end{align}
又有
\begin{align}
   \Delta q(t') &= \eta_2(t') - \eta_1(t') \nonumber
   \\
   &= \eta_2(t'+\delta t') -\eta_1(t') -\eta_2(t'+\delta t') + \eta_2(t') \nonumber
   \\
   &= \delta q - \dot{\eta_2} \delta t' \nonumber
   \\
   &\approx \delta q - \dot{q} \delta t' \quad \text{($\dot{q}$与$\dot{\eta_2}$之间的差为一阶小量)}
\end{align}
轨道两端广义固定,$\delta q(t_0)=\delta q(t)=0$,真实轨道Lagranian满足Lagrange方程,最后一项为$0$
\begin{align}
   \delta S &= -(p_i \dot{q_i}-L) \delta t \nonumber
   \\
   &=-E \delta t
\end{align}
另一方面,由积分后求变分,
\begin{align}
   \delta S &=\int (p_i \d q_i - H \d t) \nonumber
   \\
   &= \delta S_0 - E \delta t
\end{align}
定义简约作用量:
\begin{align}
   S_0 = \int p_i \d q_i
\end{align}
则得到Maupertuis原理
\begin{align}
   \delta S_0 = 0
\end{align}

考虑质量为$m$的简单粒子的运动,则有Jacobi原理
\begin{align}
   \delta \int \sqrt{2m(E-V)} \d s = 0
\end{align}

\section{正则变换}

考虑$(p,q,t)$变换为$(P,Q,t)$且仍满足正则方程,Hamiltonian由$h(q,p,t)$变为$H(Q,P,t)$
\begin{gather}
   \delta \int (p_i \d q_i - h(q,p,t)\d t) = 0
   \\
   \delta \int (P_i \d Q_i - H(Q,P,t)\d t) = 0
   \\
   p_i \d q_i - h(q,p,t)\d t = P_i \d Q_i - H(Q,P,t)\d t + \d F
   \\
   \d F = p_i \d q_i - P_i \d Q_i + (H-h) \d t
\end{gather}

$F(q,Q,t)$称为正则变换的生成函数
\begin{align}
   p_i=\pt{F}{q_i},\quad P_i=\pt{F}{Q_i},\quad H=h+\pt{F}{t}
\end{align}
以$(p,q,P,Q)$中四选二作为独立变量,可以得到四种正则变换,其余三种正则变换的生成函数可以由$F$做Legendre变换得到

正则变换不改变Poisson括号
\begin{align}
   [f,g]_{p,q}=[f,g]_{P,Q}
\end{align}

\section{Hamilton-Jacobi方程}

\begin{gather}
   \pt{S}{t}+h(q,\pt{S}{q},t)=0
   \\
   S=S(q,Q,t)+S_0
\end{gather}
$Q$为$f$个常数,$S_0$为任意相加常数,不会对系统产生影响

以$S(q,Q,t)$为生成函数做正则变换,变换后Hamiltonian
\begin{gather}
   H=h+\pt{S}{t}=0
\end{gather}

使用方法:
\begin{enumerate}
   \item 根据Hamiltonian写下Hamilton-Jacobi方程 \\
   在保守系统下是能量守恒定律
   \item 利用分离变量法,求出$S(q,Q,t)$ \\
   对于循环坐标$q_j$,$S=S'+Q_j q_j$,$S'$与$q_j$无关
   \item 求出新的广义动量$P_i=-\pt{S}{Q_i}$
   \item 反解出原广义坐标$q=q(Q,P,t)$,进而求出原广义动量$p_i=\pt{S}{q_i}$
\end{enumerate}

\subsection{Hamilton-Jacobi方程与量子力学波动方程}

取波函数
\begin{align}
   \psi(\bm{r},t)=\sqrt{\rho(\bm{r})}\e^{\frac{iS(\bm{r},t)}{\hbar}}
\end{align}
假定$\rho(\bm{r})$为常数,$S(\bm{r},t)\gg \hbar$

Schödinger方程
\begin{gather}
   \i \hbar \pt{\psi}{t} = -\frac{\hbar^2}{2m} \nabla^2 \psi + V\psi
\end{gather}
将波函数代入,
\begin{gather}
   [\frac{(\nabla S)^2}{2m}+V]+\pt{S}{t}=\frac{\i \hbar}{2m} \nabla^2 S
\end{gather}
在经典极限$\hbar \nabla^2 S \ll (\nabla S)^2$下,上式得到Hamilton-Jacobi方程。经典近似的条件是动量在de Broglie波长内变化不大,也就是势能在de Broglie波长内变化不大

\subsection{Hamilton-Jacobi方程与波动光学}

光的波函数$\phi(\bm{r},t)$满足波动方程
\begin{align}
   \nabla^2 \psi-\frac{n^2}{c^2} \pt{^2 \psi}{t^2}=0
\end{align}
在折射率随空间的变化缓慢时,波函数的解
\begin{gather}
   \psi = \e^{A(\bm{r})} \e^{ik_0 (L(\bm{r})-ct)}
\end{gather}
代入波动方程,
\begin{gather}
   \nabla^2 A+(\nabla A)^2 + k_0^2(n^2-(\nabla L)^2)=0
   \\
   \nabla^2 L+2\nabla A \cdot \nabla L = 0
\end{gather}
取短波近似(几何光学近似),介质在波长范围内变化很小,得到
\begin{gather}
   (\nabla L)^2=n(\bm{r})^2
\end{gather}
对比Hamilton-Jacobi方程(作用量$S=W-Et$)
\begin{gather}
   (\nabla W)^2 = 2m(E-V(\bm{r}))
\end{gather}
Fermat原理
\begin{gather}
   \delta \int n \d s=0
\end{gather}
对比Jacobi原理
\begin{gather}
   \delta \int \sqrt{2m(E-V)} \d s=0
\end{gather}

\end{document}
