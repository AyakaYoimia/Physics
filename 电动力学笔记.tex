\documentclass[lang=cn,newtx,12pt,scheme=chinese]{elegantbook}

\title{电动力学笔记}

\author{温晨煜}
\institute{清华大学致理书院}
\version{1.0}

\setcounter{tocdepth}{3}

\logo{THU logo.jpg}
\cover{cover.jpg}

% 本文档命令
\usepackage{array, pgfplots, mathrsfs}
\newcommand{\ccr}[1]{\makecell{{\color{#1}\rule{1cm}{1cm}}}}
\def\d{\mathrm{d}}
\def\i{\mathrm{i}}
\def\e{\mathrm{e}}
\def\pdd{\partial}
\def\Tr{\mathrm{Tr}}
\newcommand{\pt}[2]{\frac{\partial #1}{\partial #2}}
\newcommand{\dt}[2]{\frac{\d #1}{\d #2}}
\usetikzlibrary{arrows}
\pgfplotsset{compat=1.15}

% 修改标题页的橙色带
\definecolor{customcolor}{RGB}{32,178,170}
\colorlet{coverlinecolor}{customcolor}
\usepackage{cprotect}

\addbibresource[location=local]{reference.bib} % 参考文献,不要删除

\begin{document}

\maketitle
\frontmatter

\tableofcontents

\mainmatter

\chapter{电动力学基本规律}

\begin{introduction}
  \item 真空与介质中的Maxwell方程组
  \item 电磁势
  \item 电荷、能量、动量、角动量守恒
\end{introduction}

\section{真空中的Maxwell方程组}

\begin{theorem}[真空中的Maxwell方程组] \label{def:int} 
\[
\begin{cases}
   \nabla \cdot \bm{E} = \frac{\rho}{\epsilon_0}
   \\
   \nabla \times \bm{B} - \epsilon_0 \mu_0 \frac{\partial \bm{E}}{\partial t} = \mu_0 \bm{J}
   \\
   \nabla \times \bm{E} - \frac{\partial \bm{B}}{\partial t} = 0
   \\
   \nabla \cdot \bm{B} = 0
\end{cases}
\]
\end{theorem}

\begin{theorem}[洛伦兹力公式] \label{def:int}
力的密度
   \begin{equation*}
      \bm{f} = \rho \bm{E} + \bm{J} \times \bm{B}
   \end{equation*}
\end{theorem}

\begin{property}
Maxwell方程组的性质
   \begin{enumerate}
   \item 线性
   \item 洛伦兹协变性
   \item 规范对称性
   \item 分立对称性
   \begin{enumerate}
      \item 空间反射变换(宇称变换)$\bm{x} \to -\bm{x}$,有$\bm{E} \to -\bm{E}, \bm{B} \to \bm{B}$
      \begin{enumerate}
         \item 称在空间反射变换下改变符号的三维矢量为极矢量
         \item 称在空间反射变换下不变的三维矢量为轴矢量
      \end{enumerate}
      \item 时间反演变换$t \to -t$下,有$\bm{E} \to \bm{E}, \bm{B} \to \bm{-B}$
   \end{enumerate}
   \end{enumerate}
\end{property}   



\section{电磁势与规范对称性}

\begin{definition}[电磁势] \label{def:int}
   关于电磁势的方程是二阶偏微分方程

   电磁势$(\Phi, \bm{A})$中,$\Phi$称为标量势,简称标势;$\bm{A}$称为矢量势,简称矢势.
\end{definition}

电磁场与电磁势的关系为
\begin{equation}
   \begin{cases}
      \bm{B} = \nabla \times \bm{A}
      \\
      \bm{E} = -\nabla \Phi - \frac{\partial \bm{A}}{\partial t}
   \end{cases}
\end{equation}

电磁势的选取不是唯一的。若电磁势$(\Phi',\bm{A}')$与$(\Phi,\bm{A})$对应于相同的电磁场$(\bm{E},\bm{B})$,则二者满足规范变换
\begin{equation}
   \begin{cases}
      \bm{A}'=\bm{A}+\nabla \Lambda
      \\
      \Phi'=\Phi - \frac{\partial \Lambda}{\partial t}
   \end{cases}
\end{equation}
其中$\Lambda$为任意标量场。电磁场这样的对称性称为规范对称性.

经典电动力学中,电磁势并不是可观测的物理量,需要加入规范条件来确定.
\begin{enumerate}
\item Lorenz规范:
\begin{equation}
   \nabla \cdot \bm{A} + \frac{1}{c^2} \frac{\partial \Phi}{\partial t} = 0
\end{equation}
齐次方程自动满足,代入非齐次方程,得到波动方程
\begin{equation}
   \begin{cases}
      \nabla^2 \Phi -\frac{1}{c^2} \frac{\partial^2 \Phi}{\partial t^2} = -\frac{\rho}{\epsilon_0}
      \\
      \nabla^2 \bm{A} -\frac{1}{c^2} \frac{\partial^2 \bm{A}}{\partial t^2} = -\mu_0 \bm{J}
   \end{cases}
\end{equation}
\item Coulomb规范:
\begin{equation}
   \nabla \cdot \bm{A} = 0
\end{equation}
\end{enumerate}

\section{介质中的Maxwell方程组}
\begin{definition}[电极化强度]
   介质中单位体积中的平均电偶极矩称为电极化强度,记为$\bm{P}$
   
   束缚电荷密度
   \begin{equation}
      \rho_b = -\nabla \cdot \bm{P}
   \end{equation}

   介质界面束缚电荷面密度
   \begin{equation}
      \sigma_b = \bm{n} \cdot \bm{P}
   \end{equation}

   束缚电流密度
   \begin{equation}
      \bm{J}_b = \frac{\partial \bm{P}}{\partial t}
   \end{equation}
\end{definition}

\begin{definition}[磁化强度]
   介质单位体积中的平均磁偶极矩称为磁化强度,记为$\bm{M}$

   分子电流密度
   \begin{equation}
      \bm{J}_m = \nabla \times \bm{M}
   \end{equation}

   介质界面面电流密度
   \begin{equation}
      \bm{K}_m = -\bm{n} \times \bm{M}
   \end{equation}
\end{definition}

\begin{theorem}[线性介质中的Maxwell方程组]
   \[
   \begin{cases}
      \nabla \cdot \bm{D} = \rho
      \\
      \nabla \times \bm{H}-\frac{\partial \bm{D}}{\partial t} = \bm{J}
      \\
      \nabla \times \bm{E} - \frac{\partial \bm{B}}{\partial t} = 0
      \\
      \nabla \cdot \bm{B} = 0
   \end{cases}
   \]
考虑线性、均匀、各向同性介质,存在本构关系
\[
\begin{cases}
   \bm{D} = \epsilon_0 \bm{E}+\bm{P}
   \\
   \bm{H} = \frac{1}{\mu_0} \bm{B}-\bm{M}
\end{cases}
\]
\end{theorem}
在一般情形下,$\bm{D}$和$\bm{H}$与$\bm{E}$和$\bm{B}$之间存在本构关系,此时$\bm{D}$和$\bm{H}$由Maxwell方程组前二式定义.

\begin{theorem}[介质交界面的边界条件]
   $\bm{n}$是由介质1指向介质2的单位法向量
\[
\begin{cases}
   \bm{n} \cdot(\bm{D}_2-\bm{D}_1) = \sigma
   \\
   \bm{n} \cdot (\bm{B}_2-\bm{B}_1) = 0
   \\
   \bm{n} \times (\bm{E}_2-\bm{E}_1) = 0
   \\
   \bm{n} \times (\bm{H}_2-\bm{H}_1) = \bm{K}
\end{cases}
\]
\end{theorem}

\section{电磁守恒律}

\begin{theorem}[电荷守恒定律]
   Maxwell方程组自动满足电荷守恒
   \[
   \frac{\partial \rho}{\partial t} + \nabla \cdot \bm{J} = 0
   \]
\end{theorem}

\begin{theorem}[能量守恒]
   电磁场能量密度
   \[
   u = \frac{1}{2}(\bm{E} \cdot \bm{D} + \bm{B} \cdot \bm{H})
   \]
   能流密度(又称为Poynting矢量)
   \[
   \bm{S} = \bm{E} \times \bm{H}
   \]
   磁场对电流不做功,电场对电流做功功率$W$,则有
   \[
   -W = -\int_{V} \bm{J} \cdot \bm{E} \, \mathrm{d}V = \int_{V} (\frac{\partial u}{\partial t} + \nabla \cdot \bm{S}) \, \mathrm{d}V
   \]
\end{theorem}

\begin{theorem}[动量守恒]
\[
\frac{\mathrm{d}}{\mathrm{d}t}(P_i^{source}+P_i^{field}) = \oint_{\partial V} T_{ij}\, \mathrm{d}S_j
\]
$\bm{P}^{source}$是带电粒子的动量,$\bm{P}^{field}$是电磁场的动量

定义电磁场动量密度
\[
\bm{g} = \epsilon_0 \bm{E} \times \bm{B} = \epsilon_0 \mu_0 \bm{S}
\]

应力张量
\[
T_{ij} = \epsilon_0 (E_i E_j + c^2 B_i B_j - \frac{1}{2}(\bm{E} \cdot \bm{E}+c^2 \bm{B} \cdot \bm{B})\delta_{ij})
\]
\end{theorem}

\begin{theorem}
   角动量密度
   \[
   \mathcal{M} = \bm{x} \times \bm{g} = \frac{1}{c^2} \bm{E} \times \bm{H}
   \]
\end{theorem}



\chapter{静电学}

\section{Poisson方程}

\begin{theorem}
引入静电势$\Phi$,有  
\[\bm{E} = -\nabla \Phi\]
得到Poisson方程
\[\nabla^2 \Phi(\bm{x}) = -\frac{\rho(\bm{x})}{\epsilon}\]
若$\rho(\bm(x)) = 0$,有Laplace方程
\[\nabla^2 \Phi = 0\]
\end{theorem}

下面是Poisson方程的解法

\subsection{无边界的无穷大空间中的解}
\[
\Phi(\bm{x}) = \frac{1}{4\pi \epsilon} \int \frac{\rho(\bm{x'})}{|\bm{x}-\bm{x'}|} \, \mathrm{d}V'
\]

\subsection{Green函数法}
边界条件:
\begin{enumerate}
   \item Dirichlet边界条件:已知$\left . \Phi \right |_{S}$
   \item Neumann边界条件:已知$\left . \pt{\Phi}{n} \right |_{S}$
\end{enumerate}

Green函数$G(\bm{x},\bm{x'})$在$V$内满足
\begin{gather}
   \begin{cases}
      \nabla_{\bm{x'}}^2 G(\bm{x},\bm{x'})=-4\pi \delta^3(\bm{x}-\bm{x'}) \\
      \text{适当的边界条件}
   \end{cases}
\end{gather}
则
\[
\Phi(\bm{x})=\frac{1}{4\pi \epsilon_0} \int_V G(\bm{x},\bm{x'})\rho(\bm{x'}) \d V' + \frac{1}{4\pi} \oint_S (G(\bm{x},\bm{x'}) \pt{\Phi(\bm{x'})}{n'}-\Phi(\bm{x'}) \pt{G(\bm{x},\bm{x'})}{n'}) \d S'
\]

$G(\bm{x},\bm{x'})$可以写成如下形式:
\[
G(\bm{x},\bm{x'})=\frac{1}{|\bm{x}-\bm{x'}|} + F(\bm{x},\bm{x'})
\]
通过选取$F(\bm{x},\bm{x'})$来使得求解出的静电势$\Phi(\bm{x})$满足对应的边界条件.

(1)Dirichlet边界条件
\begin{gather*}
   G_D(\bm{x},\bm{x'}) = 0, \: \bm{x'} \in S \\
   \Phi(\bm{x})=\frac{1}{4\pi \epsilon_0} \int_V G(\bm{x},\bm{x'})\rho(\bm{x'}) \d V' + \frac{1}{4\pi} - \oint_S \Phi(\bm{x'}) \pt{G(\bm{x},\bm{x'})}{n'} \d S'
\end{gather*}

(2)Neumann边界条件
\begin{gather*}
   G_N(\bm{x},\bm{x'}) = -\frac{4\pi}{A_S}, \: \bm{x'} \in S \\
   \Phi(\bm{x})= \langle \Phi \rangle_S + 
   \frac{1}{4\pi \epsilon_0} \int_V G(\bm{x},\bm{x'})\rho(\bm{x'}) \d V' + \frac{1}{4\pi} \oint_S G(\bm{x},\bm{x'}) \pt{\Phi(\bm{x'})}{n'} \d S'
\end{gather*}

\subsection{静电镜像法}

\begin{theorem}[唯一性定理]
   空间$V$内满足Poisson方程和边界条件的解$\Phi(\bm{x})$是唯一的
\end{theorem}

\begin{definition}[反演变换]
   反演变换半径$R$,中心取为原点,$n$维欧氏空间中$\bm{x}$在反演变换下的像为
   \[
   \bm{\tilde{x}}=\frac{R^2}{\|\bm{x}\|^2} \bm{x}
   \]
   球面经过反演变换仍是球面
\end{definition}

\subsection{Laplace方程的分离变量法}

\textbf{(1)直角坐标系}

\begin{gather*}
   \Phi(\bm{x})=X(x)Y(y)Z(z) \\
   X(x) \propto \e^{k_1x},\quad Y(y) \propto \e^{k_2y},\quad Z(z) \propto \e^{k_3x} \\
   \text{满足}k_1^2+k_2^2+k_3^2=0
\end{gather*}

$k_i$的取值可以为实数或纯虚数

\textbf{(2)柱坐标系}

\begin{gather*}
   \Phi(\bm{x})=R(r) \Phi(\phi) Z(z) \\
   Z(z) \propto \e^{\pm kz},\quad \Phi(\phi) \propto \e^{\pm im\phi},\quad R(r) \propto J_m(kr),N_m(kr)
\end{gather*}

$m$为正整数,$k$为正实数或虚部为正的纯虚数

\textbf{(3)球坐标系}

\begin{equation*}
   \Phi(r,\bm{n})=\displaystyle \sum_{l=0}^{+\infty} \sum_{m=-l}^{l} (A_{lm}r^l+\frac{B_{lm}}{r^{l+1}})Y_{lm}(\bm{n})
\end{equation*}

\section{多极展开}

在远处的$\bm{x}$点,$r=|\bm{x}| \gg |\bm{x'}|$
\begin{align*}
   \Phi(\bm{x})&=\frac{1}{4\pi \epsilon_0} \int_V \frac{\rho(\bm{x'})}{|\bm{x}-\bm{x'}|} \d V'
   \\
   &=\frac{1}{4\pi \epsilon_0} \displaystyle \sum_{l=0}^{+\infty} \sum_{m=-l}^{l} \frac{4\pi}{2l+1} (\int_V Y_{lm}^*(\bm{n'})(r')^l \rho(\bm{x'}) \d V') \frac{Y_{lm}(\bm{n})}{r^{l+1}}
   \\
   &=\frac{1}{\epsilon_0} \displaystyle \sum_{l=0}^{+\infty} \sum_{m=-l}^{l} \frac{1}{2l+1} q_{lm} \frac{Y_{lm}(\bm{n})}{r^{l+1}}
\end{align*}

其中多极矩
\begin{gather*}
   q_{lm}=\int_V Y_{lm}^*(\bm{n'})(r')^l \rho(\bm{x'}) \d V'
\end{gather*}

另一种展开方式:
\begin{gather*}
   \Phi(\bm{x})=\frac{1}{4 \pi \epsilon_0}(\frac{Q}{r}-\bm{p}\cdot \nabla \frac{1}{r}+\frac{1}{6} \displaystyle \sum_{i,j} D_{ij} \pt{^2}{x_i \partial x_j} \frac{1}{r} + \cdots)
   \\
   \bm{p}=\int_V \bm{x'}\rho(\bm{x'}) \d V'
   \\
   D_{ij}=\int_V (3x'_i x'_j-r'^2\delta_{ij}) \rho(\bm{x'}) \d V'
\end{gather*}

电偶极矩静电场
\begin{gather*}
   \Phi(\bm{x})=\frac{1}{4\pi \epsilon_0} \frac{\bm{p}\cdot \bm{x}}{r^3}
   \\
   \bm{E}(\bm{x})=\frac{1}{4\pi \epsilon_0}(\frac{3\bm{n}(\bm{n}\cdot \bm{p})-\bm{p}}{r^3}-\frac{4\pi}{3}\bm{p}\delta^3(\bm{x}))
\end{gather*}

\chapter{静磁学}

\section{Poisson方程}

\begin{theorem}
   引入磁矢势后,磁矢势$\bm{A}$满足Poisson方程:
   \begin{equation*}
      \nabla^2 \bm{A}=-\mu \bm{J}
   \end{equation*}
   满足Coulumb规范$\nabla \cdot \bm{A}=0$
   电流连续性方程$\nabla \cdot \bm{J}=0$
\end{theorem}

无边界的无穷大空间中的解为
\begin{equation*}
   \bm{A}(\bm{x})=\frac{\mu}{4\pi} \int \frac{\bm{J}(\bm{x'})}{|\bm{x}-\bm{x'}|} \d V'
\end{equation*}

\section{磁场的能量}

\begin{align*}
   U &= \frac{1}{2} \int \bm{H} \cdot \bm{B} \d V = \frac{1}{2 \mu} \int (\nabla \times \bm{A})^2 \d V
   \\
   (\nabla \times \bm{A})^2 &= {\epsilon^i}_{jk} {\epsilon^i}_{lm} (\partial^j A^k)(\partial^l A^m)
   \\
   &=(\delta_{jl}\delta_{km}-\delta_{jm}\delta_{kl})(\partial^j A^k)(\partial^l A^m)
   \\
   &=(\partial^j A^k)(\partial^j A^k)-(\partial^j A^k)(\partial^k A^j)
   \\
   &=\partial^j(A^k \partial^j A^k)-A^k(\partial^j \partial^j A^k)-\partial^j(A^k \partial^k A^j)+A^k(\partial^j \partial^k A^j)
   \\
   &=(\bm{A} \cdot \nabla)(\nabla \cdot \bm{A})+\partial^j(A^k \partial^j A^k)-A^k(\partial^j \partial^j A^k)-\partial^j(A^k \partial^k A^j)
\end{align*}
第一项由于Coulumb规范为0,全微分对全空间积分没有贡献,因此
\begin{align*}
   U=\frac{1}{2} \int \bm{A} \cdot \bm{J} \d V   
\end{align*}

\section{磁多极展开}\label{sec:3.3}

在远处的$\bm{x}$点,$r=|\bm{x}| \gg |\bm{x'}|$

\begin{equation*}
   A_i(\bm{x})=\frac{\mu}{4\pi r}\int J_i(\bm{x'}) \d V'+\frac{\mu x_j}{4\pi r^3} \int J_i(\bm{x'})x'_j \d V'+\cdots
\end{equation*}
对于第一项,
\begin{gather*}
   \int \partial'_i(J_i(\bm{x'})x'_j) \d V'=0
   \\
   \int x'_j \partial'_i(J_i(\bm{x'})) \d V' + \int J_j(\bm{x'}) \d V'=0
   \\
   \int J_j(\bm{x'}) \d V'=0   
\end{gather*}

对于第二项,
\begin{gather*}
   \int \partial'_k(J_k(\bm{x'})x'_i x'_j) \d V'=0
   \\
   \int (x'_i J_j(\bm{x'})+x'_j J_i(\bm{x'})) \d V'=0
\end{gather*}
说明对称部分没有贡献,考虑反对称部分,
\begin{align*}
   x_j \int J_i(\bm{x'})x'_j \d V' &= -\frac{1}{2} x_j \int (x'_i J_j(\bm{x'})-x'_j J_i(\bm{x'})) \d V'
   \\
   &=-\frac{1}{2} \epsilon_{ijk} x_j \int (\bm{x'}\times \bm{J}(\bm{x'}))_k \d V'
   \\
   &=-\frac{1}{2}(\bm{x}\times \int (\bm{x'}\times \bm{J}(\bm{x'})) \d V')_i
\end{align*}

定义磁矩
\begin{equation*}
   \bm{m}=\frac{1}{2} \int (\bm{x'}\times \bm{J}(\bm{x'})) \d V'
\end{equation*}

磁场
\begin{gather*}
   \bm{A}(\bm{x})=\frac{\mu}{4\pi} \frac{\bm{m}\times \bm{x}}{r^3}
   \\
   \bm{B}(\bm{x})=\frac{\mu}{4\pi}(\frac{3\bm{n}(\bm{n}\cdot \bm{m})-\bm{m}}{r^3}+\frac{8\pi}{3}\bm{m}\delta^3(\bm{x}))
\end{gather*}

\begin{enumerate}
   \item 磁矩的受力:$\bm{F}=\nabla(\bm{m}\cdot \bm{B})$
   \item 磁矩受到的力矩:$\bm{M}=\bm{m}\times \bm{B}$
   \item 磁矩在外场中的能量:$U=-\bm{m}\cdot \bm{B}$
\end{enumerate}

\section{磁标势}
\begin{definition}
   无自由电流处,$\nabla \times \bm{H}=0$,可以引入磁标势$\Phi_M(\bm{x})$
   \begin{equation*}
      \bm{H}(\bm{x})=-\nabla \Phi_M(\bm{x})
   \end{equation*}
   满足类Poisson方程
   \begin{equation*}
      \nabla^2 \Phi_M(\bm{x})=-(-\nabla \cdot \bm{M})
   \end{equation*}
   等效磁荷密度与等效磁荷面密度
   \begin{equation*}
      \rho_M = -\nabla \cdot \bm{M},\quad \sigma_M = \bm{n} \cdot \bm{M}
   \end{equation*}
\end{definition}
   
考虑两种简单情形:
\begin{enumerate}
   \item 线性介质:$\bm{B}=\mu \bm{H}$,$\nabla^2 \Phi_M(\bm{x})=0$
   \item 硬铁磁体:$\bm{M}(\bm{x})$是已知矢量场
\end{enumerate}

\chapter{电磁波的传播}

能流
\begin{equation*}
   \bm{S}=\frac{1}{2} \sqrt{\frac{\epsilon}{\mu}} |\bm{E}_0|^2 \bm{n}
\end{equation*}

能量密度
\begin{equation*}
   u=\frac{\epsilon}{2}|\bm{E}_0|^2
\end{equation*}

\section{电磁波在介质界面的折射与反射}

\definecolor{qqwuqq}{rgb}{0,0.39215686274509803,0}
\definecolor{ffzzqq}{rgb}{1,0.6,0}
\definecolor{xdxdff}{rgb}{0.49019607843137253,0.49019607843137253,1}
\definecolor{ududff}{rgb}{0.30196078431372547,0.30196078431372547,1}
\begin{tikzpicture}[line cap=round,line join=round,>=triangle 45,x=1cm,y=1cm]
\begin{axis}[
x=1cm,y=1cm,
axis lines=middle,
ymajorgrids=true,
xmajorgrids=true,
xmin=-6,
xmax=6,
ymin=-6,
ymax=6,
xtick={-6,-5,...,6},
ytick={-6,-5,...,6},]
\clip(-10.440309090909096,-7.395563636363631) rectangle (13.523327272727276,7.8044363636363565);
\draw [shift={(0,0)},line width=2pt,color=qqwuqq,fill=qqwuqq,fill opacity=0.10000000149011612] (0,0) -- (90:0.5454545454545456) arc (90:153.434948822922:0.5454545454545456) -- cycle;
\draw [shift={(0,0)},line width=2pt,color=qqwuqq,fill=qqwuqq,fill opacity=0.10000000149011612] (0,0) -- (26.565051177077986:0.5454545454545456) arc (26.565051177077986:90:0.5454545454545456) -- cycle;
\draw [shift={(0,0)},line width=2pt,color=qqwuqq,fill=qqwuqq,fill opacity=0.10000000149011612] (0,0) -- (-90:0.5454545454545456) arc (-90:-69.39030706246793:0.5454545454545456) -- cycle;
\draw [line width=2pt,domain=-10.440309090909096:0] plot(\x,{(-0--2*\x)/-4});
\draw [line width=2pt,domain=0:13.523327272727276] plot(\x,{(-0-2.34*\x)/0.88});
\draw [line width=2pt,domain=0:13.523327272727276] plot(\x,{(-0--2*\x)/4});
\draw [->,line width=2pt] (-4,2) -- (-3.176,1.588);
\draw [->,line width=2pt] (4,2) -- (4.704,2.352);
\draw [->,line width=2pt] (0.88,-2.34) -- (1.24174336,-3.3019084800000003);
\draw [->,line width=2pt,color=ffzzqq] (-4,2) -- (-3.608,2.784);
\draw [->,line width=2pt,color=ffzzqq] (4,2) -- (3.616,2.768);
\draw [->,line width=2pt,color=ffzzqq] (0.88,-2.34) -- (1.7213516799999993,-2.02359424);
\begin{scriptsize}
\draw [fill=ududff] (-4,2) circle (2.5pt);
\draw[color=ududff] (-4.2766727272727305,1.958981818181815) node {$E_{1s}$};
\draw [fill=ududff] (0.88,-2.34) circle (2.5pt);
\draw[color=ududff] (0.41423636363636196,-2.1864727272727267) node {$E_{2s}$};
\draw [fill=xdxdff] (0,0) circle (2.5pt);
\draw [fill=ududff] (4,2) circle (2.5pt);
\draw[color=ududff] (4.505145454545454,1.8498909090909061) node {$E'_{1s}$};
\draw[color=black] (-3.367581818181821,2.1408) node {$k_{1}$};
\draw[color=black] (4.505145454545454,2.522618181818178) node {$k'_{1}$};
\draw[color=black] (1.2506,-2.513745454545454) node {$k_{2}$};
\draw[color=ffzzqq] (-4.07667272727273,3.0862545454545414) node {$E_{1p}$};
\draw[color=ffzzqq] (3.432418181818181,2.6317090909090872) node {$E'_{1p}$};
\draw[color=ffzzqq] (1.2142363636363622,-1.5682909090909094) node {$E_{2p}$};
\draw[color=qqwuqq] (-0.5675818181818202,1.0135272727272704) node {$i_{1}$};
\draw[color=qqwuqq] (0.505145454545453,0.9226181818181796) node {$i'_{1}$};
\draw[color=qqwuqq] (0.21423636363636192,-0.8773818181818189) node {$i_{2}$};
\end{scriptsize}
\end{axis}
\end{tikzpicture}

\begin{gather*}
   r_p=\frac{E'_{1p}}{E_{1p}}=\frac{\frac{n_2}{\mu_2}\cos i_1-\frac{n_1}{\mu_1}\cos i_2}{\frac{n_2}{\mu_2}\cos i_1+\frac{n_1}{\mu_1}\cos i_2}
   \\
   r_s=\frac{E'_{1s}}{E_{1s}}=\frac{\frac{n_1}{\mu_1}\cos i_1-\frac{n_2}{\mu_2}\cos i_2}{\frac{n_1}{\mu_1}\cos i_1+\frac{n_2}{\mu_2}\cos i_2}
   \\
   t_p=\frac{E_{2p}}{E_{1p}}=\frac{2\frac{n_1}{\mu_1}\cos i_1}{\frac{n_2}{\mu_2}\cos i_1+\frac{n_1}{\mu_1}\cos i_2}
   \\
   t_s=\frac{E_{2s}}{E_{1s}}=\frac{2\frac{n_1}{\mu_1}\cos i_1}{\frac{n_1}{\mu_1}\cos i_1+\frac{n_2}{\mu_2}\cos i_2}
\end{gather*}
大多数情况下$\mu_1 \approx \mu_2$

入射角为Bruster角
\begin{equation*}
   i_B=\arctan \frac{n_2}{n_1}
\end{equation*}
反射波只有垂直分量

全反射角
\begin{equation*}
   i_0=\arcsin \frac{n_2}{n_1}
\end{equation*}

\section{电磁波在导电介质中的传播}

Maxwell方程组给出
\begin{equation*}
   \frac{1}{\mu} \nabla \times \bm{B}=-\i \omega (\epsilon+\i \frac{\sigma}{\omega})\bm{E}
\end{equation*}
复介电常数
\begin{equation*}
   \tilde{\epsilon}=\epsilon+\i \frac{\sigma}{\omega}
\end{equation*}
色散关系
\begin{equation*}
   k^2=\mu \epsilon \omega^2 (1+\i \frac{\sigma}{\omega \epsilon})
\end{equation*}
\[
\begin{cases}
   t=0 \text{时导体中已经存在电磁场:} \quad \omega \text{是复数,} \bm{k} \text{是实数} \\
   \text{外界电磁波入射到导体:} \quad \omega \text{是实数,} \bm{k} \text{是复数}
\end{cases}
\]
解得
\begin{equation*}
   k=\omega \sqrt{\mu \epsilon}(\sqrt{\frac{\sqrt{1+\frac{\sigma^2}{\epsilon^2 \omega^2}}+1}{2}}+\i \sqrt{\frac{\sqrt{1+\frac{\sigma^2}{\epsilon^2 \omega^2}}-1}{2}})
\end{equation*}
\begin{enumerate}
   \item 对于不良导体,$\frac{\sigma}{\omega \epsilon}\ll 1$:$k \approx \sqrt{\mu \epsilon}\omega+\i\frac{1}{2}\sqrt{\frac{\mu}{\epsilon}}\sigma$
   \item 对于良导体,$\frac{\sigma}{\omega \epsilon}\gg 1$:$k \approx (1+\i)\sqrt{\frac{\omega \mu \sigma}{2}}$ \\
   对于良导体,传导电流远大于位移电流,称为准静态近似,有扩散方程\\
   \[\nabla^2 \bm{H} = \mu \sigma \pt{\bm{H}}{t}\]
\end{enumerate}

\section{介质色散的经典模型}

介质中的束缚电子视为阻尼简谐振子,在外场下受迫振动,电子产生电偶极矩
\begin{equation*}
   \bm{p}=\frac{e^2}{m} \frac{\bm{E}_0}{\omega_0^2-\omega^2-\i \omega \gamma}
\end{equation*}
相对介电常数
\begin{equation*}
   \epsilon_r=1+\frac{n e^2}{\epsilon_0 m} \displaystyle \sum_{i} \frac{f_i}{\omega_i^2-\omega^2-\i \omega \gamma_i}
\end{equation*}
$n$为原子体密度,$f_i$为权重,$e$为电子电荷,$m$为电子质量

对于导体,$\omega_0=0$,对比之前可得
\begin{equation*}
   \sigma=\frac{f_0ne^2}{m(\gamma-\i \omega)}
\end{equation*}

在高频极限下,
\begin{equation*}
   \epsilon_r=1-\frac{\omega_p^2}{\omega^2},\quad \omega_p^2=\frac{nZe^2}{\epsilon_0 m}
\end{equation*}
对于等离子体,所有电子都是自由的,在忽略阻尼时,上式对任何频率都成立

在一般频率下,介质阻尼较小,相对介电常数为实数,但在$\omega \approx \omega_i$附近,相对介电常数虚部增强,产生共振吸收

\section{电磁波在色散介质中的传播}

Fourier变换
\begin{gather*}
   u(x,t)=\frac{1}{\sqrt{2\pi}} \int_{-\infty}^{+\infty} A(k)\e^{\i kx-\i \omega(k)t} \d k
   \\
   A(k)=\frac{1}{\sqrt{2\pi}} \int_{-\infty}^{+\infty} u(x,0)\e^{-\i kx} \d x
\end{gather*}

相速度
\begin{equation*}
   v_p=\frac{\omega}{k}
\end{equation*}

群速度
\begin{equation*}
   v_g=\left.\dt{\omega}{k}\right|_{k=k_0}
\end{equation*}

当$k \approx k_0$时,做Taylor展开,
\begin{equation*}
   \omega(k)\approx \omega_0+\left.\dt{\omega}{k}\right|_{k=k_0}(k-k_0)
\end{equation*}
代入得
\begin{equation*}
   u(x,t) \approx u(x-v_g t,0)\e^{\i(k_0 v_g-\omega_0)t}
\end{equation*}
此时$v_g$可以代表波包信息传播的速度和能量流动的速度

\textbf{\color{red} 注意}在某些情形下,$v_g$可能超过光速或者是负的,但这不违反相对论,因为在这种情形下不能代表能量和信息传播的速度

\subsection{因果律与Kramers-Kroning关系}

\section{波导与谐振腔}

\begin{equation*}
   \text{远程通信}
   \begin{cases}
      \text{无线通信} \\
      \text{有线通信(波导)}
      \begin{cases}
         \text{金属波导} \\
         \text{介质波导}
      \end{cases}
   \end{cases}
\end{equation*}

\subsection{Maxwell方程组的分离}

\begin{gather*}
   \bm{E}=E_z \bm{e}_3+\bm{E}_{\perp}=E_z \bm{e}_3+(\bm{e}_3 \times \bm{E}) \times \bm{e}_3
   \\
   \bm{B}=B_z \bm{e}_3+\bm{B}_{\perp}=B_z \bm{e}_3+(\bm{e}_3 \times \bm{B}) \times \bm{e}_3
\end{gather*}
Maxwell方程组
\begin{gather*}
   \pt{\bm{E}_{\perp}}{z}+\i \omega \bm{e}_3 \times \bm{B}_{\perp}=\nabla_{\perp} E_z
   \\
   \bm{e}_3 \cdot (\nabla_{\perp} \times \bm{E}_{\perp})=\i \omega B_z
   \\
   \pt{\bm{B}_{\perp}}{z}-\i \mu \epsilon \omega \bm{e}_3 \times \bm{E}_{\perp}=\nabla_{\perp} B_z
   \\
   \bm{e}_3 \cdot (\nabla_{\perp} \times \bm{B}_{\perp})=-\i \mu \epsilon \omega E_z
\end{gather*}

\subsection{金属波导}

\subsection{谐振腔}

\subsection{介质波导}

\subsubsection{平面介质波导}

\subsubsection{圆形介质波导}



\chapter{电磁波的辐射和散射}

\begin{equation*}
   \text{辐射:带电粒子变速运动}
   \begin{cases}
      \text{宏观电流周期振荡(第五章)} \\
      \text{微观带电粒子变速运动(第八章)}
   \end{cases}
\end{equation*}

\section{波动方程的推迟解}

波动方程
\begin{equation*}
   \nabla^2 \varPsi -\frac{1}{c^2} \frac{\partial^2 \varPsi}{\partial t^2} = -4 \pi f(\bm{x},t)
\end{equation*}
进行Fourier变换,得到Helmholtz方程
\begin{equation*}
   (\nabla^2+k^2)\tilde{\varPsi}(\bm{x},\omega)=-4\pi \tilde{f}(\bm{x},\omega)
\end{equation*}

求解Helmholtz方程的Green函数:
\begin{gather*}
   (\nabla^2+k^2)\tilde{G}(\bm{x},\bm{x'})=-4\pi \delta^3(\bm{x}-\bm{x'})
   \\
   \tilde{G}^{(\pm)}(\bm{R})=\frac{\e^{\pm \i kR}}{R},\quad \bm{R}=\bm{x}-\bm{x'}
\end{gather*}
波动方程的含时Green函数
\begin{gather*}
   (\nabla^2_{\bm{x}} -\frac{1}{c^2} \pt{^2}{t^2})G^{(\pm)}(\bm{x},t;\bm{x'},t')=-4\pi \delta^3(\bm{x}-\bm{x'})\delta(t-t')
   \\
   G^{(\pm)}(R,\tau)=\frac{1}{2\pi}\int_{-\infty}^{+\infty} \frac{\e^{\pm \i kR}}{R} \e^{-\i \omega \tau} \d \omega
\end{gather*}
在真空中,$k=\frac{\omega}{c}$
\begin{gather*}
   G^{(\pm)}(R,\tau)=\frac{\delta(\tau \mp \frac{R}{c})}{R}=\frac{\delta(t-(t' \pm \frac{R}{c}))}{R}
\end{gather*}
取$+$时为推迟格林函数,符合因果律,得到磁矢势
\begin{equation*}
   \bm{A}(\bm{x},t)=\frac{\mu}{4\pi} \int G^{+}(R,\tau)\bm{J}(\bm{x'},t') \d V' \d t'=\frac{\mu}{4\pi} \int \frac{\bm{J}(\bm{x'},t-\frac{R}{c})}{R} \d V'
\end{equation*}

\section{辐射场的Taylor展开}

考虑谐振源,
\begin{equation*}
   \rho(\bm{x},t)=\rho(\bm{x})\e^{-\i \omega t},\quad \bm{J}(\bm{x},t)=\bm{J}(\bm{x})\e^{-\i \omega t}
\end{equation*}
复振幅应当满足电荷守恒

辐射场
\begin{gather*}
   \bm{A}(\bm{x})=\frac{\mu_0}{4\pi} \int \bm{J}(\bm{x'}) \frac{\e^{\i kR}}{R} \d V'
   \\
   \bm{H}=\frac{1}{\mu_0} \nabla \times \bm{A}
   \\
   \bm{E}=\frac{\i}{ck \epsilon_0} \nabla \times \bm{H}
\end{gather*}

按照辐射源尺度$d$,场点与原点距离$r=|\bm{x}|$,波长$\lambda$分类,假定总有$r\gg d$:
\begin{enumerate}
   \item 近场区(静态区):$d \ll r \ll \lambda$
   \item 中间区(感应区):$d \ll r \sim \lambda$
   \item 远场区(辐射区):$d \ll \lambda \ll r$
\end{enumerate}

在$\lambda \gg d$情况下,
\begin{equation*}
   \frac{\e^{\i k|\bm{x}-\bm{x'}|}}{|\bm{x}-\bm{x'}|} = \frac{\e^{\i kr}}{r} (1+\frac{\hat{\bm{e}}_r\cdot \bm{x'}}{r}+\cdots)(1-\i k \hat{\bm{e}}_r\cdot \bm{x'}+\cdots)
\end{equation*}

\subsection{电偶极辐射}

保留最低阶
\begin{gather*}
   \bm{A}(\bm{x})=\frac{\mu_0}{4\pi} \frac{\e^{\i kr}}{r} \int \bm{J}(\bm{x'}) \d V'
\end{gather*}

根据\ref{sec:3.3}的方法,
\begin{align*}
   \int \bm{J}(\bm{x'}) \d V'&=- \int \bm{x'}(\nabla' \cdot \bm{J}) \d V'
   \\
   &=-\i \omega \int \bm{x'} \rho(\bm{x'}) \d V'
   \\
   &=-\i \omega \bm{p}
\end{align*}
得到
\begin{gather*}
   \bm{A}(\bm{x})=-\frac{\i \mu_0 \omega}{4\pi} \frac{\e^{\i kr}}{r} \bm{p}
   \\
   \bm{H}=\frac{ck^2}{4\pi} \frac{\e^{\i kr}}{r} (1-\frac{1}{\i kr}) \hat{\bm{e}}_r \times \bm{p} 
   \\
   \bm{E}=\frac{1}{4\pi \epsilon_0} \{k^2\frac{\e^{\i kr}}{r} (\hat{\bm{e}}_r \times \bm{p}) \times \hat{\bm{e}}_r+(\frac{1}{r^3}-\frac{\i k}{r^2})\e^{\i kr}(3(\hat{\bm{e}}_r \cdot \bm{p})\hat{\bm{e}}_r-\bm{p})\}
\end{gather*}
取长波近似,
\begin{gather*}
   \bm{H}=\frac{ck^2}{4\pi} \frac{\e^{\i kr}}{r} \hat{\bm{e}}_r \times \bm{p}
   \\
   \bm{E}=\frac{k^2}{4\pi \epsilon_0} \frac{\e^{\i kr}}{r} (\hat{\bm{e}}_r \times \bm{p}) \times \hat{\bm{e}}_r
\end{gather*}
辐射功率
\begin{gather*}
   \dt{P}{\Omega}=\frac{1}{2} \Re[r^2\hat{\bm{e}}_r \cdot (\bm{E} \times \bm{H^{*}})]=\frac{ck^4}{32 \pi^2 \epsilon_0} p^2 \sin^2{\theta}
   \\
   P=\frac{ck^4}{12 \pi \epsilon_0} p^2
\end{gather*}
$\theta$为$\bm{p}$与$\hat{\bm{e}}_r$夹角

\subsection{磁偶极辐射}

保留到一阶,
\begin{align*}
   \bm{A}(\bm{x})&=\frac{\mu_0}{4\pi} \frac{\e^{\i kr}}{r} (\frac{1}{r}-\i k) \int (\hat{\bm{e}}_r \cdot \bm{x'}) \bm{J}(\bm{x'})\d V'
   \\
   &=\frac{\mu_0}{4\pi} \frac{\e^{\i kr}}{r} (\frac{1}{r}-\i k) \int \hat{\bm{e}}_r \cdot (\frac{\bm{x'}\bm{J}-\bm{J}\bm{x'}}{2})\d V'+\frac{\mu_0}{4\pi} \frac{\e^{\i kr}}{r} (\frac{1}{r}-\i k) \int \hat{\bm{e}}_r \cdot (\frac{\bm{x'}\bm{J}+\bm{J}\bm{x'}}{2})\d V'
\end{align*}
考虑反对称部分
\begin{align*}
   \bm{A}(\bm{x})&=\frac{\mu_0}{4\pi} \frac{\e^{\i kr}}{r} (\frac{1}{r}-\i k) \int \hat{\bm{e}}_r \cdot (\frac{\bm{x'}\bm{J}-\bm{J}\bm{x'}}{2})\d V'
   \\
   &=\frac{\i k \mu_0}{4\pi} \frac{\e^{\i kr}}{r} (1-\frac{1}{\i kr}) \hat{\bm{e}}_r \times \bm{m}
\end{align*}
$\bm{m}$为磁矩,$\bm{A}$类似电偶极辐射中的$\bm{H}$,则磁场类似电偶极辐射中的电场
\begin{gather*}
   \bm{H}=\frac{1}{4\pi} \{k^2\frac{\e^{\i kr}}{r} (\hat{\bm{e}}_r \times \bm{m}) \times \hat{\bm{e}}_r+(\frac{1}{r^3}-\frac{\i k}{r^2})\e^{\i kr}(3(\hat{\bm{e}}_r \cdot \bm{m})\hat{\bm{e}}_r-\bm{m})\}
   \\
   \bm{E}=-\pt{\bm{A}}{t}=-\frac{\mu_0 ck^2}{4\pi} \frac{\e^{\i kr}}{r} (1-\frac{1}{\i kr}) \hat{\bm{e}}_r \times \bm{m}
\end{gather*}
辐射功率
\begin{gather*}
   \dt{P}{\Omega}=\frac{1}{2} \Re[r^2\hat{\bm{e}}_r \cdot (\bm{E} \times \bm{H^{*}})]=\frac{\mu_0 ck^4}{32 \pi^2} m^2 \sin^2{\theta}
   \\
   P=\frac{\mu_0 ck^4}{12 \pi} m^2
\end{gather*}
$\theta$为$\bm{m}$与$\hat{\bm{e}}_r$夹角,要求磁偶极矩分量之间没有相位差

形式上电偶极辐射到磁偶极辐射的变换规则:
\begin{equation*}
   \begin{cases}
      \bm{p} \to \frac{\bm{m}}{c} \\
      \bm{E} \to \frac{\mu_0}{c} \bm{H} \\
      \frac{\mu_0}{c} \bm{H} \to -\bm{E}
   \end{cases}
\end{equation*}

\subsection{电四极辐射}

考虑对称项,利用电荷守恒$\nabla \cdot \bm{J}=\i \omega \rho$
\begin{align*}
   \int & \partial'_k(J_k(\bm{x'})x'_i x'_j) \d V'=0
   \\
   \int & (x'_i J_j(\bm{x'})+x'_j J_i(\bm{x'})) \d V'+ \i \omega \int \rho(\bm{x'}) x'_i x'_j \d V'=0
   \\
   \bm{A}(\bm{x})&=\frac{\mu_0}{4\pi} \frac{\e^{\i kr}}{r} (\frac{1}{r}-\i k) \int \hat{\bm{e}}_r \cdot (\frac{\bm{x'}\bm{J}+\bm{J}\bm{x'}}{2})\d V'
   \\
   &=-\frac{\mu_0ck^2}{8\pi} \frac{\e^{\i kr}}{r} (1-\frac{1}{\i kr}) \int \rho(\bm{x'}) (\hat{\bm{e}}_r \cdot \bm{x'}) \bm{x'} \d V'
\end{align*}
仅考虑远场区,
\begin{align*}
   \bm{H}&=\frac{\i k}{\mu_0} \hat{\bm{e}}_r \times \bm{A}
   \\
   &=-\frac{\i ck^3}{24\pi} \frac{\e^{\i kr}}{r} \hat{\bm{e}}_r \times (\bm{D} \cdot \hat{\bm{e}}_r)
   \\
   \bm{E}&=\frac{\i}{ck \epsilon_0} \i k \hat{\bm{e}}_r \times \bm{H}
   \\
   &=\frac{\i k^3}{24\pi \epsilon_0} \frac{\e^{\i kr}}{r} \hat{\bm{e}}_r \times (\hat{\bm{e}}_r \times (\bm{D} \cdot \hat{\bm{e}}_r))
\end{align*}
辐射功率
\begin{align*}
   \dt{P}{\Omega}&=\frac{ck^6}{1152\pi^2 \epsilon_0} \hat{\bm{e}}_r \cdot [(\hat{\bm{e}}_r \times (\bm{D} \cdot \hat{\bm{e}}_r)) \times (\hat{\bm{e}}_r \times (\hat{\bm{e}}_r \times (\bm{D} \cdot \hat{\bm{e}}_r)))]
   \\
   &=\frac{ck^6}{1152\pi^2 \epsilon_0} |\hat{\bm{e}}_r \times (\hat{\bm{e}}_r \times (\bm{D} \cdot \hat{\bm{e}}_r))|^2
   \\
   &=\frac{ck^6}{1152\pi^2 \epsilon_0} |\hat{\bm{e}}_r \times (\bm{D} \cdot \hat{\bm{e}}_r)|^2
\end{align*}
积分
\begin{align*}
   \epsilon^{ijk}\epsilon_{ilm}&={\delta^j}_l {\delta^{k}}_m-{\delta^j}_m {\delta^{k}}_l
   \\
   (\bm{A} \times \bm{B}) \cdot (\bm{C} \times \bm{D})&=(\bm{A} \cdot \bm{C})(\bm{B} \cdot \bm{D})-(\bm{A} \cdot \bm{D})(\bm{B} \cdot \bm{C})
   \\
   (\hat{\bm{e}}_r \times (\bm{D}^{*} \cdot \hat{\bm{e}}_r))\cdot(\hat{\bm{e}}_r \times (\bm{D} \cdot \hat{\bm{e}}_r))&=(\hat{\bm{e}}_r \cdot \bm{D}^{*}) \cdot (\hat{\bm{e}}_r \cdot \bm{D})-(\hat{\bm{e}}_r \cdot \bm{D}^{*} \cdot \hat{\bm{e}}_r)(\hat{\bm{e}}_r \cdot \bm{D} \cdot \hat{\bm{e}}_r)
\end{align*}
由对称性,
\begin{align*}
   \int (\hat{\bm{e}}_r \cdot \bm{D}^{*}) \cdot (\hat{\bm{e}}_r \cdot \bm{D}) r^2 \d \Omega&={D^*_i}^k\delta_{kl}{D_j}^l \int n^i n^j \d \Omega
   \\
   &={D^*_i}^k\delta_{kl}{D_j}^l K_1 \delta^{ij}
   \\
   &=K_1 \Tr(\bm{D}^{\dagger} \bm{D})
   \\
   \int (\hat{\bm{e}}_r \cdot \bm{D}^{*} \cdot \hat{\bm{e}}_r)(\hat{\bm{e}}_r \cdot \bm{D} \cdot \hat{\bm{e}}_r)r^2 \d \Omega &=D_{ij}^{*}D_{kl} \int r^i r^j r^k r^l \d \Omega
   \\
   &=D_{ij}^{*}D_{kl} K_2 (\delta^{ij}\delta^{kl}+\delta^{ik}\delta^{jl}+\delta^{il}\delta^{jk})
   \\
   &=K_2(\Tr(\bm{D}^{\dagger})\Tr(\bm{D})+2\Tr(\bm{D}^{\dagger} \bm{D}))
   \\
   \delta_{ij}\int n^i n^j \d \Omega &= \int n^i n^i \d \Omega =4\pi
   \\
   \delta_{ij} \delta_{kl} \int n^i n^j n^k n^l \d \Omega &= \int (n^i)^2 (n^k)^2 \d \Omega = 4\pi
   \\
   K_1 d = 4\pi,&\quad K_2 (d^2+2d)=4\pi
   \\
   K_1=\frac{4}{3} \pi,&\quad K_2=\frac{4}{15} \pi
   \\
   \Tr(\bm{D}^{\dagger})&=\Tr(\bm{D})=0
\end{align*}
则有
\begin{gather*}
   P=\frac{ck^6}{1440\pi \epsilon_0} \Tr(\bm{D}^{\dagger} \bm{D})
\end{gather*}

\section{辐射场的多极展开}

球面波加法定理:
\begin{equation*}
   \frac{\e^{\i k|\bm{x}-\bm{x'}|}}{|\bm{x}-\bm{x'}|}=4\pi \i k \displaystyle \sum_{l=0}^{\infty} \sum_{m=-l}^{l} j_l(kr_{<})h_l^{(1)}(kr_{>}) Y_{lm}^*(\bm{n'})Y_{lm}(\bm{n})
\end{equation*}

角动量算符
\begin{gather*}
   \hat{\bm{L}}=-\i \bm{x} \times \nabla
   \\
   \hat{\bm{L}}^2=-\frac{1}{\sin \theta} \pt{}{\theta} (\sin \theta\pt{}{\theta})-\frac{1}{\sin^2 \theta}\pt{^2}{\phi^2}
   \\
   \hat{L}_{\pm}=\hat{L}_x \pm \hat{L}_y
   \\
   \intertext{本征方程}
   \hat{\bm{L}}^2 Y_{lm}=l(l+1)Y_{lm}
   \\
   \hat{L}_z Y_{lm}=mY_{lm}
\end{gather*}
代入计算
\begin{align*}
   \hat{L}_+ Y_{lm}&=\sqrt{l(l+1)-m(m+1)}Y_{l,m+1}
   \\
   \hat{L}_- Y_{lm}&=\sqrt{l(l+1)-m(m-1)}Y_{l,m-1}
   \\
   |\hat{\bm{L}}Y_{lm}|^2&=|\hat{L}_x Y_{lm}|^2+|\hat{L}_y Y_{lm}|^2+|\hat{L}_z Y_{lm}|^2
   \\
   &=|\frac{\hat{L}_+ + \hat{L}_-}{2} Y_{lm}|^2+|\frac{\hat{L}_+ - \hat{L}_-}{2\i} Y_{lm}|^2+|\hat{L}_z Y_{lm}|^2
   \\
   &=\frac{l(l+1)-m(m+1)}{2}|Y_{l,m+1}|^2+\frac{l(l+1)-m(m-1)}{2}|Y_{l,m-1}|^2+m^2|Y_{lm}|^2
\end{align*}

\section{电磁波的散射}

\chapter{狭义相对论}

\section{狭义相对论的基本假设与实验验证}

基本假设:
\begin{enumerate}
   \item 相对性原理:Galileo$\to$Einstein
   \item 光速不变原理:信号传递的最大速度是光速
\end{enumerate}
加上时空均匀、各向同性可以得到狭义相对论和相对论性电动力学

实验验证:
\begin{enumerate}
   \item Michelson-Morley实验:光速不变
   \item 动源光速测量:光学灭绝问题(Ewald-Oseen灭绝定理):当光进入探测器的介质时会使介质极化,完全抵消真空中的入射场并发出电磁波 \\
   Alvager实验:$\pi^0$介子衰变 \\
   自由电子激光装置FLASH \\
   遥远星体辐射
\end{enumerate}

\section{洛伦兹变换}

时空均匀性$\implies$坐标变换为线性变换
各向同性
\begin{gather*}
   \Delta s^2=A(|\bm{v'}|) \Delta s'^2
   \\
   \Delta s'^2=A(|-\bm{v'}|) \Delta s^2=A(|\bm{v'}|) \Delta s^2=A(|\bm{v'}|)^2 \Delta s^2
\end{gather*}
由变换的连续性,$A$=1,
\begin{equation*}
   \Delta s^2=\Delta s'^2
\end{equation*}

Minkowski时空:四维间隔不变

\begin{itemize}
   \item 类时:$\Delta s^2>0$
   \item 类空:$\Delta s^2<0$
   \item 类光:$\Delta s^2=0$
\end{itemize}

Lorentz变换:\\
考虑$K'$系沿K系$x$轴以匀速$v$运动,$t=0$时两系坐标原点重合
\begin{gather*}
   ct'=x \sinh \psi + ct \cosh \psi \\
   x'=x \cosh \psi + ct \sinh \psi
\end{gather*}
其中$\tanh \psi=-\frac{v}{c}$
得到Lorentz变换
\begin{gather*}
   ct'=\gamma(ct-\beta x) \\
   x'=\gamma(x-\beta ct) \\
   y'=y \\
   z'=z
\end{gather*}
速度方向一般情况下:
\begin{gather*}
   ct'=\gamma(ct-\bm{\beta} \cdot \bm{x}) \\
   \bm{x'}=\bm{x}+(\gamma-1)\frac{\bm{\beta} \cdot \bm{x}}{\bm{\beta}^2}\bm{\beta}-\gamma \bm{\beta} ct
\end{gather*}
同时相对性,因果性

\section{四维张量}

标量:Lorentz变换下不变的量,间隔平方$\Delta s^2$,四维体积元$\d^4 x$

四维矢量:
\begin{enumerate}
   \item 逆变矢量:$A'^{\mu}={\Lambda^{\mu}}_{\nu} A^{\nu}$
   \begin{gather*}
      x^{\mu}=(ct,\bm{x}) \\
      k^{\mu}=(\frac{\omega}{c}, \bm{k}) \\
      \partial^{\mu}=\pt{}{x_{\mu}}=(\frac{1}{c}\pt{}{t},\nabla)
   \end{gather*}
   \item 协变矢量:$A_{\mu}=\eta_{\mu \nu} A^{\nu}$
   \begin{gather*}
      x_{\mu}=(ct,-\bm{x})
   \end{gather*}
\end{enumerate}

度规$\eta_{\mu \nu}$和度规的逆$\eta^{\mu \nu}$,
\begin{gather*}
   \eta_{\mu \beta} \eta^{\beta \nu}={\delta_{\mu}}^{\nu}
\end{gather*}

Minkowski时空的度规:
\begin{gather*}
   \eta_{\mu \nu}=\eta^{\mu \nu}=
   \begin{pmatrix}
      1 & 0 & 0 & 0 \\
      0 & -1 & 0 & 0 \\
      0 & 0 & -1 & 0 \\
      0 & 0 & 0 & -1
   \end{pmatrix}
\end{gather*}

指标缩并

内积:$A^{\mu}B_{\mu}=A_{\mu}B^{\mu}$任意两个矢量的内积是标量
\begin{align*}
   \d s^2&=\d x^{\mu} \d x_{\mu}=\eta_{\mu \nu} \d x^{\mu} \d x^{\nu}=\eta^{\mu \nu} \d x_{\mu} \d x_{\nu}
   \\
   \phi&=k^{\mu}x_{\mu}
   \\
   \text{d'Alembert算符:}\Box&=-\partial^{\mu} \partial_{\mu}=-\frac{1}{c^2} \pt{^2}{t^2}+\nabla^2
\end{align*}

相对性原理给出正确物理方程的必要条件:以正确的张量形式写出,称为协变性

\section{Lorentz变换的数学性质}

内积不变给出
\begin{gather*}
   \eta_{\mu \nu} {\Lambda^{\mu}}_{\alpha} {\Lambda^{\nu}}_{\beta}=\eta_{\alpha \beta}
   \\
   \implies \mathrm{det} \Lambda=\pm 1
\end{gather*}

Lorentz变换构成Lorentz群,$O(1,3)$,是无限群,是Lie群,每个元素可以用六个实参数表示,
\begin{equation*}
   \Lambda=\e^{\i \theta_i S_i-\omega_i K_i}
\end{equation*}
这里三维矢量$\bm{\theta},\bm{\omega}$和三阶矩阵$\bm{S},\bm{K}$不遵循上下标约定,仍有重复求和\\
其中$\theta_i$为旋转角,$\omega_i$为boost角,$S_i$为旋转生成元,$K_i$为boost生成元
\begin{gather*}
   \i S_{1,2,3}=
   \begin{pmatrix}
      0 & 0 & 0 & 0 \\
      0 & 0 & 0 & 0 \\
      0 & 0 & 0 & -1 \\
      0 & 0 & +1 & 0
   \end{pmatrix},
   \begin{pmatrix}
      0 & 0 & 0 & 0 \\
      0 & 0 & 0 & +1 \\
      0 & 0 & 0 & 0 \\
      0 & -1 & 0 & 0
   \end{pmatrix},   
   \begin{pmatrix}
      0 & 0 & 0 & 0 \\
      0 & 0 & -1 & 0 \\
      0 & +1 & 0 & 0 \\
      0 & 0 & 0 & 0
   \end{pmatrix}
   \\
   K_{1,2,3}=
   \begin{pmatrix}
      0 & 1 & 0 & 0 \\
      1 & 0 & 0 & 0 \\
      0 & 0 & 0 & 0 \\
      0 & 0 & 0 & 0
   \end{pmatrix},
   \begin{pmatrix}
      0 & 0 & 1 & 0 \\
      0 & 0 & 0 & 0 \\
      1 & 0 & 0 & 0 \\
      0 & 0 & 0 & 0
   \end{pmatrix},
   \begin{pmatrix}
      0 & 0 & 0 & 1 \\
      0 & 0 & 0 & 0 \\
      0 & 0 & 0 & 0 \\
      1 & 0 & 0 & 0
   \end{pmatrix}
\end{gather*}
对易关系
\begin{gather*}
   [S_i,S_j]=\i \epsilon_{ijk} S_k, 
   \\
   [K_i,K_j]=\i \epsilon_{ijk} S_k, 
   \\
   [S_i,K_j]=\i \epsilon_{ijk} K_k
\end{gather*}

仅涉及boost的Lorentz变换:
\begin{equation*}
   \Lambda(\bm{\beta})=
   \begin{pmatrix}
      \gamma & -\gamma \beta_1 & -\gamma \beta_2 & -\gamma \beta_3 \\
      -\gamma \beta_1 & 1+(\gamma-1)\frac{\beta_1^2}{\beta^2} & (\gamma-1)\frac{\beta_1 \beta_2}{\beta^2} & (\gamma-1)\frac{\beta_1 \beta_3}{\beta^2} \\
      -\gamma \beta_2 & (\gamma-1)\frac{\beta_1 \beta_2}{\beta^2} & 1+(\gamma-1)\frac{\beta_2^2}{\beta^2} & (\gamma-1)\frac{\beta_2 \beta_3}{\beta^2} \\
      -\gamma \beta_3 & (\gamma-1)\frac{\beta_1 \beta_3}{\beta^2} & (\gamma-1)\frac{\beta_3 \beta_1}{\beta^2} & 1+(\gamma-1)\frac{\beta_3^2}{\beta^2}
   \end{pmatrix}
\end{equation*}

\chapter{相对论性电动力学}

{\color{red} 本章采用高斯单位制}

\section{自由粒子}

\begin{gather*}
   S=-mc \int \d s=-mc^2 \int \d \tau \\
   L=-mc^2 \sqrt{1-\frac{v^2}{c^2}}
\end{gather*}
$m$是一个非负Lorentz标量,称为静止质量

重参数化不变性:
\begin{gather*}
   x^{\mu}=x^{\mu}(\tau)=x^{\mu}(\tau(\tilde{tau})) \\
   S=-mc \int \d \tau \sqrt{\dt{x^{\mu}}{\tau} \dt{x_{\mu}}{\tau}}=-mc \int \d \tilde{\tau} \sqrt{\dt{x^{\mu}}{\tilde{\tau}} \dt{x_{\mu}}{\tilde{\tau}}}
\end{gather*}
$\tau$可以是固有时,也可以是固有时的任意单调递增函数

正则动量
\begin{gather*}
   \bm{p}=\pt{L}{\bm{v}}=\frac{m \bm{v}}{\sqrt{1-\frac{v^2}{c^2}}}
\end{gather*}
Hamiltonian
\begin{gather*}
   H=\bm{p} \cdot \bm{v}-L=\frac{mc^2}{\sqrt{1-\frac{v^2}{c^2}}}
\end{gather*}

定义协变四维速度
\begin{gather*}
   u^{\mu}=\frac{\d x^{\mu}}{\d s}=\gamma(1,\frac{\bm{v}}{c})
\end{gather*}
根据最小作用量原理得到运动方程
\begin{gather*}
   \dt{u_{\mu}}{s}=0
\end{gather*}

四维动量
\begin{gather*}
   p^{\mu}=mcu^{\mu}=(\frac{E}{c},\bm{p})
\end{gather*}

零质量粒子:\\
引入世界线的单元基:正定标量函数$e(\tau)$,作用量
\begin{gather*}
   S=-\frac{1}{2} \int (\frac{1}{e(\tau)} \dt{x_{\mu}}{\tau} \dt{x^{\mu}}{\tau} + e(\tau)m^2c^2) \d \tau
   \\
   \intertext{对$e(\tau)$变分}
   \dt{x_{\mu}}{\tau} \dt{x^{\mu}}{\tau}-e(\tau)^2 m^2c^2 =0
   \\
   e(\tau)=\frac{1}{mc} \sqrt{\dt{x_{\mu}}{\tau} \dt{x^{\mu}}{\tau}}
\end{gather*}
代入后得到原来的结果. 对于零质量粒子也适用\\
重参数化不变性
\begin{equation*}
   \tilde{e}(\tilde{\tau}) \d \tilde{\tau}=e(\tau) \d \tau
\end{equation*}

\section{电磁场中的粒子}

电荷$e$也是Lorentz标量

作用量
\begin{gather*}
   S=-mc \int \d s-\frac{e}{c} \int A_{\mu} \d x^{\mu}
\end{gather*}
四维矢量势
\begin{gather*}
   A^{\mu}(x)=(\Phi(x),\bm{A}(x))
\end{gather*}
Lagrangian
\begin{gather*}
   L=-mc^2 \sqrt{1-\frac{v^2}{c^2}}+\frac{e}{c} \bm{A} \cdot \bm{v}-e \Phi
\end{gather*}
正则动量
\begin{gather*}
   \bm{P}=\pt{L}{\bm{v}}=\frac{m \bm{v}}{\sqrt{1-\frac{v^2}{c^2}}}+\frac{e}{c} \bm{A}
\end{gather*}
Hamiltonian
\begin{gather*}
   H=\bm{P} \cdot \bm{v}-L=\sqrt{m^2c^4+c^2(\bm{P}-\frac{e}{c}\bm{A})^2}+e \Phi
\end{gather*}

最小作用量原理
\begin{align*}
   \delta S&=mc \int \dt{u_{\mu}}{s} \delta x^{\mu} \d s-\frac{e}{c} \int (\delta A_{\mu} \d x^{\mu} + A_{\mu} \delta \d x^{\mu})
   \\
   &=mc \int \dt{u_{\mu}}{s} \delta x^{\mu} \d s-\frac{e}{c} \int (\partial_{\nu} A_{\mu} \delta x^{\nu} \d x^{\mu} -\partial_{\nu} A_{\mu} \d x^{\nu} \delta x^{\mu})
   \\
   &=mc \int \dt{u_{\mu}}{s} \delta x^{\mu} \d s-\frac{e}{c} \int (\partial_{\mu} A_{\nu} \delta x^{\mu} \d x^{\nu} -\partial_{\nu} A_{\mu} \d x^{\nu} \delta x^{\mu})
\end{align*}
得到运动方程
\begin{align*}
   mc \dt{u_{\mu}}{s}&=\frac{e}{c} F_{\mu\nu} u^{\nu}
\end{align*}
电磁场场强张量
\begin{align*}
   F_{\mu\nu}&=\partial_{\mu} A_{\nu}-\partial_{\nu} A_{\mu}
   \\
   &=
   \begin{bmatrix}
      0 & E_1 & E_2 & E_3 \\
      -E_1 & 0 & -B_3 & B_2 \\
      -E_2 & B_3 & 0 & -B_1 \\
      -E_3 & -B_2 & B_1 & 0
   \end{bmatrix}
\end{align*}

\section{电磁场}

\begin{gather*}
   F'^{\mu\nu}={\Lambda^{\mu}}_{\alpha} {\Lambda^{\nu}}_{\beta} F^{\alpha\beta}
\end{gather*}
在仅有推促的Lorentz变换下
\begin{gather*}
   \bm{E'}=\gamma(\bm{E}+\bm{\beta} \times \bm{B})-\frac{\gamma^2}{1+\gamma} (\bm{\beta} \cdot \bm{E}) \bm{\beta} \\
   \bm{B'}=\gamma(\bm{B}-\bm{\beta} \times \bm{E})-\frac{\gamma^2}{1+\gamma} (\bm{\beta} \cdot \bm{B}) \bm{\beta}
\end{gather*}

Bianchi恒等式(对应无源齐次Maxwell方程)
\begin{gather*}
   \partial_{\mu} F_{\nu\alpha}+\partial_{\nu} F_{\alpha\mu}+\partial_{\alpha} F_{\mu\nu}=0
\end{gather*}
或者定义对偶张量
\begin{align*}
   \tilde{F}^{\mu\nu}&=\frac{1}{2} \epsilon^{\mu\nu\alpha\beta} F_{\alpha\beta}
   \\
   &=
   \begin{bmatrix}
      0 & -B_1 & -B_2 & -B_3 \\
      B_1 & 0 & -E_3 & E_2 \\
      B_2 & E_3 & 0 & -E_1 \\
      B_3 & -E_2 & E_1 & 0
   \end{bmatrix}
   \\
   \intertext{Bianchi恒等式}
   \partial_{\mu} \tilde{F}^{\mu\nu}&=0
\end{align*}

作用量
\begin{gather*}
   S=-\frac{1}{16 \pi c} \int F_{\mu\nu} F^{\mu\nu} \d^4 x
\end{gather*}
带电粒子作用量电磁场项
\begin{gather*}
   S_{int}=-\frac{e}{c} \int A_{\mu} \d x^{\mu}=-\frac{1}{c^2} \int A_{\mu} J^{\mu} \d^4 x
\end{gather*}
合在一起
\begin{gather*}
   S=-\frac{1}{16 \pi c} \int F_{\mu\nu} F^{\mu\nu} \d^4 x-\frac{1}{c^2} \int A_{\mu} J^{\mu} \d^4 x
\end{gather*}
最小作用量原理$\frac{\delta S}{\delta A_{\mu}}=0$得到有源Maxwell方程
\begin{gather*}
   \partial_{\mu} F^{\mu\nu}=\frac{4\pi}{c} J^{\nu}
\end{gather*}

电流密度四维矢量
\begin{gather*}
   J^{\mu}=\rho \dt{x^{\mu}}{t}=(c \rho,\bm{J})
\end{gather*}
电荷守恒定律
\begin{gather*}
   \partial_{\mu} J^{\mu}=0
\end{gather*}

四维协变电磁规律:
\begin{gather*}
   \begin{cases}
      \partial_{\mu} F^{\mu\nu}=\frac{4\pi}{c} J^{\nu}
      \\
      \partial_{\mu} F_{\nu\alpha}+\partial_{\nu} F_{\alpha\mu}+\partial_{\alpha} F_{\mu\nu}=0
      \\
      \partial_{\mu} J^{\mu}=0
   \end{cases}
\end{gather*}
   
\section{宏观运动物体中的电磁场}

考虑相对论效应的一阶修正

\subsection{运动的电介质}

$(\bm{D},\bm{H})$构成场张量$H_{\mu\nu}$

电介质中Maxwell方程组
\begin{equation*}
   \partial_{\mu} H^{\mu \nu}=0
\end{equation*}

本构关系:Minkowski方程
\begin{gather*}
   H^{\mu \nu}u_{\nu}=\epsilon F_{\mu \nu} u^{\nu}
   \\
   F_{\mu \nu}u_{\lambda}+F_{\nu \lambda}u_{\mu}+F_{\lambda \mu}u_{\nu}=\mu(H_{\mu \nu}u_{\lambda}+H_{\nu \lambda}u_{\mu}+H_{\lambda \mu}u_{\nu})
\end{gather*}
三维形式
\begin{gather*}
   \bm{D}+\bm{\beta} \times \bm{H}=\epsilon (\bm{E}+\bm{\beta} \times \bm{B})
   \\
   \bm{B}-\bm{\beta} \times \bm{E}=\mu (\bm{H}-\bm{\beta} \times \bm{D})
\end{gather*}
一阶近似
\begin{gather*}
   \bm{D}=\epsilon \bm{E}+(\epsilon \mu -1) \bm{\beta} \times \bm{H}
   \\
   \bm{B}=\mu \bm{H}-(\epsilon \mu -1) \bm{\beta} \times \bm{E}
\end{gather*}

边界条件
\begin{gather*}
   \begin{cases}
      \bm{n} \cdot (\bm{D}_2-\bm{D}_1)=0
      \\
      \bm{n} \cdot (\bm{B}_2-\bm{B}_1)=0
      \\ 
      \bm{n} \times (\bm{E}_2-\bm{E}_1)=\beta_n (\mu_2-\mu_1) \bm{H}_t
      \\
      \bm{n} \times (\bm{H}_2-\bm{H}_1)=-\beta_n (\epsilon_2-\epsilon_1) \bm{E}_t
   \end{cases}
\end{gather*}
$\bm{E}_t$和$\bm{H}_t$不需要区分是何种介质,因为两种介质切向分量差值为小量

\subsection{运动的导体}

\begin{gather*}
   \bm{E'}=\bm{E}+\bm{\beta} \times \bm{B}
   \\
   \bm{J}=\sigma \bm{E'}
\end{gather*}
代入到电磁感应定律和环路定律,
\begin{gather*}
   \pt{\bm{H}}{t}-\nabla \times (\bm{v} \times \bm{H})=\frac{c^2}{4 \pi \sigma \mu} \nabla^2 \bm{H}
\end{gather*}

\section{均匀静电磁场中带电粒子的运动}

\subsection{在均匀静电场中的运动}

\begin{gather*}
   \bm{p}(t)=\bm{p}(0)+e\bm{E_0}t
\end{gather*}

\subsection{在均匀静磁场中的运动}

设$\bm{B}=B \bm{e}_3$
\begin{gather*}
   \intertext{回旋频率}
   \omega_B=\frac{e\bm{B}}{\gamma mc}
   \\
   \intertext{回旋半径}
   a=\frac{cp_{\perp}}{eB}
   \\
   \bm{v}(t)=\bm{v}_{\parallel}+\omega_B a (\cos \omega_B t \bm{e}_1-\sin \omega_B t \bm{e}_2)
\end{gather*}

\subsection{在正交静电磁场中的运动}

换到参照系$K'$,回归到只有静电场或静磁场的情况. 设$K'$速度为$\bm{u}$

\begin{enumerate}
   \item $|\bm{B}|>|\bm{E}|$:$\bm{u}=c \frac{\bm{E}\times \bm{B}}{|\bm{B}|^2}$ 
   \[\bm{E'}=0,\bm{B'}=\sqrt{\frac{B^2-E^2}{E^2}}\bm{B}\]
   \item $|\bm{B}|<|\bm{E}|$:$\bm{u}=c \frac{\bm{E}\times \bm{B}}{|\bm{E}|^2}$ 
   \[\bm{E'}=\sqrt{\frac{E^2-B^2}{E^2}}\bm{E},\bm{B'}=0\]
\end{enumerate}

\subsection{在一般电磁场中的运动}

\begin{gather*}
   \dt{u^{\mu}}{\tau}=\frac{e}{mc} {F^{\mu}}_{\nu} u^{\nu}
   \\
   \intertext{利用Lorentz变换生成元,}
   {F^{\mu}}_{\nu} \to \bm{E} \cdot \bm{K} - \bm{B} \cdot \bm{S}
   \\
   u(\tau)=u(0) \e^{\frac{e\tau}{mc}[\bm{E} \cdot \bm{K} - \bm{B} \cdot \bm{S}]}
\end{gather*}

\chapter{运动带电粒子的辐射}

\section{Liénard-Wiechert势}

直接在四维协变形式下求解\\
规范条件$\partial_{\mu} A^{\mu}=0$
\begin{gather*}
   \partial_{\mu} F^{\mu\nu}=\frac{4\pi}{c} J^{\nu}
   \\
   \partial_{\mu} \partial^{\mu}=\frac{4\pi}{c} J^{\nu}
\end{gather*}
考虑d'Alembert算符的四维协变形式Green函数$D(x,x')$
\begin{gather*}
   -\partial_{\mu} \partial^{\mu} D(x,x')=\delta^4(x-x')
   \\
   \intertext{Fourier变换解得}
   D(x,x')=\int \frac{1}{(2\pi)^4} \frac{\e^{-\i k \cdot (x-x')}}{k^2} \d^4 k
   \\
   \intertext{$k^2$在实轴上有奇点,选取推迟Green函数}
   D^{(+)}(x,x')=\int \frac{1}{(2\pi)^4} \frac{\e^{-\i k \cdot (x-x')}}{(k^0+\i \epsilon)^2-\bm{k}^2} \d^4 k
\end{gather*}
由复变函数Jordan引理,
\begin{enumerate}
   \item 若$x_0<x'_0$,积分围道在上半平面,结果为0
   \item 若$x_0>x'_0$,积分围道在下半平面,结果为$k_0=\pm|\bm{k}|$处留数之和
\end{enumerate}
\begin{align*}
   D^{(+)}(x,x')&=\int \frac{1}{(2\pi)^3} \e^{\i \bm{k} \cdot \bm{R}} \d^3 \bm{k} \int \frac{1}{2\pi} \frac{\e^{-\i k^0 R_0}}{(k^0+\i \epsilon)^2-\bm{k}^2} \d k^0
   \\
   &=\int \frac{1}{(2\pi)^3} \e^{\i \bm{k} \cdot \bm{R}} \d^3 \bm{k} \i (\frac{\e^{-\i |\bm{k}|R_0}}{2|\bm{k}|}-\frac{\e^{\i |\bm{k}|R_0}}{2|\bm{k}|})
   \\
   &=-\i \int \frac{1}{16\pi^3} \e^{\i \bm{k} \cdot \bm{R}} \d^3 \bm{k} \frac{\e^{\i |\bm{k}|R_0}-\e^{-\i |\bm{k}|R_0}}{|\bm{k}|}
   \\
   &=-\frac{\i }{8\pi^2} \int_{0}^{+\infty} \int_{0}^{\pi} \frac{\e^{\i |\bm{k}|R_0}-\e^{-\i |\bm{k}|R_0}}{|\bm{k}|}\e^{\i |\bm{k}| |\bm{R}| \cos \theta} |\bm{k}|^2 \sin \theta \d |\bm{k}| \d \theta
   \\
   &=-\frac{\i}{8\pi^2} \int_{0}^{+\infty} |\bm{k}| (\e^{\i |\bm{k}|R_0}-\e^{-\i |\bm{k}|R_0}) \frac{\e^{\i |\bm{k}||\bm{R}|}-\e^{-\i |\bm{k}||\bm{R}|}}{\i |\bm{k}||\bm{R}|}\d |\bm{k}|
   \\
   &=-\frac{1}{4\pi |\bm{R}|} \frac{1}{2\pi} \int_{-\infty}^{+\infty} (\e^{\i |\bm{k}|(R_0+|\bm{R}|)}-\e^{-\i |\bm{k}|(R_0-|\bm{R}|)}) \d |\bm{k}|
   \\
   &=-\frac{1}{4\pi |\bm{R}|} (
      \delta(R_0+|\bm{R}|)-\delta(R_0-|\bm{R}|))
   \\
   &=\frac{1}{4\pi |\bm{R}|}  \delta(x_0-x'_0-|\bm{R}|)
\end{align*}
结合上述讨论,推迟Green函数
\begin{gather*}
   D^{(+)}(x,x')=\frac{\theta(x_0-x_0')}{4\pi |\bm{R}|}  \delta(x_0-x'_0-|\bm{R}|)=\frac{\theta(x_0-x_0')}{2\pi |\bm{R}|}  \delta((x-x')^2)
\end{gather*}
推迟势
\begin{gather}
   A^{\mu}(x)=\frac{4\pi}{c} \int D^{(+)}(x,x') J^{\mu}(x') \d^4 x'
\end{gather}

真空中运动的粒子产生的势$A^{\mu}(x)$,世界线$r^{\mu}(\tau)$
\begin{align*}
   J^{\mu}(x')&=ec^2 \int u^{\mu}(\tau) \delta^4(x'-r(\tau)) \d \tau
   \\
   A^{\mu}(x)&=\frac{4\pi}{c} \int D^{(+)}(x,x') J^{\mu}(x') \d \tau
   \\
   &=2ec \int u^{\mu}(\tau) \theta(x^0-r^0(\tau)) \delta((x-r(\tau))^2) \d \tau
\end{align*}
$\delta$函数只在$\tau_0$处有贡献,$(x-r(\tau_0))^2=0,x^0-r^0(\tau_0)=|\bm{R}|$
\begin{gather*}
   \delta((x-r(\tau))^2)=\frac{\delta(\tau-\tau_0)}{\left.\dt{}{\tau}(x-r(\tau))^2\right|_{\tau=\tau_0}}=\frac{\delta(\tau-\tau_0)}{\left.2(x-r(\tau))\dt{}{\tau}(x-r(\tau))\right|_{\tau=\tau_0}}
   \\
   A^{\mu}(x)=\left.\frac{eu^{\mu}(\tau)}{u \cdot (x-r(\tau))}\right|_{\tau=\tau_0}
\end{gather*}
改写成三维形式
\begin{gather*}
   \Phi(\bm{x},t)=\frac{e}{(1-\bm{\beta} \cdot \bm{n})R}
   \\
   \bm{A}(\bm{x},t)=\frac{e\bm{\beta}}{(1-\bm{\beta} \cdot \bm{n})R}
\end{gather*}

电磁场
\begin{align*}
   \partial^{\mu} \delta((x-r(\tau))^2)&=-\frac{(x-r)^{\mu}}{u \cdot (x-r)}\dt{}{\tau}\delta((x-r(\tau))^2)
   \\
   \partial^{\nu} A^{\mu}&=2ec \int u^{\mu}(\tau) \theta(x^0-r^0(\tau)) \partial^{\nu} \delta((x-r(\tau))^2) \d \tau
   \\
   &=2ec \int \delta((x-r(\tau))^2) \dt{}{\tau}[u^{\mu}(\tau) \frac{(x-r)^{\nu}}{u \cdot (x-r)}] \d \tau
   \\
   &=2ec \left. \dt{}{\tau}[u^{\mu}(\tau) \frac{(x-r)^{\nu}}{u \cdot (x-r)}] \right|_{\tau=\tau_0}
   \\
   F^{\mu\nu}&=\left. 2ec \dt{}{\tau} [\frac{(x-r)^{\mu}u^{\nu}-(x-r)^{\nu}u^{\mu}}{u \cdot (x-r)}] \right|_{\tau=\tau_0}
\end{align*}
三维形式
\begin{gather*}
   \bm{E}=\frac{e(\bm{n}-\bm{\beta})}{\gamma^2 (1-\bm{\beta} \cdot \bm{n})^3 R^2}+\frac{e}{c} \frac{\bm{n} \times [(\bm{n}-\bm{\beta}) \times \dot{\bm{\beta}}]}{(1-\bm{\beta} \cdot \bm{n})^3 R}
   \\
   \bm{B}=\bm{n} \times \bm{E}
\end{gather*}

\section{Larmor公式与Thomson散射}

Larmor公式:在非相对论极限$\beta \ll 1$下,辐射功率
\begin{gather*}
   \dt{P}{\Omega}=\frac{cR^2}{4\pi} |\bm{E} \times \bm{H}|=\frac{e^2}{4\pi c^3} |\dot{\bm{v}}|^2 \sin^2\Theta
   \\
   P=\frac{2}{3} \frac{e^2}{c^3} |\dot{\bm{v}}|^2
\end{gather*}

Liénard公式:推广到相对论情况,$P$是Lorentz不变量,
\begin{align*}
   P&=-\frac{2}{3} \frac{e^2}{m^2 c^3} \dt{p^{\mu}}{\tau} \dt{p_{\mu}}{\tau}
   \\
   &=\frac{2}{3} \frac{e^2}{m^2 c^3} (\dt{\bm{p}}{\tau} \cdot \dt{\bm{p}}{\tau}-\frac{1}{c^2} (\dt{E}{\tau})^2)
   \\
   &=\frac{2}{3} \frac{e^2}{c} [\dot{\bm{\beta}}^2-(\bm{\beta} \times \dot{\bm{\beta}})^2]
\end{align*}

$t$时刻观测的辐射,在$t'=t-\frac{R}{c}$时刻发出,
\begin{align*}
   \d t&=\d t'(1-\bm{\beta} \cdot \bm{n})
   \\
   \dt{P(t')}{\Omega}&=R^2 \bm{S} \cdot \bm{n} \dt{t}{t'}
   \\
   &=\frac{e^2}{4\pi c} \frac{|\bm{n}\times(\bm{n}-\bm{\beta})\times \dot{\bm{\beta}}|^2}{(1-\bm{\beta} \cdot \bm{n})^5}
\end{align*}
\begin{enumerate}
   \item 沿$z$轴直线运动:
   \begin{equation*}
      \dt{P(t')}{\Omega}=\frac{e^2\dot{v}^2}{4\pi c^3} \frac{\sin^2\theta}{(1-\beta \cos\theta)^5}
   \end{equation*}
   \item 圆周运动:$\bm{\beta}$沿$z$轴,$\dot{\bm{\beta}}$沿$x$轴
   \begin{equation*}
      \dt{P(t')}{\Omega}=\frac{e^2}{4\pi c}\frac{\dot{v}^2}{(1-\beta \cos\theta)^3}[1-\frac{\sin^2 \theta \cos^2 \theta}{\gamma^2 (1-\beta \cos \theta)^2}]
   \end{equation*}
\end{enumerate}

Thomson散射:在低频情况下自由电子对电磁波的散射\\
入射波
\begin{gather*}
   \bm{E}=E_0 \bm{e}_0 \e^{\i \bm{k}_0 \cdot \bm{x}-\omega t}
   \\
   \bm{e}_0=\cos \phi_0 \hat{x}+\sin \phi_0 \hat{y}
   \\
   \bm{k}_0=k_0 \hat{z}
\end{gather*}
偏振为$\bm{e}$的散射波
\begin{gather*}
   \left<\dt{P}{\Omega}\right>=\frac{cR^2}{8\pi}|\bm{E} \times \bm{H}^*|=\frac{c}{8\pi} |E_0|^2 (\frac{e^2}{mc^2})^2 |\bm{e}^* \cdot \bm{e}_0|^2
   \\
   \intertext{能流}
   I=\frac{c}{8\pi} |E_0|^2
   \\
   \intertext{微分散射截面}
   \dt{\sigma}{\Omega}=(\frac{e^2}{mc^2})^2 |\bm{e}^* \cdot \bm{e}_0|^2
   \\
   \intertext{对$\bm{e}_{\theta}$和$\bm{e}_{\phi}$两个偏振方向取平均后相加}
   \dt{\sigma}{\Omega}=(\frac{e^2}{mc^2})^2 \frac{1+\cos^2 \theta}{2}
   \\
   \sigma=\frac{8\pi}{3} (\frac{e^2}{mc^2})^2
\end{gather*}

\section{频谱分析}

频谱:带电粒子辐射电磁波的能量按角频率的分布

\subsection{连续谱}

连续谱:非周期性运动粒子

\begin{align*}
   \dt{P(t)}{\Omega}&=\frac{e^2}{4\pi c} |\frac{\bm{n}\times(\bm{n}-\bm{\beta})\times \dot{\bm{\beta}}}{(1-\bm{\beta} \cdot \bm{n})^3}|^2=|\bm{A}(t)|^2
   \\
   \dt{W}{\Omega}&=\int_{-\infty}^{+\infty} |\bm{A}(t)|^2 \d t
   \\
   &=\int_{-\infty}^{+\infty} |\bm{A}(\omega)|^2 \d \omega
   \\
   &=\int_{0}^{+\infty} \dt{^2 I}{\omega \d \Omega} \d \omega
   \\
   \dt{^2 I}{\omega \d \Omega}&=2|\bm{A}(\omega)|^2
   \\
   \bm{A}(\omega)&=\sqrt{\frac{e^2}{8\pi^2 c}} \int_{-\infty}^{+\infty} \e^{\i \omega t}\frac{\bm{n}\times(\bm{n}-\bm{\beta})\times \dot{\bm{\beta}}}{(1-\bm{\beta} \cdot \bm{n})^3} \d t
   \\
   &=\sqrt{\frac{e^2}{8\pi^2 c}} \int_{-\infty}^{+\infty} \e^{\i \omega (t'+\frac{R(t')}{c})}\frac{\bm{n}\times(\bm{n}-\bm{\beta})\times \dot{\bm{\beta}}}{(1-\bm{\beta} \cdot \bm{n})^2} \d t'
\end{align*}
假定辐射粒子的运动在原点附近,$R(t') \approx |\bm{x}|-\bm{n} \cdot \bm{r}(t')$
\begin{align*}
   \bm{A}(\omega)&=\sqrt{\frac{e^2}{8\pi^2 c}} \e^{\i \omega \frac{|\bm{x}|}{c}} \int_{-\infty}^{+\infty} \e^{\i \omega (t'-\frac{\bm{n} \cdot \bm{r}(t')}{c})}\frac{\bm{n}\times(\bm{n}-\bm{\beta}(t'))\times \dot{\bm{\beta}}(t')}{(1-\bm{\beta}(t') \cdot \bm{n})^2} \d t'
   \\
   \dt{^2 I}{\omega \d \Omega}&=\frac{e^2}{4\pi^2 c} |\int_{-\infty}^{+\infty} \e^{\i \omega (t'-\frac{\bm{n} \cdot \bm{r}(t')}{c})}\frac{\bm{n}\times(\bm{n}-\bm{\beta}(t'))\times \dot{\bm{\beta}}(t')}{(1-\bm{\beta}(t') \cdot \bm{n})^2} \d t'|^2
   \\
   &=\frac{e^2 \omega^2}{4\pi^2 c} |\int_{-\infty}^{+\infty} \e^{\i \omega (t'-\frac{\bm{n} \cdot \bm{r}(t')}{c})} [\bm{n} \times (\bm{n} \times \bm{\beta}(t))]\d t'|^2
\end{align*}
\nocite{*}

\subsection{分立谱}

分立谱:周期性运动粒子

\begin{align*}
   \dt{P}{\Omega}&=\displaystyle \sum_{n=-\infty}^{+\infty} |\bm{A}_n|^2 = |\bm{A}_0|^2+2\displaystyle \sum_{n=1}^{+\infty} |\bm{A}_n|^2
\end{align*}
$\omega_n=n\omega_0$平均功率角分布
\begin{align*}
   \dt{P_n}{\Omega}&=2|\bm{A}_n|^2
   \\
   &=\frac{e^2 n^2 \omega_0^2}{2\pi cT^2} |\int_{-\frac{T}{2}}^{+\frac{T}{2}} \e^{\i \omega (t'-\frac{\bm{n} \cdot \bm{r}(t')}{c})} [\bm{n} \times (\bm{n} \times \bm{\beta}(t'))]\d t'|^2
\end{align*}

同步辐射:相对论性带电粒子周期性圆周运动的辐射\\
带电粒子圆周运动圆频率为$\omega_0$,半径为$a$,在$xOy$平面内做匀速圆周运动,取圆心为原点
\begin{gather*}
   \bm{r}(t)=a(\cos\omega_0 t,\sin\omega_0 t,0)
   \\
   \bm{\beta}(t)=\beta(-\sin\omega_0 t,\cos\omega_0 t,0)
   \\
   \beta=\frac{\omega_0 a}{c}
   \\
   \bm{n}=(\sin\theta,0,\cos\theta)
   \\
   \intertext{偏振方向}
   \bm{\epsilon}_{\parallel}=-\cos\theta \hat{x}+\sin\theta \hat{z}
   \\
   \bm{\epsilon}_{\perp}=-\hat{y}
\end{gather*}
Schott公式:频谱角分布
\begin{align*}
   \dt{P_n}{\Omega}&=\frac{e^2 n^2 \omega_0^2}{2\pi c} |\frac{1}{2\pi} \int_{-\pi}^{+\pi} \e^{\i n (\phi-\beta \sin\theta \cos\phi)} [\bm{n} \times (\bm{n} \times \bm{\beta})]\d \phi|^2
   \\
   &=\frac{e^2 n^2 \omega_0^2}{2\pi c} |\cot\theta J_n(n\beta \sin\theta)\bm{\epsilon_{\parallel}}+\i \beta J'_n(n\beta \cos\theta)\bm{\epsilon}_{\perp}|^2
   \\
   &=\frac{e^2 n^2 \omega_0^2}{2\pi c} [\cot^2\theta J_n^2(n\beta \sin\theta)+\beta^2 J^{'2}_n(n\beta \cos\theta)]
\end{align*}
积分得到
\begin{equation*}
   P_n=\frac{2e^2\omega_0}{v}[n\beta^2 J'_{2n}(2n\beta)-\frac{n^2}{\gamma^2}\int_{0}^{\beta} J_{2n}(2n\xi)\d \xi]
\end{equation*}
对于极端相对论情形,$n\gg 1$起主要作用,
\begin{equation*}
   P_n=-\frac{2e^2 \omega_0^2 n^{\frac{1}{3}}}{\sqrt{\pi}c}[\Phi'(u)+\frac{u}{2}\int_{u}^{+\infty} \Phi(u) \d u]
\end{equation*}
其中Airy函数$\Phi(x)=\frac{1}{\sqrt{\pi}} \int_{0}^{+\infty} \cos(\frac{t^3}{3}-xt) \d t$,$u=n^{\frac{2}{3}}\gamma^{-2}$
\begin{enumerate}
   \item $1\ll n \ll \gamma$
   \begin{equation*}
      P_n \approx 0.52 \frac{e^2 \omega_0^2}{c} n^{\frac{1}{3}}
   \end{equation*}
   \item $n \ll \gamma$
   \begin{equation*}
      P_n=\frac{e^2 \omega_0^2}{2\sqrt{\pi}c} \sqrt{\frac{n}{\gamma}} e^{-\frac{2}{3} n \gamma^{-3}}
   \end{equation*}
\end{enumerate}

\section{Cherenkov辐射}

高速带电粒子穿过介质

波动方程
\begin{gather*}
   \nabla^2 \Phi -\frac{\epsilon}{c^2} \frac{\partial^2 \Phi}{\partial t^2} = -\frac{4\pi}{\epsilon} \rho
   \\
   \nabla^2 \bm{A} -\frac{\epsilon}{c^2} \frac{\partial^2 \bm{A}}{\partial t^2} = -\frac{4\pi}{\epsilon} \bm{J}
\end{gather*}
Fourier变换得到
\begin{gather*}
   \bm{J}(\bm{k},\omega)=2\pi e \bm{v} \delta(\omega-\bm{k} \cdot \bm{v})
   \\
   \bm{A}(\bm{k},\omega)=\frac{8\pi^2 e \bm{\beta}}{\bm{k}^2-\frac{\omega^2}{c^2} \epsilon(\omega)} \delta(\omega-\bm{k} \cdot \bm{v})
   \\
   \intertext{假设速度沿$z$方向}
   \bm{A}(\bm{x},t)=4\pi e \bm{\beta} \int \frac{\e^{\i k_3(x_3-vt)} \e^{\i \bm{k}_{\perp} \cdot \bm{x}_{\perp}}}{k_3^2(1-\beta^2 \epsilon)+\bm{k}_{\perp}^2} \frac{\d^3 \bm{k}}{(2\pi)^3}
\end{gather*}
近似将$\epsilon$视为常数,当$\beta>\frac{1}{\sqrt{\epsilon}}$时,类似于推迟Green函数,在$x_3<vt$积分才不为$0$,则以粒子为顶点,电磁势不为$0$的点构成一个锥体,半锥角
\begin{equation*}
   \theta_C=\arccos\frac{c}{v\sqrt{\epsilon}}
\end{equation*}
矢势
\begin{align*}
   \bm{A}(\bm{x},t)&=\frac{4\pi e \bm{\beta}}{1-\beta^2\epsilon} \iiint \frac{\e^{\i k_3(x_3-vt)} \e^{\i \bm{k}_{\perp} \cdot \bm{x}_{\perp}}}{k_3^2-(\frac{\bm{k}_{\perp}}{\sqrt{\eta^2\epsilon-1}})^2} \frac{\d^3 \bm{k}}{(2\pi)^3}
   \\
   &=\frac{4\pi e \bm{\beta}}{1-\beta^2\epsilon} \iint \frac{\d k_1 \d k_2}{(2\pi)^2} \e^{\i \bm{k}_{\perp} \cdot \bm{x}_{\perp}} \i \frac{\e^{\i k_{\perp} \frac{x_3-vt}{\sqrt{\beta^2\epsilon-1}}}-\e^{-\i k_{\perp} \frac{x_3-vt}{\sqrt{\beta^2\epsilon-1}}}}{2\frac{k_{\perp}}{\sqrt{\beta^2\epsilon-1}}}
   \\
   &=\frac{e \bm{\beta}}{\pi \sqrt{\beta^2\epsilon-1}} \int_{0}^{+\infty} \int_{0}^{2\pi} \e^{\i k |\bm{x}_{\perp}| \cos\theta} \sin(k \frac{x_3-vt}{\sqrt{\beta^2\epsilon-1}}) \d k \d \theta
   \\
   &=\frac{2e \bm{\beta}}{\sqrt{\beta^2\epsilon-1}} \int_{0}^{+\infty} J_0(k|\bm{x}_{\perp}|) \sin(k \frac{x_3-vt}{\sqrt{\beta^2\epsilon-1}}) \d k
   \\
   &=\frac{2e \bm{\beta}}{\sqrt{(x_3-vt)^2-(\beta^2 \epsilon-1)\bm{x}_{\perp}^2}}
\end{align*}

\section{辐射阻尼}

辐射阻尼:带电粒子自身辐射的电磁场对自身运动的影响(经典电动力学不能完美解决)

Abraham-Lorentz方程:能量角度的反作用
\begin{gather*}
   m\dot{\bm{v}}=\bm{F}_{ext}+\bm{F}_{rad}
   \\
   \int_{t_1}^{t_2} \bm{F}_{rad} \cdot \bm{v} \d t=-\int_{t_1}^{t_2} \frac{2}{3} \frac{e^2}{c^3} \dot{\bm{v}} \cdot \dot{\bm{v}} \d t
   \\
   \intertext{分部积分得到}
   \bm{F}_{rad}=\frac{2}{3} \frac{e^2}{c^3} \ddot{\bm{v}}=m\tau \ddot{\bm{v}}
   \\
   \intertext{得到Abraham-Lorentz方程}
   m(\dot{\bm{v}}-\tau \ddot{\bm{v}})=\bm{F}_{ext}
\end{gather*}
该方程存在发散解,对于发散解,分部积分的边界项不可以舍去

Abraham-Lorentz模型:动量角度的反作用
\begin{gather*}
   \intertext{系统动量守恒}
   \int (\rho \bm{E}+\frac{1}{c} \bm{J} \times \bm{B}) \d V=0
   \\
   \bm{E}=\bm{E}_{ext}+\bm{E}_{self},\quad \bm{B}=\bm{B}_{ext}+\bm{B}_{self}
   \\
   \dt{\bm{p}}{t}=\bm{F}_{ext}=-\int (\rho \bm{E}_{self}+\frac{1}{c} \bm{J} \times \bm{B}_{self}) \d V
   \\
   \intertext{选定合适参考系使电流为0}
   \dt{\bm{p}}{t}=\int \d V \rho(\bm{x},t) [\nabla \Phi(\bm{x},t)+\frac{1}{c} \pt{\bm{A}(\bm{x},t)}{t}]
   \\
   \intertext{Taylor展开}
   \dt{\bm{p}}{t}=\displaystyle \sum_{n=0}^{+\infty} \frac{(-1)^n}{n!c^n} \int \d V \int \d V' \rho(\bm{x},t) \pt{^n}{t^n}[\rho(\bm{x'},t)\nabla R^{n-1} + \frac{R^{n-1}}{c^2} \pt{\bm{J}(\bm{x'})}{t}]
   \\
   \dt{\bm{p}}{t}=\displaystyle \sum_{n=0}^{+\infty} \frac{(-1)^n}{c^{n+2}} \frac{2}{3n!} \pt{^{n+1}\bm{v}}{t^{n+1}} \int \d V \int \d V' \rho(\bm{x},t)\rho(\bm{x'},t)R^{n-1}
   \\
   \intertext{$n=0$项}
   \left(\dt{\bm{p}}{t}\right)_{n=0}=\frac{2}{3c^2} \dot{\bm{v}} \int \d V \int \d V' \frac{\rho(\bm{x})\rho(\bm{x'})}{R}=\frac{4U_{em}}{3c^2} \dot{\bm{v}}
   \\
   \intertext{静电能}
   U_{em}=\frac{1}{2} \int \d V \int \d V' \frac{\rho(\bm{x})\rho(\bm{x'})}{R}
   \\
   \intertext{电磁质量}
   m_{em}=\frac{U_{em}}{c^2}
   \\
   \intertext{$n=1$项}
   \left(\dt{\bm{p}}{t}\right)_{n=1}=-\frac{2e^2}{3c^3} \ddot{\bm{v}}
   \\
   \intertext{更高阶项在带电粒子尺度$a\to 0$时会趋于0}
   \frac{4}{3} m_{em} \dot{\bm{v}} - \frac{2e^2}{3c^3} \ddot{\bm{v}} =\bm{F}_{ext}
\end{gather*}
该模型的缺陷在于,$a \to 0$时静电能会发散,要求带电粒子有非零尺度(电子的经典半径)$r_0=\frac{e^2}{mc^2}$. 在$r_0$尺度之下,需要运用量子电动力学,电子自能按$a$的对数发散,需要重整化才能解决

\printbibliography[heading=bibintoc, title=\ebibname]
\appendix

\chapter{特殊函数}

\section{球函数}

Legendre方程:
\begin{equation*}
   \dt{}{x}((1-x^2)\dt{y}{x})+\nu(\nu+1)y=0
\end{equation*}

复方程的解为$\nu$次Legendre函数
\begin{gather*}
   P_{\nu}(z)=\displaystyle \sum_{n=0}^{+\infty} \frac{1}{(n!)^2} \frac{\Gamma(\nu+n+1)}{\Gamma(\nu-n+1)}(\frac{z-1}{2})^n \\
   Q_{\nu}(z)=
   \begin{cases}
      \frac{\pi}{2} \frac{e^{-\i \nu \pi}P_{\nu}(z)-P_{\nu}(-z)}{\sin \nu \pi},\quad \Im z>0 \\
      \frac{\pi}{2} \frac{e^{\i \nu \pi}P_{\nu}(z)-P_{\nu}(-z)}{\sin \nu \pi},\quad \Im z<0
   \end{cases}
\end{gather*}

在$[-1,1]$上的解:
\begin{itemize}
   \item 在$x=1$处,$P_{\nu}(x)$有界,$Q_{\nu}(x)$对数发散,解中不含$Q_{\nu}(x)$
   \item 在$x \to -1$时,\\
   \begin{align*}
      P_{\nu}(x)&=\displaystyle \sum_{n=0}^{+\infty} \frac{1}{(n!)^2} \frac{\Gamma(\nu+n+1)}{\Gamma(\nu-n+1)}(\frac{x-1}{2})^n \\
      &=-\frac{\sin \nu \pi}{\pi} \displaystyle \sum_{n=0}^{+\infty} \frac{\Gamma(\nu+n+1) \Gamma(n-\nu)}{(n!)^2} (-1)^n(\frac{x-1}{2})^n \\
      &\sim 
      \frac{\sin \nu \pi}{\pi} \ln \frac{2}{1+x}
   \end{align*}
   对数发散,需要截断为多项式,$\nu$为自然数$l$,得到$l$次Legendre多项式
   \begin{equation*}
      P_l(x)=\displaystyle \sum_{n=0}^{l} \frac{1}{(n!)^2} \frac{(l+n)!}{(l-n)!}(\frac{x-1}{2})^n
   \end{equation*}
\end{itemize}

连带Legendre方程:
\begin{equation*}
   \dt{}{x}((1-x^2)\dt{y}{x})+(\nu(\nu+1)-\frac{m^2}{1-x^2})y=0
\end{equation*}

\begin{enumerate}

\item 复方程的解为
\begin{gather*}
   P_{\nu}^m(z)=(z^2-1)^{\frac{m}{2}} \dt{^m P_{\nu}(z)}{z^m} \\
   Q_{\nu}^m(z)=(z^2-1)^{\frac{m}{2}} \dt{^m Q_{\nu}(z)}{z^m}
\end{gather*}

\item 实轴上的解定义为
\begin{align*}
   P_{\nu}^m(x)&=\i^m P_{\nu}^m(x+0\i)=\i^{-m} P_{\nu}^m(x-0\i) \\
   &=(-)^m (1-x^2)^{\frac{m}{2}} \dt{^m P_{\nu}(x)}{z^m} \\
   Q_{\nu}^m(x)&=\frac{(-)^m}{x}(\i^{-m}Q_{\nu}^m(x+0 \i)+\i^mQ_{\nu}^m(x-0\i)) \\
   &=(-)^m (1-x^2)^{\frac{m}{2}} \dt{^m Q_{\nu}(x)}{z^m}
\end{align*}

\item 在$[-1,1]$上的解:
\begin{itemize}
   \item 在$x=1$处,$P_{\nu}^m(x)$有界,$Q_{\nu}^m(x)$对数发散,解中不含$Q_{\nu}^m(x)$
   \item 在$x=-1$处,$P_{\nu}^m(x)$发散,要求$P_{\nu}(x)$截断为多项式,$\nu$为自然数且$\nu \geq m$
\end{itemize}
本征函数为$m$阶$l$次连带Legendre多项式:
\begin{gather*}
   P_l^m(x)=(-)^m(1-x^2)^{\frac{m}{2}} \dt{^m P_l(x)}{x^m} \\
   P_l^{-m}(x)=(-)^m \frac{(l-m)!}{(l+m)!}P_l^m(x)
\end{gather*}

\end{enumerate}

球谐函数
\begin{gather*}
   Y_{l,m}(\bm{n})=\sqrt{\frac{2l+1}{4\pi} \frac{(l-m)!}{(l+m)!}} P_l^m(\cos \theta)e^{\i m \phi} \\
   Y_{l,-m}(\bm{n})=(-)^m Y_{l,m}^*(\bm{n})
\end{gather*}

\subsection{Legendre多项式的性质}

\begin{itemize}
   \item 微分形式:Rodrigues公式\\
   \begin{equation*}
      P_l(x)=\frac{1}{2^l l!} \dt{^l}{x^l} (x^2-1)^l
   \end{equation*}
   \item 正交完备性 \\
   \begin{gather*}
      \int_{-1}^{1} P_l(x)P_{l'}(x) \d x=\frac{2}{2l+1} \delta_{ll'} \\
      \int_{-1}^{1} P_l^m(x)P_{l'}^{m}(x) \d x=\frac{2}{2l+1} \frac{(l+m)!}{(l-m)!} \delta_{ll'} \\
      \int Y_{lm}^*(\bm{n})Y_{l'm'}(\bm{n}) \d \Omega = \delta_{ll'} \delta_{mm'} \\
      \displaystyle \sum_{l=0}^{+\infty} \sum_{m=-l}^{l} Y_{lm}^*(\bm{n'})Y_{lm}(\bm{n})=\delta(\cos \theta'-\cos \theta)\delta(\phi'-\phi)
   \end{gather*}
   \item 生成函数 \\
   \begin{equation*}
      \frac{1}{\sqrt{1-2xt+t^2}}=\displaystyle \sum_{l=0}^{+\infty} P_l(x)t^l,\quad |t|<\min |x\pm \sqrt{x^2-1}|
   \end{equation*}
   \item 递推关系 \\
   \begin{equation*}
      \begin{cases}
         P'_{l+1}(x)=xP'_l(x)+(l+1)P_l(x) \\
         P'_{l-1}(x)=xP'_l(x)-lP_l(x)
      \end{cases}
   \end{equation*}
\end{itemize}

\section{柱函数}

Bessel方程
\begin{equation*}
   \frac{1}{x} \dt{}{x^2}(x\dt{y}{x}) + (1-\frac{\nu^2}{x^2})y=0
\end{equation*}

Bessel函数
\begin{equation*}
   J_{\nu}(x)=\displaystyle \sum_{k=0}^{+\infty} \frac{(-1)^k}{k!\Gamma(k+\nu+1)}(\frac{x}{2})^{2k+\nu}
\end{equation*}

Neumann函数
\begin{equation*}
   N_{\nu}(x)=\frac{\cos{\nu \pi}J_{\nu}(x)-J_{-\nu}(x)}{\sin{\nu \pi}}
\end{equation*}

   (1)当$\nu \notin \mathbb{Z}$时,Bessel方程的解为$J_{\nu}(x)$和$J_{-\nu}(x)$的线性组合

   (2)当$\nu =m\in \mathbb{Z}$时,$J_{-m}(x)=(-1)^m J_{m}(x)$,Bessel方程的解为$J_{m}(x)$与$N_{m}(x)$的线性组合

Hankel函数
\begin{gather*}
   H_{\nu}^{(1)}(x)=J_{\nu}(x)+iN_{\nu}(x) \\
   H_{\nu}^{(2)}(x)=J_{\nu}(x)-iN_{\nu}(x)
\end{gather*}

虚宗量Bessel函数
\begin{gather*}
   I_{\nu}(x)=i^{-\nu}J_{\nu}(ix) \\
   K_{\nu}(x)=\frac{\pi}{2\sin{\nu \pi}}(I_{-\nu}(x)-I_{\nu}(x))
\end{gather*}

球Bessel方程
\begin{gather*}
   \frac{1}{x^2} \dt{}{x}(x^2\dt{y}{x})+(1-\frac{l(l+1)}{x^2})y=0
\end{gather*}

解为球Bessel函数
\begin{gather*}
   j_l(x)=\sqrt{\frac{\pi}{2x}}J_{l+\frac{1}{2}}(x) \\
   n_l(x)=(-)^{l+1}j_{-l-1}(x)=\sqrt{\frac{\pi}{2x}}N_{l+\frac{1}{2}}(x)
\end{gather*}

球Hankel函数
\begin{gather*}
   h_l^{(1)}(x)=j_l(x)+\i n_l(x) \\
   h_l^{(2)}(x)=j_l(x)-\i n_l(x)
\end{gather*}

\subsection{递推关系}

上述函数统称为柱函数,均满足:
\begin{gather*}
   \dt{}{x}(x^{\nu}C_{\nu}(x))=x^{\nu}C_{\nu-1}(x) \\
   \dt{}{x}(x^{-\nu}C_{\nu}(x))=-x^{-\nu}C_{\nu+1}(x)
\end{gather*}

\subsection{渐进展开}

$x\to 0$:
\begin{gather*}
   J_{\nu}(x) \sim \frac{1}{\Gamma(\nu+1)}(\frac{x}{2})^{\nu} \\
   N_{\nu}(x) \sim -\frac{\Gamma(\nu)}{\pi}(\frac{x}{2})^{-\nu},\quad N_0(x) \sim \frac{2}{\pi} \ln{\frac{x}{2}}
\end{gather*}

$x\to \infty,\:|arg(x)|<\pi$:
\begin{gather*}
   J_{\nu}(x) \sim \sqrt{\frac{2}{\pi x}} \cos(x-\frac{\nu \pi}{2}-\frac{\pi}{4}) \\
   N_{\nu}(x) \sim \sqrt{\frac{2}{\pi x}} \sin(x-\frac{\nu \pi}{2}-\frac{\pi}{4}) \\
   H_{\nu}^{(1)}(x) \sim \sqrt{\frac{2}{\pi x}} e^{\i(x-\frac{\nu \pi}{2}-\frac{\pi}{4})} \\
   H_{\nu}^{(2)}(x) \sim \sqrt{\frac{2}{\pi x}} e^{-\i(x-\frac{\nu \pi}{2}-\frac{\pi}{4})} \\
   j_l(x) \sim \frac{1}{x} \sin(x-\frac{l\pi}{2}) \\
   n_l(x) \sim -\frac{1}{x} \cos(x-\frac{l\pi}{2})
\end{gather*}

$|x| \to \infty$:
\begin{gather*}
   I_{\nu}(x) \sim \sqrt{\frac{1}{2\pi x}}e^x \\
   K_{\nu}(x) \sim \sqrt{\frac{\pi}{2x}}e^{-x}
\end{gather*}

\subsection{整数阶的生成函数与积分表示}

生成函数:
\begin{equation*}
   e^{\frac{x}{2}(t-\frac{1}{t})}= \displaystyle \sum_{n=-\infty}^{+\infty}J_n(x)t^n,\qquad 0<|t|<\infty \\
\end{equation*}

积分表示:
\begin{equation*}
   J_n(x)=\frac{1}{\pi} \int_{-\pi}^{\pi} \cos(x\sin\theta-n\theta) \d \theta
\end{equation*}

\section{常用展开式}

下取$r=|\bm{x}|, r'=|\bm{x'}|,r_>=\max\{r,r'\},r_<=\min\{r,r'\}$,$\gamma$为$\bm{x}$与$\bm{k}$夹角

加法定理:
\begin{gather*}
   \frac{1}{|\bm{x}-\bm{x'}|}=\displaystyle \sum_{l=0}^{+\infty} \sum_{m=-l}^{l} \frac{4\pi}{2l+1} \frac{r_<^l}{r_>^{l+1}} Y_{lm}^*(\bm{n'})Y_{lm}(\bm{n})
\end{gather*}

平面波按柱面波展开:生成函数中取$t=ie^{\i \theta}$,
\begin{equation*}
   e^{\i kr \cos \theta} = J_0(kr)+2 \displaystyle \sum_{n=1}^{+\infty} \i^n \cos n\theta J_n(kr)
\end{equation*}

平面波按球面波展开:
\begin{gather*}
   e^{\i \bm{k}\cdot \bm{x}}=\displaystyle 4\pi \sum_{l=0}^{+\infty} \sum_{m=-l}^{l} \i^l j_l(kr) Y_{lm}^*(\hat{\bm{n}})Y_{lm}(\hat{\bm{k}})
\end{gather*}
\end{document}
